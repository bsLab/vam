\documentclass{report}
\usepackage{newcent}
\usepackage{pifont}
\usepackage[dvips]{color}

\parindent 0pt
\pagestyle{plain}
\topmargin0cm
\headheight0pt
\headsep0pt
\textwidth16.5cm
\textheight23cm
\footskip1cm
\oddsidemargin0cm
\evensidemargin0cm

\renewcommand{\labelitemi}{\textcolor{blue}{\ding{228}}}
\renewcommand{\labelitemii}{\textcolor{blue}{\ding{169}}}
\renewcommand{\labelitemiii}{\textcolor{blue}{\ding{72}}}
\renewcommand{\arraystretch}{1.1}
\begin{document}
\begin{center}
\vbox{\hrule\vskip2pt\hrule\bf\large\vskip \baselineskip
{\sl VaMSyS }\\
Version 1.2 \\
The Virtual Amoeba Machine Environment \\
Written by \\
Stefan Bosse (sbosse@physik.uni$-$bremen.de) \\
\vskip \baselineskip
\hrule\vskip2pt\hrule}
\end{center}
\vfill \eject
\vfill \eject
\vbox{\hsize \textwidth
  \hrule
  \vbox{\rule[-.55em]{0pt}{1.7em}\strut \vrule \quad \hfill
    \label{VAMAmoebaSystem}\textcolor{blue}{\sc Package: {\sl VAM Amoeba System 
    }} \hfill
    \quad \vrule}\hrule}
  
\vskip \baselineskip
\vskip .5\baselineskip 
{\parindent 0pt \vbox{\sc Description}
\leftskip 20pt \rightskip 20pt \vskip .5\baselineskip\begin{list}{}{}\item{
\parindent 0pt 
  This package contains the Amoeba System Environment, entirely written in 
  OCaML. Both, the low level Amoeba base system, and higher user interfaces 
  like name or directory services are provided. }\end{list}}
\vskip .5\baselineskip
\vfill \eject
\vbox{\hsize \textwidth
  \hrule
  \vbox{\rule[-.55em]{0pt}{1.7em}\strut \vrule \quad 
    \label{Amoeba}\textcolor{blue}{\sc ML-Module: {\sl Amoeba }} \hfill 
    \textcolor[rgb]{.0,.7,.0}{\sc Package:  VAM Amoeba System}
    \quad \vrule}\hrule}
  
\vskip \baselineskip
\vskip .5\baselineskip 
{\parindent 0pt \vbox{\sc Description}
\leftskip 20pt \rightskip 20pt \vskip .5\baselineskip\begin{list}{}{}\item{
\parindent 0pt 
  This is the base Amobea module containing the following parts: 
  \vskip \baselineskip
  \begin{minipage}{.9\textwidth}\vbox{
    \begin{itemize}
      \item Basic types 
      \item Capability and Port functions 
      \item Buffer management 
      \item RPC header strcuture 
      \item Encryption 
      \end{itemize}
    }\end{minipage}
  \vskip \baselineskip
  }\end{list}}
\vskip .5\baselineskip
\vskip .5\baselineskip 
{\parindent 0pt \vbox{\sc \label{Basictypes}Basic types }
\leftskip 20pt \rightskip 20pt \vskip .5\baselineskip\begin{list}{}{}\item{
\parindent 0pt 
  The following types are used in capabilities and headers. 
  \vskip 2\baselineskip
  \vbox{\hfil
    \vbox{\hrule
      \hbox{\vrule\qquad
        \textcolor{red}{\rule[-.55em]{0pt}{1.7em}\sc Programming Interface}
        \quad\vrule}
      \hrule}
    }
  \nobreak\vskip \baselineskip\nobreak
  {
    \nobreak\vskip .5\baselineskip 
    \halign{
      
      \textcolor{red}{\strut\vrule width 5pt\hskip 20pt \hfill #} & # \hfil & # \hfil & #\hfill & \qquad # \hfill \cr
      &&&&\cr
      \textcolor{red}{type} & {\bf rights$\_$bits } $=$  & 
      &Rights$\_$bits of int  & \cr
      }
    \halign{
      
      \textcolor{red}{\strut\vrule width 5pt\hskip 20pt \hfill #} & # \hfil & # \hfil & #\hfill & \qquad # \hfill \cr
      &&&&\cr
      \textcolor{red}{type} & {\bf obj$\_$num } $=$  & 
      &Objnum of int  & \cr
      }
    \halign{
      
      \textcolor{red}{\strut\vrule width 5pt\hskip 20pt \hfill #} & # \hfil & # \hfil & #\hfill & \qquad # \hfill \cr
      &&&&\cr
      \textcolor{red}{type} & {\bf command } $=$  & 
      &Command of int  & \cr
      }
    \halign{
      
      \textcolor{red}{\strut\vrule width 5pt\hskip 20pt \hfill #} & # \hfil & # \hfil & #\hfill & \qquad # \hfill \cr
      &&&&\cr
      \textcolor{red}{type} & {\bf errstat } $=$  & 
      &Errstat of int  & \cr
      }
    \halign{
      
      \textcolor{red}{\strut\vrule width 5pt\hskip 20pt \hfill #} & # \hfil & # \hfil & #\hfill & \qquad # \hfill \cr
      &&&&\cr
      \textcolor{red}{type} & {\bf status } $=$  & 
      &Status of int  & \cr
      }
    \halign{
      
      \textcolor{red}{\strut\vrule width 5pt\hskip 20pt \hfill #} & # \hfil & # \hfil & #\hfill & \qquad # \hfill \cr
      &&&&\cr
      \textcolor{red}{type} & {\bf port } $=$  & 
      &Port of string  & \cr
      }
    \halign{
      
      \textcolor{red}{\strut\vrule width 5pt\hskip 20pt \hfill #} & # \hfil & # \hfil & #\hfill & \qquad # \hfill \cr
      &&&&\cr
      \textcolor{red}{type} & {\bf privat } $=$  $\{$  & 
      &\textcolor{red}{mutable }prv$\_$object : obj$\_$num $;$ & \cr
      && &\textcolor{red}{mutable }prv$\_$rights : rights$\_$bits $;$ & \cr
      && &\textcolor{red}{mutable }prv$\_$random : port  $\}$  & \cr
      }
    \halign{\textcolor{red}{\strut\vrule width 5pt#}&#\cr&\cr}
    }
  \vskip 2\baselineskip
  The main Amoeba structure is the capability. It's used to give an arbitrary 
  object in Amoeba a uniqe identifier. Objects can be: 
  \vskip \baselineskip
  \begin{minipage}{.9\textwidth}\vbox{
    \begin{itemize}
      \item Files 
      \item Directories 
      \item Processes 
      \item Memory segments 
      \item Server 
      \end{itemize}
    }\end{minipage}
  \vskip \baselineskip
  and many more. The RPC header structure is needed for communication between 
  clients and servers. 
  \vskip 2\baselineskip
  \vbox{\hfil
    \vbox{\hrule
      \hbox{\vrule\qquad
        \textcolor{red}{\rule[-.55em]{0pt}{1.7em}\sc Programming Interface}
        \quad\vrule}
      \hrule}
    }
  \nobreak\vskip \baselineskip\nobreak
  {
    \nobreak\vskip .5\baselineskip 
    \halign{
      
      \textcolor{red}{\strut\vrule width 5pt\hskip 20pt \hfill #} & # \hfil & # \hfil & #\hfill & \qquad # \hfill \cr
      &&&&\cr
      \textcolor{red}{type} & {\bf capability } $=$  $\{$  & 
      &\textcolor{red}{mutable }cap$\_$port : port $;$ & \cr
      && &\textcolor{red}{mutable }cap$\_$priv : privat  $\}$  & \cr
      }
    \halign{
      
      \textcolor{red}{\strut\vrule width 5pt\hskip 20pt \hfill #} & # \hfil & # \hfil & #\hfill & \qquad # \hfill \cr
      &&&&\cr
      \textcolor{red}{type} & {\bf header } $=$  $\{$  & 
      &\textcolor{red}{mutable }h$\_$port : port $;$ & \cr
      && &\textcolor{red}{mutable }h$\_$priv : privat $;$ & \cr
      && &\textcolor{red}{mutable }h$\_$command : command $;$ & \cr
      && &\textcolor{red}{mutable }h$\_$status : status $;$ & \cr
      && &\textcolor{red}{mutable }h$\_$offset : int $;$ & \cr
      && &\textcolor{red}{mutable }h$\_$size : int $;$ & \cr
      && &\textcolor{red}{mutable }h$\_$extra : int  $\}$  & \cr
      }
    \halign{\textcolor{red}{\strut\vrule width 5pt#}&#\cr&\cr}
    }
  \vskip 2\baselineskip
  }\end{list}}
\vskip .5\baselineskip
\vskip .5\baselineskip 
{\parindent 0pt \vbox{\sc \label{Basicfunctionsandvalues}Basic functions and 
values }
\leftskip 20pt \rightskip 20pt \vskip .5\baselineskip\begin{list}{}{}\item{
\parindent 0pt 
  Function to manage and manipulate ports, capabilities and headers: 
  \vskip 2\baselineskip
  \vbox{\hfil
    \vbox{\hrule
      \hbox{\vrule\qquad
        \textcolor{red}{\rule[-.55em]{0pt}{1.7em}\sc Programming Interface}
        \quad\vrule}
      \hrule}
    }
  \nobreak\vskip \baselineskip\nobreak
  {
    \nobreak\vskip .5\baselineskip 
    \halign{
      
      \textcolor{red}{\strut\vrule width 5pt\hskip 20pt \hfill #} & # \hfil & # & #\hfill & \qquad # \hfill \cr
      &&&& \cr
      $[$ &  port  & $]=$ & {\bf port$\_$new } & \cr
      & & & \hskip 10pt  unit  & \cr
      }
    \halign{
      
      \textcolor{red}{\strut\vrule width 5pt\hskip 20pt \hfill #} & # \hfil & # & #\hfill & \qquad # \hfill \cr
      &&&& \cr
      $[$ &  privat  & $]=$ & {\bf priv$\_$new } & \cr
      & & & \hskip 10pt  unit  & \cr
      }
    \halign{
      
      \textcolor{red}{\strut\vrule width 5pt\hskip 20pt \hfill #} & # \hfil & # & #\hfill & \qquad # \hfill \cr
      &&&& \cr
      $[$ &  capability  & $]=$ & {\bf cap$\_$new } & \cr
      & & & \hskip 10pt  unit  & \cr
      }
    \halign{
      
      \textcolor{red}{\strut\vrule width 5pt\hskip 20pt \hfill #} & # \hfil & # & #\hfill & \qquad # \hfill \cr
      &&&& \cr
      $[$ &  header  & $]=$ & {\bf header$\_$new } & \cr
      & & & \hskip 10pt  unit  & \cr
      }
    \halign{
      
      \textcolor{red}{\strut\vrule width 5pt\hskip 20pt \hfill #} & # \hfil & # & #\hfill & \qquad # \hfill \cr
      &&&& \cr
      $[$ &  bool  & $]=$ & {\bf port$\_$cmp } & \cr
      & & & \hskip 10pt {\it p1 }: port $\rightarrow$ & \cr
      & & & \hskip 10pt {\it p2 }: port  & \cr
      }
    \halign{
      
      \textcolor{red}{\strut\vrule width 5pt\hskip 20pt \hfill #} & # \hfil & # & #\hfill & \qquad # \hfill \cr
      &&&& \cr
      $[$ &  bool  & $]=$ & {\bf nullport } & \cr
      & & & \hskip 10pt  port  & \cr
      }
    \halign{
      
      \textcolor{red}{\strut\vrule width 5pt\hskip 20pt \hfill #} & # \hfil & # & #\hfill & \qquad # \hfill \cr
      &&&& \cr
      $[$ & {\it port$\_$value  }:  int  & $]=$ & {\bf get$\_$portbyte } & \cr
      & & & \hskip 10pt  $^\sim$port  :  port $\rightarrow$ & \cr
      & & & \hskip 10pt  $^\sim$byte  :  int  & \cr
      }
    \halign{
      
      \textcolor{red}{\strut\vrule width 5pt\hskip 20pt \hfill #} & # \hfil & # & #\hfill & \qquad # \hfill \cr
      &&&& \cr
      $[$ &  unit  & $]=$ & {\bf set$\_$portbyte } & \cr
      & & & \hskip 10pt  $^\sim$port  :  port $\rightarrow$ & \cr
      & & & \hskip 10pt  $^\sim$byte  :  int $\rightarrow$ & \cr
      & & & \hskip 10pt  $^\sim$value  :  int  & \cr
      }
    \halign{\textcolor{red}{\strut\vrule width 5pt#}&#\cr&\cr}
    }
  \vskip 2\baselineskip
  The {\sl XX$\_$new }functions return fresh values of the specified type, the 
  {\sl port$\_$cmp }function tests two port for equality, the {\sl null$\_$port 
  }function tests for a zero port (all bytes zero), the {\sl XX$\_$portbyte }
  are used to get and set single bytes of a port. \\
  For each basic structure there is a so called nil value $-$ a dummy for 
  references: 
  \vskip 2\baselineskip
  \vbox{\hfil
    \vbox{\hrule
      \hbox{\vrule\qquad
        \textcolor{red}{\rule[-.55em]{0pt}{1.7em}\sc Programming Interface}
        \quad\vrule}
      \hrule}
    }
  \nobreak\vskip \baselineskip\nobreak
  {
    \nobreak\vskip .5\baselineskip 
    \halign{
      
      \textcolor{red}{\strut\vrule width 5pt\hskip 20pt \hfill } # & # \hfil  & \qquad # \cr
      &&\cr
      \textcolor{red}{val} {\bf nilport }: 
      &  port  &  \cr
      }
    \halign{
      
      \textcolor{red}{\strut\vrule width 5pt\hskip 20pt \hfill } # & # \hfil  & \qquad # \cr
      &&\cr
      \textcolor{red}{val} {\bf nilpriv }: 
      &  privat  &  \cr
      }
    \halign{
      
      \textcolor{red}{\strut\vrule width 5pt\hskip 20pt \hfill } # & # \hfil  & \qquad # \cr
      &&\cr
      \textcolor{red}{val} {\bf nilcap }: 
      &  capability  &  \cr
      }
    \halign{
      
      \textcolor{red}{\strut\vrule width 5pt\hskip 20pt \hfill } # & # \hfil  & \qquad # \cr
      &&\cr
      \textcolor{red}{val} {\bf nilheader }: 
      &  header  &  \cr
      }
    \halign{\textcolor{red}{\strut\vrule width 5pt#}&#\cr&\cr}
    }
  \vskip 2\baselineskip
  Some care must be taken in the case of multithreaded programming in OCaML. 
  Because of the highly degree of optimizing in CaML, different threads of the 
  same function can share physically the same variables with undeterministic 
  behaviour. To get a physical new copy of an existing value, there are some 
  copy functions provided: 
  \vskip 2\baselineskip
  \vbox{\hfil
    \vbox{\hrule
      \hbox{\vrule\qquad
        \textcolor{red}{\rule[-.55em]{0pt}{1.7em}\sc Programming Interface}
        \quad\vrule}
      \hrule}
    }
  \nobreak\vskip \baselineskip\nobreak
  {
    \nobreak\vskip .5\baselineskip 
    \halign{
      
      \textcolor{red}{\strut\vrule width 5pt\hskip 20pt \hfill #} & # \hfil & # & #\hfill & \qquad # \hfill \cr
      &&&& \cr
      $[$ &  command  & $]=$ & {\bf cmd$\_$copy } & \cr
      & & & \hskip 10pt  command  & \cr
      }
    \halign{
      
      \textcolor{red}{\strut\vrule width 5pt\hskip 20pt \hfill #} & # \hfil & # & #\hfill & \qquad # \hfill \cr
      &&&& \cr
      $[$ &  status  & $]=$ & {\bf stat$\_$copy } & \cr
      & & & \hskip 10pt  status  & \cr
      }
    \halign{
      
      \textcolor{red}{\strut\vrule width 5pt\hskip 20pt \hfill #} & # \hfil & # & #\hfill & \qquad # \hfill \cr
      &&&& \cr
      $[$ &  header  & $]=$ & {\bf header$\_$copy } & \cr
      & & & \hskip 10pt  header  & \cr
      }
    \halign{\textcolor{red}{\strut\vrule width 5pt#}&#\cr&\cr}
    }
  \vskip 2\baselineskip
  }\end{list}}
\vskip .5\baselineskip
\vskip .5\baselineskip 
{\parindent 0pt \vbox{\sc \label{EncryptionandRights}Encryption and Rights }
\leftskip 20pt \rightskip 20pt \vskip .5\baselineskip\begin{list}{}{}\item{
\parindent 0pt 
  Amoeba uses currently a standard 48 Bit data encryption service. To encrypt a 
  port value, the {\sl one$\_$way }function is used internally. This function 
  uses itself the {\sl Des48 }module for the real encryption. The user level 
  function {\sl priv2pub }is equipped with an additional cache for already 
  encrypted ports. This function is used to convert a private server port to a 
  public client port. The private ports are used only by the server {\sl getreq 
  }function, and the public port is used only by the client {\sl trans }RPC 
  function. There are two functions for encoding and decoding of private parts 
  from a capability: 
  \vskip \baselineskip
  \begin{minipage}{.9\textwidth}\vbox{
    \begin{itemize}
      \item {\sl prv$\_$encode }
      \item {\sl prv$\_$decode }
      \end{itemize}
    }\end{minipage}
  \vskip \baselineskip
  To extract the rights field and the object number, the {\sl prv$\_$rights }
  and the {\sl prv$\_$numebr }functions can be used. There are also several 
  operation for {\sl rights$\_$bits }values. 
  \vskip 2\baselineskip
  \vbox{\hfil
    \vbox{\hrule
      \hbox{\vrule\qquad
        \textcolor{red}{\rule[-.55em]{0pt}{1.7em}\sc Programming Interface}
        \quad\vrule}
      \hrule}
    }
  \nobreak\vskip \baselineskip\nobreak
  {
    \nobreak\vskip .5\baselineskip 
    \halign{
      
      \textcolor{red}{\strut\vrule width 5pt\hskip 20pt \hfill } # & # \hfil  & \qquad # \cr
      &&\cr
      \textcolor{red}{val} {\bf prv$\_$all$\_$rights }: 
      &  rights$\_$bits  &  \cr
      }
    \halign{\textcolor{red}{\strut\vrule width 5pt\hskip 20pt #} & # \cr
    &\cr
    & {$\gg\:$\it Encrypt a port value $\ll$}\cr}
    \halign{
      
      \textcolor{red}{\strut\vrule width 5pt\hskip 20pt \hfill #} & # \hfil & # & #\hfill & \qquad # \hfill \cr
      &&&& \cr
      $[$ & {\it oport  }:  port  & $]=$ & {\bf one$\_$way } & \cr
      & & & \hskip 10pt {\it iport  }:  port  & \cr
      }
    \halign{\textcolor{red}{\strut\vrule width 5pt\hskip 20pt #} & # \cr
    &\cr
    & {$\gg\:$\it Similar to one$\_$way, but with an additional port cache 
    $\ll$}\cr}
    \halign{
      
      \textcolor{red}{\strut\vrule width 5pt\hskip 20pt \hfill #} & # \hfil & # & #\hfill & \qquad # \hfill \cr
      &&&& \cr
      $[$ & {\it pubport  }:  port  & $]=$ & {\bf priv2pub } & \cr
      & & & \hskip 10pt {\it prvport  }:  port  & \cr
      }
    \halign{\textcolor{red}{\strut\vrule width 5pt\hskip 20pt #} & # \cr
    &\cr
    & {$\gg\:$\it Decode a private structure $\ll$}\cr}
    \halign{
      
      \textcolor{red}{\strut\vrule width 5pt\hskip 20pt \hfill #} & # \hfil & # & #\hfill & \qquad # \hfill \cr
      &&&& \cr
      $[$ &  bool  & $]=$ & {\bf prv$\_$decode } & \cr
      & & & \hskip 10pt  $^\sim$prv  :  privat $\rightarrow$ & \cr
      & & & \hskip 10pt  $^\sim$rand  :  port  & \cr
      }
    \halign{\textcolor{red}{\strut\vrule width 5pt\hskip 20pt #} & # \cr
    &\cr
    & {$\gg\:$\it Encode a private part $\ll$}\cr}
    \halign{
      
      \textcolor{red}{\strut\vrule width 5pt\hskip 20pt \hfill #} & # \hfil & # & #\hfill & \qquad # \hfill \cr
      &&&& \cr
      $[$ & {\it eprivate  }:  privat  & $]=$ & {\bf prv$\_$encode } & \cr
      & & & \hskip 10pt  $^\sim$obj  :  obj$\_$num $\rightarrow$ & \cr
      & & & \hskip 10pt  $^\sim$rights  :  rights$\_$bits $\rightarrow$ & \cr
      & & & \hskip 10pt  $^\sim$rand  :  port  & \cr
      }
    \halign{\textcolor{red}{\strut\vrule width 5pt\hskip 20pt #} & # \cr
    &\cr
    & {$\gg\:$\it Return the object number $\ll$}\cr}
    \halign{
      
      \textcolor{red}{\strut\vrule width 5pt\hskip 20pt \hfill #} & # \hfil & # & #\hfill & \qquad # \hfill \cr
      &&&& \cr
      $[$ &  obj$\_$num  & $]=$ & {\bf prv$\_$number } & \cr
      & & & \hskip 10pt  privat  & \cr
      }
    \halign{\textcolor{red}{\strut\vrule width 5pt\hskip 20pt #} & # \cr
    &\cr
    & {$\gg\:$\it Return the rights $\ll$}\cr}
    \halign{
      
      \textcolor{red}{\strut\vrule width 5pt\hskip 20pt \hfill #} & # \hfil & # & #\hfill & \qquad # \hfill \cr
      &&&& \cr
      $[$ &  rights$\_$bits  & $]=$ & {\bf prv$\_$rights } & \cr
      & & & \hskip 10pt  privat  & \cr
      }
    \halign{\textcolor{red}{\strut\vrule width 5pt\hskip 20pt #} & # \cr
    &\cr
    & {$\gg\:$\it Return a new random port $\ll$}\cr}
    \halign{
      
      \textcolor{red}{\strut\vrule width 5pt\hskip 20pt \hfill #} & # \hfil & # & #\hfill & \qquad # \hfill \cr
      &&&& \cr
      $[$ & {\it newport  }:  port  & $]=$ & {\bf uniqport } & \cr
      & & & \hskip 10pt  ()  & \cr
      }
    \halign{
      
      \textcolor{red}{\strut\vrule width 5pt\hskip 20pt \hfill #} & # \hfil & # & #\hfill & \qquad # \hfill \cr
      &&&& \cr
      $[$ &  rights$\_$bits  & $]=$ & {\bf rights$\_$and $\_$or $\_$xor } & \cr
      & & & \hskip 10pt  rights$\_$bits $\rightarrow$ & \cr
      & & & \hskip 10pt  rights$\_$bits  & \cr
      }
    \halign{
      
      \textcolor{red}{\strut\vrule width 5pt\hskip 20pt \hfill #} & # \hfil & # & #\hfill & \qquad # \hfill \cr
      &&&& \cr
      $[$ &  rights$\_$bits  & $]=$ & {\bf rights$\_$not } & \cr
      & & & \hskip 10pt  rights$\_$bits  & \cr
      }
    \halign{
      
      \textcolor{red}{\strut\vrule width 5pt\hskip 20pt \hfill #} & # \hfil & # & #\hfill & \qquad # \hfill \cr
      &&&& \cr
      $[$ &  bool  & $]=$ & {\bf rights$\_$req } & \cr
      & & & \hskip 10pt  $^\sim$rights  :  rights$\_$bits $\rightarrow$ & \cr
      & & & \hskip 10pt  $^\sim$required  :  rights$\_$bits list  & \cr
      }
    \halign{
      
      \textcolor{red}{\strut\vrule width 5pt\hskip 20pt \hfill #} & # \hfil & # & #\hfill & \qquad # \hfill \cr
      &&&& \cr
      $[$ &  rights$\_$bits  & $]=$ & {\bf rights$\_$set } & \cr
      & & & \hskip 10pt  rights$\_$bits list  & \cr
      }
    \halign{\textcolor{red}{\strut\vrule width 5pt#}&#\cr&\cr}
    }
  \vskip 2\baselineskip
  The {\sl rights$\_$req }function checks wether all required rights are 
  present in the rights field. The {\sl rights$\_$set }builds a rights field 
  from a {\sl rights$\_$bits }list. }\end{list}}
\vskip .5\baselineskip
\vfill \eject
\vbox{\hsize \textwidth
  \hrule
  \vbox{\rule[-.55em]{0pt}{1.7em}\strut \vrule \quad 
    \label{Ar}\textcolor{blue}{\sc ML-Module: {\sl Ar }} \hfill 
    \textcolor[rgb]{.0,.7,.0}{\sc Package:  VAM Amoeba System}
    \quad \vrule}\hrule}
  
\vskip \baselineskip
\vskip .5\baselineskip 
{\parindent 0pt \vbox{\sc Description}
\leftskip 20pt \rightskip 20pt \vskip .5\baselineskip\begin{list}{}{}\item{
\parindent 0pt 
  This module delivers the programmer with a collection of functions to convert 
  Amoeba native structures into ASCII string format and vice versa. }\end{list}}
\vskip .5\baselineskip
\vskip .5\baselineskip 
{\parindent 0pt \vbox{\sc \label{Amoebatostring}Amoeba to string }
\leftskip 20pt \rightskip 20pt \vskip .5\baselineskip\begin{list}{}{}\item{
\parindent 0pt 
  \vskip 2\baselineskip
  \vbox{\hfil
    \vbox{\hrule
      \hbox{\vrule\qquad
        \textcolor{red}{\rule[-.55em]{0pt}{1.7em}\sc Programming Interface}
        \quad\vrule}
      \hrule}
    }
  \nobreak\vskip \baselineskip\nobreak
  {
    \nobreak\vskip .5\baselineskip 
    \halign{\textcolor{red}{\strut\vrule width 5pt\hskip 20pt #} & # \cr
    &\cr
    & {$\gg\:$\it Return a string representation of a port $\ll$}\cr}
    \halign{
      
      \textcolor{red}{\strut\vrule width 5pt\hskip 20pt \hfill #} & # \hfil & # & #\hfill & \qquad # \hfill \cr
      &&&& \cr
      $[$ &  string  & $]=$ & {\bf ar$\_$port } & \cr
      & & & \hskip 10pt {\it port  }:  port  & \cr
      }
    \halign{\textcolor{red}{\strut\vrule width 5pt\hskip 20pt #} & # \cr
    &\cr
    & {$\gg\:$\it Return a string representation of a private structure $\ll$}
    \cr}
    \halign{
      
      \textcolor{red}{\strut\vrule width 5pt\hskip 20pt \hfill #} & # \hfil & # & #\hfill & \qquad # \hfill \cr
      &&&& \cr
      $[$ &  string  & $]=$ & {\bf ar$\_$priv } & \cr
      & & & \hskip 10pt {\it private  }:  privat  & \cr
      }
    \halign{\textcolor{red}{\strut\vrule width 5pt\hskip 20pt #} & # \cr
    &\cr
    & {$\gg\:$\it Return a string representation of a capability $\ll$}\cr}
    \halign{
      
      \textcolor{red}{\strut\vrule width 5pt\hskip 20pt \hfill #} & # \hfil & # & #\hfill & \qquad # \hfill \cr
      &&&& \cr
      $[$ &  string  & $]=$ & {\bf ar$\_$cap } & \cr
      & & & \hskip 10pt {\it cap  }:  capability  & \cr
      }
    \halign{\textcolor{red}{\strut\vrule width 5pt#}&#\cr&\cr}
    }
  \vskip 2\baselineskip
  The string output has the following format: 
  \vskip 2\baselineskip
  \halign{
    \hskip3em#\hfil\cr
     \cr
    {\tt \hskip 1emPort: x:x:x:x:x [x: hexadecimal value] } \cr
    {\tt \hskip 1em Private: d(x)/x:x:x:x:x:x [d: decimal, x: hexadecimal value] } \cr
    {\tt \hskip 1em \hskip 1em \hskip 1em \hskip 1em d: object number } \cr
    {\tt \hskip 1em \hskip 1em \hskip 1em \hskip 1em (x): rights field } \cr
    {\tt \hskip 1em Capability: x:x:x:x:x:x/d(x)/x:x:x:x:x:x } \cr
    {\tt \hskip 1em \hskip 1em \hskip 1em \hskip 1em Left part: port, Right part: private } \cr
    {\tt \hskip 1em \hfil} \cr 
    }
  
\vskip 2\baselineskip
}\end{list}}
  \vskip .5\baselineskip
  \vskip .5\baselineskip 
  {\parindent 0pt \vbox{\sc \label{StringtoAmoeba}String to Amoeba }
  \leftskip 20pt \rightskip 20pt \vskip .5\baselineskip\begin{list}{}{}\item{
  \parindent 0pt 
\vskip 2\baselineskip
\vbox{\hfil
  \vbox{\hrule
    \hbox{\vrule\qquad
      \textcolor{red}{\rule[-.55em]{0pt}{1.7em}\sc Programming Interface}
      \quad\vrule}
    \hrule}
  }
\nobreak\vskip \baselineskip\nobreak
{
  \nobreak\vskip .5\baselineskip 
  \halign{\textcolor{red}{\strut\vrule width 5pt\hskip 20pt #} & # \cr
  &\cr
  & {$\gg\:$\it Convert a string representation of a port to a port $\ll$}\cr}
  \halign{
    
    \textcolor{red}{\strut\vrule width 5pt\hskip 20pt \hfill #} & # \hfil & # & #\hfill & \qquad # \hfill \cr
    &&&& \cr
    $[$ &  port  & $]=$ & {\bf ar$\_$toport } & \cr
    & & & \hskip 10pt {\it sport  }:  string  & \cr
    }
  \halign{\textcolor{red}{\strut\vrule width 5pt\hskip 20pt #} & # \cr
  &\cr
  & {$\gg\:$\it Convert a string representation of a private structure $\ll$}
  \cr}
  \halign{
    
    \textcolor{red}{\strut\vrule width 5pt\hskip 20pt \hfill #} & # \hfil & # & #\hfill & \qquad # \hfill \cr
    &&&& \cr
    $[$ &  privat  & $]=$ & {\bf ar$\_$topriv } & \cr
    & & & \hskip 10pt {\it sprivate  }:  string  & \cr
    }
  \halign{\textcolor{red}{\strut\vrule width 5pt\hskip 20pt #} & # \cr
  &\cr
  & {$\gg\:$\it Convert a string representation of a capability $\ll$}\cr}
  \halign{
    
    \textcolor{red}{\strut\vrule width 5pt\hskip 20pt \hfill #} & # \hfil & # & #\hfill & \qquad # \hfill \cr
    &&&& \cr
    $[$ &  capability  & $]=$ & {\bf ar$\_$tocap } & \cr
    & & & \hskip 10pt {\it scap  }:  string  & \cr
    }
  \halign{\textcolor{red}{\strut\vrule width 5pt#}&#\cr&\cr}
  }
\vskip 2\baselineskip
The input string must have the following format: 
\vskip 2\baselineskip
\halign{
  \hskip3em#\hfil\cr
   \cr
  {\tt \hskip 1emPort: x:x:x:x:x [x: hexadecimal value] } \cr
  {\tt \hskip 1em Private: d(x)/x:x:x:x:x:x [d: decimal, x: hexadecimal value] } \cr
  {\tt \hskip 1em \hskip 1em \hskip 1em \hskip 1em d: object number } \cr
  {\tt \hskip 1em \hskip 1em \hskip 1em \hskip 1em (x): rights field } \cr
  {\tt \hskip 1em Capability: x:x:x:x:x:x/d(x)/x:x:x:x:x:x } \cr
  {\tt \hskip 1em \hskip 1em \hskip 1em \hskip 1em Left part: port, Right part: private } \cr
  {\tt \hskip 1em \hfil} \cr 
  }

  \vskip 2\baselineskip
  The {\sl ar$\_$cap }and the {\sl ar$\_$tocap }functions are also available under 
  the names {\sl c2a }and {\sl a2c }respectively. }\end{list}}
    \vskip .5\baselineskip
    \vskip .5\baselineskip 
    {\parindent 0pt \vbox{\sc \label{ModuleDependencies}Module Dependencies }
    \leftskip 20pt \rightskip 20pt \vskip .5\baselineskip\begin{list}{}{}\item{
    \parindent 0pt 
  \vskip \baselineskip
  \begin{minipage}{.9\textwidth}\vbox{
\begin{itemize}
  \item Amoeba 
  \end{itemize}
}\end{minipage}
  \vskip \baselineskip
  }\end{list}}
    \vskip .5\baselineskip
    \vfill \eject
    \vbox{\hsize \textwidth
  \hrule
  \vbox{\rule[-.55em]{0pt}{1.7em}\strut \vrule \quad 
\label{Buf}\textcolor{blue}{\sc ML-Module: {\sl Buf }} \hfill 
\textcolor[rgb]{.0,.7,.0}{\sc Package:  VAM Amoeba System}\quad \vrule}\hrule}
  
    \vskip \baselineskip
    \vskip .5\baselineskip 
    {\parindent 0pt \vbox{\sc Description}
    \leftskip 20pt \rightskip 20pt \vskip .5\baselineskip\begin{list}{}{}\item{
    \parindent 0pt 
  Machine independent functions for storing and extracting of Amoeba structures in 
  and from buffers with bound checking. }\end{list}}
    \vskip .5\baselineskip
    \vskip .5\baselineskip 
    {\parindent 0pt \vbox{\sc \label{Putfunctions}Put functions }
    \leftskip 20pt \rightskip 20pt \vskip .5\baselineskip\begin{list}{}{}\item{
    \parindent 0pt 
  \vskip 2\baselineskip
  \vbox{\hfil
\vbox{\hrule
  \hbox{\vrule\qquad
    \textcolor{red}{\rule[-.55em]{0pt}{1.7em}\sc Programming Interface}
    \quad\vrule}
  \hrule}
}
  \nobreak\vskip \baselineskip\nobreak
  {
\nobreak\vskip .5\baselineskip 
\halign{\textcolor{red}{\strut\vrule width 5pt\hskip 20pt #} & # \cr
&\cr
& {$\gg\:$\it Put a short int16 value into a buffer $\ll$}\cr}
\halign{
  
  \textcolor{red}{\strut\vrule width 5pt\hskip 20pt \hfill #} & # \hfil & # & #\hfill & \qquad # \hfill \cr
  &&&& \cr
  $[$ & {\it newpos  }:  int  & $]=$ & {\bf buf$\_$put$\_$int16 } & \cr
  & & & \hskip 10pt  $^\sim$buf  :  buffer $\rightarrow$ & \cr
  & & & \hskip 10pt  $^\sim$pos  :  int $\rightarrow$ & \cr
  & & & \hskip 10pt  $^\sim$int16  :  int  & \cr
  }
\halign{\textcolor{red}{\strut\vrule width 5pt\hskip 20pt #} & # \cr
&\cr
& {$\gg\:$\it Put a long int32 value into a buffer $\ll$}\cr}
\halign{
  
  \textcolor{red}{\strut\vrule width 5pt\hskip 20pt \hfill #} & # \hfil & # & #\hfill & \qquad # \hfill \cr
  &&&& \cr
  $[$ & {\it newpos  }:  int  & $]=$ & {\bf buf$\_$put$\_$int32 } & \cr
  & & & \hskip 10pt  $^\sim$buf  :  buffer $\rightarrow$ & \cr
  & & & \hskip 10pt  $^\sim$pos  :  int $\rightarrow$ & \cr
  & & & \hskip 10pt  $^\sim$int32  :  int  & \cr
  }
\halign{\textcolor{red}{\strut\vrule width 5pt\hskip 20pt #} & # \cr
&\cr
& {$\gg\:$\it Put a string into a buffer and add the char '$\setminus$000' 
$\ll$}\cr}
\halign{
  
  \textcolor{red}{\strut\vrule width 5pt\hskip 20pt \hfill #} & # \hfil & # & #\hfill & \qquad # \hfill \cr
  &&&& \cr
  $[$ & {\it newpos  }:  int  & $]=$ & {\bf buf$\_$put$\_$string } & \cr
  & & & \hskip 10pt  $^\sim$buf  :  buffer $\rightarrow$ & \cr
  & & & \hskip 10pt  $^\sim$pos  :  int $\rightarrow$ & \cr
  & & & \hskip 10pt  $^\sim$str  :  string  & \cr
  }
\halign{\textcolor{red}{\strut\vrule width 5pt\hskip 20pt #} & # \cr
&\cr
& {$\gg\:$\it Put a port into a buffer $\ll$}\cr}
\halign{
  
  \textcolor{red}{\strut\vrule width 5pt\hskip 20pt \hfill #} & # \hfil & # & #\hfill & \qquad # \hfill \cr
  &&&& \cr
  $[$ & {\it newpos  }:  int  & $]=$ & {\bf buf$\_$put$\_$port } & \cr
  & & & \hskip 10pt  $^\sim$buf  :  buffer $\rightarrow$ & \cr
  & & & \hskip 10pt  $^\sim$pos  :  int $\rightarrow$ & \cr
  & & & \hskip 10pt  $^\sim$port  :  port  & \cr
  }
\halign{\textcolor{red}{\strut\vrule width 5pt\hskip 20pt #} & # \cr
&\cr
& {$\gg\:$\it Put a private part of a cap. into a buffer $\ll$}\cr}
\halign{
  
  \textcolor{red}{\strut\vrule width 5pt\hskip 20pt \hfill #} & # \hfil & # & #\hfill & \qquad # \hfill \cr
  &&&& \cr
  $[$ & {\it newpos  }:  int  & $]=$ & {\bf buf$\_$put$\_$priv } & \cr
  & & & \hskip 10pt  $^\sim$buf  :  buffer $\rightarrow$ & \cr
  & & & \hskip 10pt  $^\sim$pos  :  int $\rightarrow$ & \cr
  & & & \hskip 10pt  $^\sim$priv  :  private  & \cr
  }
\halign{\textcolor{red}{\strut\vrule width 5pt\hskip 20pt #} & # \cr
&\cr
& {$\gg\:$\it Put a capability into a buffer $\ll$}\cr}
\halign{
  
  \textcolor{red}{\strut\vrule width 5pt\hskip 20pt \hfill #} & # \hfil & # & #\hfill & \qquad # \hfill \cr
  &&&& \cr
  $[$ & {\it newpos  }:  int  & $]=$ & {\bf buf$\_$put$\_$cap } & \cr
  & & & \hskip 10pt  $^\sim$buf  :  buffer $\rightarrow$ & \cr
  & & & \hskip 10pt  $^\sim$pos  :  int $\rightarrow$ & \cr
  & & & \hskip 10pt  $^\sim$cap  :  capability  & \cr
  }
\halign{\textcolor{red}{\strut\vrule width 5pt\hskip 20pt #} & # \cr
&\cr
& {$\gg\:$\it Put a capset into a buffer $\ll$}\cr}
\halign{
  
  \textcolor{red}{\strut\vrule width 5pt\hskip 20pt \hfill #} & # \hfil & # & #\hfill & \qquad # \hfill \cr
  &&&& \cr
  $[$ & {\it newpos  }:  int  & $]=$ & {\bf buf$\_$put$\_$capset } & \cr
  & & & \hskip 10pt  $^\sim$buf  :  buffer $\rightarrow$ & \cr
  & & & \hskip 10pt  $^\sim$pos  :  int $\rightarrow$ & \cr
  & & & \hskip 10pt  $^\sim$cs  :  capset  & \cr
  }
\halign{\textcolor{red}{\strut\vrule width 5pt\hskip 20pt #} & # \cr
&\cr
& {$\gg\:$\it Put rights bits into a buffer $\ll$}\cr}
\halign{
  
  \textcolor{red}{\strut\vrule width 5pt\hskip 20pt \hfill #} & # \hfil & # & #\hfill & \qquad # \hfill \cr
  &&&& \cr
  $[$ & {\it newpos  }:  int  & $]=$ & {\bf buf$\_$put$\_$right$\_$bits } & \cr
  & & & \hskip 10pt  $^\sim$buf  :  buffer $\rightarrow$ & \cr
  & & & \hskip 10pt  $^\sim$pos  :  int $\rightarrow$ & \cr
  & & & \hskip 10pt  $^\sim$right  :  int  & \cr
  }
\halign{\textcolor{red}{\strut\vrule width 5pt\hskip 20pt #} & # \cr
&\cr
& {$\gg\:$\it Put rights bits into a buffer $\ll$}\cr}
\halign{
  
  \textcolor{red}{\strut\vrule width 5pt\hskip 20pt \hfill #} & # \hfil & # & #\hfill & \qquad # \hfill \cr
  &&&& \cr
  $[$ & {\it newpos  }:  int  & $]=$ & {\bf buf$\_$put$\_$rights$\_$bits } & \cr
  & & & \hskip 10pt  $^\sim$buf  :  buffer $\rightarrow$ & \cr
  & & & \hskip 10pt  $^\sim$pos  :  int $\rightarrow$ & \cr
  & & & \hskip 10pt  $^\sim$rights  :  rights$\_$bits  & \cr
  }
\halign{\textcolor{red}{\strut\vrule width 5pt#}&#\cr&\cr}
}
  \vskip 2\baselineskip
  The {\sl pos }argument specifies the start position in the buffer. All functions 
  return the next position in the buffer. }\end{list}}
    \vskip .5\baselineskip
    \vskip .5\baselineskip 
    {\parindent 0pt \vbox{\sc \label{Getfunctions}Get functions }
    \leftskip 20pt \rightskip 20pt \vskip .5\baselineskip\begin{list}{}{}\item{
    \parindent 0pt 
  \vskip 2\baselineskip
  \vbox{\hfil
\vbox{\hrule
  \hbox{\vrule\qquad
    \textcolor{red}{\rule[-.55em]{0pt}{1.7em}\sc Programming Interface}
    \quad\vrule}
  \hrule}
}
  \nobreak\vskip \baselineskip\nobreak
  {
\nobreak\vskip .5\baselineskip 
\halign{\textcolor{red}{\strut\vrule width 5pt\hskip 20pt #} & # \cr
&\cr
& {$\gg\:$\it Get a short int16 value from a buffer $\ll$}\cr}
\halign{
  
  \textcolor{red}{\strut\vrule width 5pt\hskip 20pt \hfill #} & # \hfil & # & #\hfill & \qquad # \hfill \cr
  &&&& \cr
  $[$ & {\it newpos  }:  int $*$ & &  & \cr
  & {\it int16  }:  int  & $]=$ & {\bf buf$\_$get$\_$int16 } & \cr
  & & & \hskip 10pt  $^\sim$buf  :  buffer $\rightarrow$ & \cr
  & & & \hskip 10pt  $^\sim$pos  :  int  & \cr
  }
\halign{\textcolor{red}{\strut\vrule width 5pt\hskip 20pt #} & # \cr
&\cr
& {$\gg\:$\it Get a long int32 value from a buffer $\ll$}\cr}
\halign{
  
  \textcolor{red}{\strut\vrule width 5pt\hskip 20pt \hfill #} & # \hfil & # & #\hfill & \qquad # \hfill \cr
  &&&& \cr
  $[$ & {\it newpos  }:  int $*$ & &  & \cr
  & {\it int32  }:  int  & $]=$ & {\bf buf$\_$get$\_$int32 } & \cr
  & & & \hskip 10pt  $^\sim$buf  :  buffer $\rightarrow$ & \cr
  & & & \hskip 10pt  $^\sim$pos  :  int  & \cr
  }
\halign{\textcolor{red}{\strut\vrule width 5pt\hskip 20pt #} & # \cr
&\cr
& {$\gg\:$\it Get a string with EOS char from a buffer $\ll$}\cr}
\halign{
  
  \textcolor{red}{\strut\vrule width 5pt\hskip 20pt \hfill #} & # \hfil & # & #\hfill & \qquad # \hfill \cr
  &&&& \cr
  $[$ & {\it newpos  }:  int $*$ & &  & \cr
  & {\it str  }:  string  & $]=$ & {\bf buf$\_$get$\_$string } & \cr
  & & & \hskip 10pt  $^\sim$buf  :  buffer $\rightarrow$ & \cr
  & & & \hskip 10pt  $^\sim$pos  :  int  & \cr
  }
\halign{\textcolor{red}{\strut\vrule width 5pt\hskip 20pt #} & # \cr
&\cr
& {$\gg\:$\it Get a port from a buffer $\ll$}\cr}
\halign{
  
  \textcolor{red}{\strut\vrule width 5pt\hskip 20pt \hfill #} & # \hfil & # & #\hfill & \qquad # \hfill \cr
  &&&& \cr
  $[$ & {\it newpos  }:  int $*$ & &  & \cr
  & {\it port  }:  port  & $]=$ & {\bf buf$\_$get$\_$port } & \cr
  & & & \hskip 10pt  $^\sim$buf  :  buffer $\rightarrow$ & \cr
  & & & \hskip 10pt  $^\sim$pos  :  int  & \cr
  }
\halign{\textcolor{red}{\strut\vrule width 5pt\hskip 20pt #} & # \cr
&\cr
& {$\gg\:$\it Get a private part of a cap. from a buffer $\ll$}\cr}
\halign{
  
  \textcolor{red}{\strut\vrule width 5pt\hskip 20pt \hfill #} & # \hfil & # & #\hfill & \qquad # \hfill \cr
  &&&& \cr
  $[$ & {\it newpos  }:  int $*$ & &  & \cr
  & {\it priv  }:  privat  & $]=$ & {\bf buf$\_$get$\_$priv } & \cr
  & & & \hskip 10pt  $^\sim$buf  :  buffer $\rightarrow$ & \cr
  & & & \hskip 10pt  $^\sim$pos  :  int  & \cr
  }
\halign{\textcolor{red}{\strut\vrule width 5pt\hskip 20pt #} & # \cr
&\cr
& {$\gg\:$\it Get a capability from a buffer $\ll$}\cr}
\halign{
  
  \textcolor{red}{\strut\vrule width 5pt\hskip 20pt \hfill #} & # \hfil & # & #\hfill & \qquad # \hfill \cr
  &&&& \cr
  $[$ & {\it newpos  }:  int $*$ & &  & \cr
  & {\it cap  }:  capability  & $]=$ & {\bf buf$\_$get$\_$cap } & \cr
  & & & \hskip 10pt  $^\sim$buf  :  buffer $\rightarrow$ & \cr
  & & & \hskip 10pt  $^\sim$pos  :  int  & \cr
  }
\halign{\textcolor{red}{\strut\vrule width 5pt\hskip 20pt #} & # \cr
&\cr
& {$\gg\:$\it Get a capset from a buffer $\ll$}\cr}
\halign{
  
  \textcolor{red}{\strut\vrule width 5pt\hskip 20pt \hfill #} & # \hfil & # & #\hfill & \qquad # \hfill \cr
  &&&& \cr
  $[$ & {\it newpos  }:  int $*$ & &  & \cr
  & {\it cs  }:  capset  & $]=$ & {\bf buf$\_$get$\_$capset } & \cr
  & & & \hskip 10pt  $^\sim$buf  :  buffer $\rightarrow$ & \cr
  & & & \hskip 10pt  $^\sim$pos  :  int  & \cr
  }
\halign{\textcolor{red}{\strut\vrule width 5pt\hskip 20pt #} & # \cr
&\cr
& {$\gg\:$\it Get rights bits from a buffer $\ll$}\cr}
\halign{
  
  \textcolor{red}{\strut\vrule width 5pt\hskip 20pt \hfill #} & # \hfil & # & #\hfill & \qquad # \hfill \cr
  &&&& \cr
  $[$ & {\it newpos  }:  int $*$ & &  & \cr
  & {\it right  }:  int  & $]=$ & {\bf buf$\_$get$\_$right$\_$bits } & \cr
  & & & \hskip 10pt  $^\sim$buf  :  buffer $\rightarrow$ & \cr
  & & & \hskip 10pt  $^\sim$pos  :  int  & \cr
  }
\halign{\textcolor{red}{\strut\vrule width 5pt\hskip 20pt #} & # \cr
&\cr
& {$\gg\:$\it Get rights bits from a buffer $\ll$}\cr}
\halign{
  
  \textcolor{red}{\strut\vrule width 5pt\hskip 20pt \hfill #} & # \hfil & # & #\hfill & \qquad # \hfill \cr
  &&&& \cr
  $[$ & {\it newpos  }:  int $*$ & &  & \cr
  & {\it rights  }:  rights$\_$bits  & $]=$ & {\bf buf$\_$get$\_$rights$\_$bits 
  } & \cr
  & & & \hskip 10pt  $^\sim$buf  :  buffer $\rightarrow$ & \cr
  & & & \hskip 10pt  $^\sim$pos  :  int  & \cr
  }
\halign{\textcolor{red}{\strut\vrule width 5pt#}&#\cr&\cr}
}
  \vskip 2\baselineskip
  The {\sl pos }argument specifies the start position in the buffer. All functions 
  return the next position in the buffer. }\end{list}}
    \vskip .5\baselineskip
    \vskip .5\baselineskip 
    {\parindent 0pt \vbox{\sc \label{ModuleDependencies}Module Dependencies }
    \leftskip 20pt \rightskip 20pt \vskip .5\baselineskip\begin{list}{}{}\item{
    \parindent 0pt 
  \vskip \baselineskip
  \begin{minipage}{.9\textwidth}\vbox{
\begin{itemize}
  \item Amoeba 
  \item Capset 
  \end{itemize}
}\end{minipage}
  \vskip \baselineskip
  }\end{list}}
    \vskip .5\baselineskip
    \vfill \eject
    \vbox{\hsize \textwidth
  \hrule
  \vbox{\rule[-.55em]{0pt}{1.7em}\strut \vrule \quad 
\label{Cache}\textcolor{blue}{\sc ML-Module: {\sl Cache }} \hfill 
\textcolor[rgb]{.0,.7,.0}{\sc Package:  VAM Amoeba System}\quad \vrule}\hrule}
  
    \vskip \baselineskip
    \vskip .5\baselineskip 
    {\parindent 0pt \vbox{\sc Description}
    \leftskip 20pt \rightskip 20pt \vskip .5\baselineskip\begin{list}{}{}\item{
    \parindent 0pt 
  This cache module provides a fixed table cache. The fixed table is treated like a 
  circular buffer, therefore if fille up, the oldest entry is overwritten. 
  Seraching is done from the newest entry downto the oldest. Assumption: the cache 
  is mostly filled up. It's possible to invalidate (remove) a cache entry. 
  }\end{list}}
    \vskip .5\baselineskip
    \vskip 2\baselineskip
    \vbox{\hfil
  \vbox{\hrule
\hbox{\vrule\qquad
  \textcolor{red}{\rule[-.55em]{0pt}{1.7em}\sc Programming Interface}
  \quad\vrule}
\hrule}
  }
    \nobreak\vskip \baselineskip\nobreak
    {
  \nobreak\vskip .5\baselineskip 
  \halign{\textcolor{red}{\strut\vrule width 5pt\hskip 20pt #} & # \cr
  &\cr
  & {$\gg\:$\it The cache entry structure $\ll$}\cr}
  \halign{

\textcolor{red}{\strut\vrule width 5pt\hskip 20pt \hfill #} & # \hfil & # \hfil & #\hfill & \qquad # \hfill \cr
&&&&\cr
\textcolor{red}{type} & {\bf ('a,'b) cache$\_$entry } $=$  $\{$  & 
&cache$\_$key : 'a $;$ & \cr
&& &cache$\_$data : 'b  $\}$  & \cr
}
  \halign{\textcolor{red}{\strut\vrule width 5pt\hskip 20pt #} & # \cr
  &\cr
  & {$\gg\:$\it The cache structure $\ll$}\cr}
  \halign{

\textcolor{red}{\strut\vrule width 5pt\hskip 20pt \hfill #} & # \hfil & # \hfil & #\hfill & \qquad # \hfill \cr
&&&&\cr
\textcolor{red}{type} & {\bf ('a,'b) t } $=$  $\{$  & 
&\textcolor{red}{mutable }cache$\_$size : int $;$ & \cr
&& &\textcolor{red}{mutable }cache$\_$head : int $;$ & \cr
&& &\textcolor{red}{mutable }cache$\_$hit : int $;$ & \cr
&& &\textcolor{red}{mutable }cache$\_$miss : int $;$ & \cr
&& &\textcolor{red}{mutable }cache$\_$table : ('a, 'b) cache$\_$entry option 
array  $\}$  & \cr
}
  \halign{\textcolor{red}{\strut\vrule width 5pt\hskip 20pt #} & # \cr
  &\cr
  & {$\gg\:$\it Create a new cache $\ll$}\cr}
  \halign{

\textcolor{red}{\strut\vrule width 5pt\hskip 20pt \hfill #} & # \hfil & # & #\hfill & \qquad # \hfill \cr
&&&& \cr
$[$ &  ('a,'b) t  & $]=$ & {\bf create } & \cr
& & & \hskip 10pt  $^\sim$size  :  n  & \cr
}
  \halign{\textcolor{red}{\strut\vrule width 5pt\hskip 20pt #} & # \cr
  &\cr
  & {$\gg\:$\it Add a key:data tuple to the cache $\ll$}\cr}
  \halign{

\textcolor{red}{\strut\vrule width 5pt\hskip 20pt \hfill #} & # \hfil & # & #\hfill & \qquad # \hfill \cr
&&&& \cr
$[$ &  unit  & $]=$ & {\bf add } & \cr
& & & \hskip 10pt  $^\sim$cache  :  ('a,'b) t $\rightarrow$ & \cr
& & & \hskip 10pt  $^\sim$key  :  'a $\rightarrow$ & \cr
& & & \hskip 10pt  $^\sim$data  :  'b  & \cr
}
  \halign{\textcolor{red}{\strut\vrule width 5pt\hskip 20pt #} & # \cr
  &\cr
  & {$\gg\:$\it Lookup a key in the cache and return the data assoziated with it 
  $\ll$}\cr}
  \halign{

\textcolor{red}{\strut\vrule width 5pt\hskip 20pt \hfill #} & # \hfil & # & #\hfill & \qquad # \hfill \cr
&&&& \cr
$[$ & {\it data  }:  $^\sim$b  & $]=$ & {\bf lookup } & \cr
& & & \hskip 10pt  $^\sim$cache  :  ('a,'b) t $\rightarrow$ & \cr
& & & \hskip 10pt  $^\sim$key  :  'a  & \cr
}
  \halign{\textcolor{red}{\strut\vrule width 5pt\hskip 20pt #} & # \cr
  &\cr
  & {$\gg\:$\it Invalidate the cache entry determined by the key $\ll$}\cr}
  \halign{

\textcolor{red}{\strut\vrule width 5pt\hskip 20pt \hfill #} & # \hfil & # & #\hfill & \qquad # \hfill \cr
&&&& \cr
$[$ &  unit  & $]=$ & {\bf invalidate } & \cr
& & & \hskip 10pt  $^\sim$cache  :  ('a,'b) t $\rightarrow$ & \cr
& & & \hskip 10pt  $^\sim$key  :  'a  & \cr
}
  \halign{\textcolor{red}{\strut\vrule width 5pt#}&#\cr&\cr}
  }
    \vskip 2\baselineskip
    \vfill \eject
    \vbox{\hsize \textwidth
  \hrule
  \vbox{\rule[-.55em]{0pt}{1.7em}\strut \vrule \quad 
\label{Cap$_$env}\textcolor{blue}{\sc ML-Module: {\sl Cap$\_$env }} \hfill 
\textcolor[rgb]{.0,.7,.0}{\sc Package:  VAM Amoeba System}\quad \vrule}\hrule}
  
    \vskip \baselineskip
    \vskip .5\baselineskip 
    {\parindent 0pt \vbox{\sc Description}
    \leftskip 20pt \rightskip 20pt \vskip .5\baselineskip\begin{list}{}{}\item{
    \parindent 0pt 
  Access to Amoeba's capability environment. }\end{list}}
    \vskip .5\baselineskip
    \vskip 2\baselineskip
    \vbox{\hfil
  \vbox{\hrule
\hbox{\vrule\qquad
  \textcolor{red}{\rule[-.55em]{0pt}{1.7em}\sc Programming Interface}
  \quad\vrule}
\hrule}
  }
    \nobreak\vskip \baselineskip\nobreak
    {
  \nobreak\vskip .5\baselineskip 
  \halign{\textcolor{red}{\strut\vrule width 5pt\hskip 20pt #} & # \cr
  &\cr
  & {$\gg\:$\it Named Amoeba environement capabilities $\ll$}\cr}
  \halign{

\textcolor{red}{\strut\vrule width 5pt\hskip 20pt \hfill #} & # \hfil & # \hfil & #\hfill & \qquad # \hfill \cr
&&&&\cr
\textcolor{red}{type} & {\bf amoeba$\_$env } $=$  & 
&Root &{$\gg\:$\it The root directory $\ll$}\cr
&&  $|$ &Work &{$\gg\:$\it The working directory $\ll$}\cr
&&  $|$ &Stdin &{$\gg\:$\it Standard Input Channel $\ll$}\cr
&&  $|$ &Stdout &{$\gg\:$\it Standard Output Channel $\ll$}\cr
&&  $|$ &Stderr &{$\gg\:$\it Standard Error channl $\ll$}\cr
&&  $|$ &Session &{$\gg\:$\it The Session Server $\ll$}\cr
&&  $|$ &Tty &{$\gg\:$\it The Terminal Server $\ll$}\cr
}
  \halign{\textcolor{red}{\strut\vrule width 5pt\hskip 20pt #} & # \cr
  &\cr
  & {$\gg\:$\it Get an environment capability $\ll$}\cr}
  \halign{

\textcolor{red}{\strut\vrule width 5pt\hskip 20pt \hfill #} & # \hfil & # & #\hfill & \qquad # \hfill \cr
&&&& \cr
$[$ & {\it envcap  }:  capability  & $]=$ & {\bf get$\_$env$\_$cap } & \cr
& & & \hskip 10pt {\it envname  }:  amoeba$\_$env  & \cr
}
  \halign{\textcolor{red}{\strut\vrule width 5pt#}&#\cr&\cr}
  }
    \vskip 2\baselineskip
    Without a Session Server, either an external or builtin one, only The {\sl Root }
    capability is available. \vskip .5\baselineskip 
    {\parindent 0pt \vbox{\sc \label{ModuleDependencies}Module Dependencies }
    \leftskip 20pt \rightskip 20pt \vskip .5\baselineskip\begin{list}{}{}\item{
    \parindent 0pt 
  \vskip \baselineskip
  \begin{minipage}{.9\textwidth}\vbox{
\begin{itemize}
  \item Amoeba 
  \end{itemize}
}\end{minipage}
  \vskip \baselineskip
  }\end{list}}
    \vskip .5\baselineskip
    \vfill \eject
    \vbox{\hsize \textwidth
  \hrule
  \vbox{\rule[-.55em]{0pt}{1.7em}\strut \vrule \quad 
\label{Capset}\textcolor{blue}{\sc ML-Module: {\sl Capset }} \hfill 
\textcolor[rgb]{.0,.7,.0}{\sc Package:  VAM Amoeba System}\quad \vrule}\hrule}
  
    \vskip \baselineskip
    \vskip .5\baselineskip 
    {\parindent 0pt \vbox{\sc Description}
    \leftskip 20pt \rightskip 20pt \vskip .5\baselineskip\begin{list}{}{}\item{
    \parindent 0pt 
  Functions and structures for building capability sets. }\end{list}}
    \vskip .5\baselineskip
    \vskip .5\baselineskip 
    {\parindent 0pt \vbox{\sc \label{Structures}Structures }
    \leftskip 20pt \rightskip 20pt \vskip .5\baselineskip\begin{list}{}{}\item{
    \parindent 0pt 
  \vskip 2\baselineskip
  \vbox{\hfil
\vbox{\hrule
  \hbox{\vrule\qquad
    \textcolor{red}{\rule[-.55em]{0pt}{1.7em}\sc Programming Interface}
    \quad\vrule}
  \hrule}
}
  \nobreak\vskip \baselineskip\nobreak
  {
\nobreak\vskip .5\baselineskip 
\halign{\textcolor{red}{\strut\vrule width 5pt\hskip 20pt #} & # \cr
&\cr
& {$\gg\:$\it Capability set structure $\ll$}\cr}
\halign{
  
  \textcolor{red}{\strut\vrule width 5pt\hskip 20pt \hfill #} & # \hfil & # \hfil & #\hfill & \qquad # \hfill \cr
  &&&&\cr
  \textcolor{red}{type} & {\bf capset } $=$  $\{$  & 
  &\textcolor{red}{mutable }cs$\_$initial : int $;$ & \cr
  && &\textcolor{red}{mutable }cs$\_$final : int $;$ & \cr
  && &\textcolor{red}{mutable }cs$\_$suite : suite array  $\}$  & \cr
  }
\halign{
  
  \textcolor{red}{\strut\vrule width 5pt\hskip 20pt \hfill #} & # \hfil & # \hfil & #\hfill & \qquad # \hfill \cr
  &&&&\cr
  \textcolor{red}{type} & {\bf suite } $=$  $\{$  & 
  &\textcolor{red}{mutable }s$\_$object : capability $;$ & \cr
  && &\textcolor{red}{mutable }s$\_$current : bool  $\}$  & \cr
  }
\halign{\textcolor{red}{\strut\vrule width 5pt\hskip 20pt #} & # \cr
&\cr
& {$\gg\:$\it Empty capability set $\ll$}\cr}
\halign{
  
  \textcolor{red}{\strut\vrule width 5pt\hskip 20pt \hfill } # & # \hfil  & \qquad # \cr
  &&\cr
  \textcolor{red}{val} {\bf nilcapset }: 
  &  capset  &  \cr
  }
\halign{\textcolor{red}{\strut\vrule width 5pt#}&#\cr&\cr}
}
  \vskip 2\baselineskip
  }\end{list}}
    \vskip .5\baselineskip
    \vskip .5\baselineskip 
    {\parindent 0pt \vbox{\sc \label{Functions}Functions }
    \leftskip 20pt \rightskip 20pt \vskip .5\baselineskip\begin{list}{}{}\item{
    \parindent 0pt 
  \vskip 2\baselineskip
  \vbox{\hfil
\vbox{\hrule
  \hbox{\vrule\qquad
    \textcolor{red}{\rule[-.55em]{0pt}{1.7em}\sc Programming Interface}
    \quad\vrule}
  \hrule}
}
  \nobreak\vskip \baselineskip\nobreak
  {
\nobreak\vskip .5\baselineskip 
\halign{\textcolor{red}{\strut\vrule width 5pt\hskip 20pt #} & # \cr
&\cr
& {$\gg\:$\it Convert a capability to a cap set $\ll$}\cr}
\halign{
  
  \textcolor{red}{\strut\vrule width 5pt\hskip 20pt \hfill #} & # \hfil & # & #\hfill & \qquad # \hfill \cr
  &&&& \cr
  $[$ & {\it cs  }:  capset  & $]=$ & {\bf cs$\_$singleton } & \cr
  & & & \hskip 10pt {\it cap  }:  capability  & \cr
  }
\halign{\textcolor{red}{\strut\vrule width 5pt\hskip 20pt #} & # \cr
&\cr
& {$\gg\:$\it Get a useable capability from a capset $\ll$}\cr}
\halign{
  
  \textcolor{red}{\strut\vrule width 5pt\hskip 20pt \hfill #} & # \hfil & # & #\hfill & \qquad # \hfill \cr
  &&&& \cr
  $[$ & {\it cap  }:  capability  & $]=$ & {\bf cs$\_$goodcap } & \cr
  & & & \hskip 10pt {\it cs  }:  capset  & \cr
  }
\halign{\textcolor{red}{\strut\vrule width 5pt\hskip 20pt #} & # \cr
&\cr
& {$\gg\:$\it Get a working capability from a capset $\ll$}\cr}
\halign{
  
  \textcolor{red}{\strut\vrule width 5pt\hskip 20pt \hfill #} & # \hfil & # & #\hfill & \qquad # \hfill \cr
  &&&& \cr
  $[$ & {\it stat  }:  status $*$ & &  & \cr
  & {\it cap  }:  capability  & $]=$ & {\bf cs$\_$to$\_$cap } & \cr
  & & & \hskip 10pt {\it cs  }:  capset  & \cr
  }
\halign{\textcolor{red}{\strut\vrule width 5pt\hskip 20pt #} & # \cr
&\cr
& {$\gg\:$\it Return a fresh capset and copy the original contents $\ll$}\cr}
\halign{
  
  \textcolor{red}{\strut\vrule width 5pt\hskip 20pt \hfill #} & # \hfil & # & #\hfill & \qquad # \hfill \cr
  &&&& \cr
  $[$ & {\it newcs  }:  capset  & $]=$ & {\bf cs$\_$copy } & \cr
  & & & \hskip 10pt {\it origcs  }:  capset  & \cr
  }
\halign{\textcolor{red}{\strut\vrule width 5pt#}&#\cr&\cr}
}
  \vskip 2\baselineskip
  The {\sl cs$\_$goodcap }functions returns the first capability in the set for 
  which the {\sl std$\_$info }request returns the {\sl std$\_$OK }status. If there 
  are no caps in the set for which the {\sl std$\_$info }returns {\sl std$\_$OK }, 
  then the last capability in the set is returned and the error status from the 
  {\sl std$\_$info }request. The {\sl cs$\_$to$\_$cap }functions gets a capability 
  from a capset giving preference to a working capability. If there is only one 
  capability in the set, this capability is returned. If and only if there is more 
  than one, try {\sl std$\_$info }on each of them to obtain one that is useable, 
  and return this one. If none of the multiple capabilities work, the last one is 
  returned. Callers who need to know whether the cap is useable should use {\sl 
  cs$\_$goodcap }function, above. Returns the {\sl std$\_$OK }status, unless the 
  capability set has no capabilities, in which case, it returns the {\sl 
  std$\_$SYSERR }status. }\end{list}}
    \vskip .5\baselineskip
    \vskip .5\baselineskip 
    {\parindent 0pt \vbox{\sc \label{ModuleDependencies}Module Dependencies }
    \leftskip 20pt \rightskip 20pt \vskip .5\baselineskip\begin{list}{}{}\item{
    \parindent 0pt 
  \vskip \baselineskip
  \begin{minipage}{.9\textwidth}\vbox{
\begin{itemize}
  \item Amoeba 
  \item Stderr 
  \end{itemize}
}\end{minipage}
  \vskip \baselineskip
  }\end{list}}
    \vskip .5\baselineskip
    \vfill \eject
    \vbox{\hsize \textwidth
  \hrule
  \vbox{\rule[-.55em]{0pt}{1.7em}\strut \vrule \quad 
\label{Circbuf}\textcolor{blue}{\sc ML-Module: {\sl Circbuf }} \hfill 
\textcolor[rgb]{.0,.7,.0}{\sc Package:  VAM Amoeba System}\quad \vrule}\hrule}
  
    \vskip \baselineskip
    \vskip .5\baselineskip 
    {\parindent 0pt \vbox{\sc Description}
    \leftskip 20pt \rightskip 20pt \vskip .5\baselineskip\begin{list}{}{}\item{
    \parindent 0pt 
  This is the Circular buffer module. \\
  Circular buffers are used to transfer a stream of bytes between a reader and a 
  writer, usually in different threads. The stream is ended after the writer closes 
  the stream; when the reader has read the last byte, the next read call returns an 
  end indicator. Flow control is simple: the reader will block when no data is 
  immediately available, and the writer will block when no buffer space is 
  immediately available. \\
  This module directly supports concurrent access by multiple readers and/or 
  writers. }\end{list}}
    \vskip .5\baselineskip
    \vskip .5\baselineskip 
    {\parindent 0pt \vbox{\sc \label{Structuresandtypes}Structures and types }
    \leftskip 20pt \rightskip 20pt \vskip .5\baselineskip\begin{list}{}{}\item{
    \parindent 0pt 
  \vskip 2\baselineskip
  \vbox{\hfil
\vbox{\hrule
  \hbox{\vrule\qquad
    \textcolor{red}{\rule[-.55em]{0pt}{1.7em}\sc Programming Interface}
    \quad\vrule}
  \hrule}
}
  \nobreak\vskip \baselineskip\nobreak
  {
\nobreak\vskip .5\baselineskip 
\halign{\textcolor{red}{\strut\vrule width 5pt\hskip 20pt #} & # \cr
&\cr
& {$\gg\:$\it The circular buffer structure $\ll$}\cr}
\halign{
  
  \textcolor{red}{\strut\vrule width 5pt\hskip 20pt \hfill #} & # \hfil & # \hfil & #\hfill & \qquad # \hfill \cr
  &&&&\cr
  \textcolor{red}{type} & {\bf circbuf } $=$  $\{$  & 
  &empty : semaphore $;$&{$\gg\:$\it locked if no data in buffer $\ll$}\cr
  && &full : semaphore $;$&{$\gg\:$\it locked if no free space in buffer $\ll$}
  \cr
  && &lock : Mutex.t $;$&{$\gg\:$\it serializes access to other members $\ll$}
  \cr
  && &cbuf : buffer $;$ & \cr
  && &\textcolor{red}{mutable }in$\_$pos : int $;$ & \cr
  && &\textcolor{red}{mutable }out$\_$pos : int $;$ & \cr
  && &last$\_$pos : int $;$ & \cr
  && &\textcolor{red}{mutable }nbytes : int $;$ & \cr
  && &\textcolor{red}{mutable }getpcount : int $;$ & \cr
  && &\textcolor{red}{mutable }putpcount : int $;$ & \cr
  && &\textcolor{red}{mutable }closed : bool  $\}$ &{$\gg\:$\it when set, no 
  new data is accepted $\ll$}\cr
  }
\halign{\textcolor{red}{\strut\vrule width 5pt#}&#\cr&\cr}
}
  \vskip 2\baselineskip
  Sema {\sl empty }is used to block until at least one byte of data becomes 
  available. Similarly, you can block on {\sl full }until at least one free byte 
  becomes available in the buffer. To prevent a deadlock, always acquire {\sl full }
  or {\sl empty }before lock, and never acquire full and empty together. When 
  closed, {\sl full }and {\sl empty }should always be unlocked. Data is between out 
  and in, and wraps from last to first. When in $=$$=$ out, use {\sl nbytes }to see 
  if the buffer is full or empty. }\end{list}}
    \vskip .5\baselineskip
    \vskip .5\baselineskip 
    {\parindent 0pt \vbox{\sc \label{GenericFunctionstocreateandclosecircularbuffers}
    Generic Functions to create and close circular buffers }
    \leftskip 20pt \rightskip 20pt \vskip .5\baselineskip\begin{list}{}{}\item{
    \parindent 0pt 
  \vskip 2\baselineskip
  \vbox{\hfil
\vbox{\hrule
  \hbox{\vrule\qquad
    \textcolor{red}{\rule[-.55em]{0pt}{1.7em}\sc Programming Interface}
    \quad\vrule}
  \hrule}
}
  \nobreak\vskip \baselineskip\nobreak
  {
\nobreak\vskip .5\baselineskip 
\halign{\textcolor{red}{\strut\vrule width 5pt\hskip 20pt #} & # \cr
&\cr
& {$\gg\:$\it Nil cb structure, needed for references on cb's $\ll$}\cr}
\halign{
  
  \textcolor{red}{\strut\vrule width 5pt\hskip 20pt \hfill } # & # \hfil  & \qquad # \cr
  &&\cr
  \textcolor{red}{val} {\bf nilcb }: 
  &  circular$\_$buf  &  \cr
  }
\halign{\textcolor{red}{\strut\vrule width 5pt\hskip 20pt #} & # \cr
&\cr
& {$\gg\:$\it Allocate a new circular buffer of a given size $\ll$}\cr}
\halign{
  
  \textcolor{red}{\strut\vrule width 5pt\hskip 20pt \hfill #} & # \hfil & # & #\hfill & \qquad # \hfill \cr
  &&&& \cr
  $[$ & {\it cb  }:  circular$\_$buf  & $]=$ & {\bf cb$\_$create } & \cr
  & & & \hskip 10pt  $^\sim$size  :  int  & \cr
  }
\halign{\textcolor{red}{\strut\vrule width 5pt\hskip 20pt #} & # \cr
&\cr
& {$\gg\:$\it Set closed flag $\ll$}\cr}
\halign{
  
  \textcolor{red}{\strut\vrule width 5pt\hskip 20pt \hfill #} & # \hfil & # & #\hfill & \qquad # \hfill \cr
  &&&& \cr
  $[$ &  unit  & $]=$ & {\bf cb$\_$close } & \cr
  & & & \hskip 10pt {\it cb  }:  circular$\_$buf  & \cr
  }
\halign{\textcolor{red}{\strut\vrule width 5pt#}&#\cr&\cr}
}
  \vskip 2\baselineskip
  {\sl Cb$\_$close }may be called as often as you want, by readers and writers. 
  Once closed, no new data can be pushed into the buffer, but data already in it is 
  still available to readers. }\end{list}}
    \vskip .5\baselineskip
    \vskip .5\baselineskip 
    {\parindent 0pt \vbox{\sc \label{Functionstogetthestateofacircularbuffer}Functions 
    to get the state of a circular buffer }
    \leftskip 20pt \rightskip 20pt \vskip .5\baselineskip\begin{list}{}{}\item{
    \parindent 0pt 
  \vskip 2\baselineskip
  \vbox{\hfil
\vbox{\hrule
  \hbox{\vrule\qquad
    \textcolor{red}{\rule[-.55em]{0pt}{1.7em}\sc Programming Interface}
    \quad\vrule}
  \hrule}
}
  \nobreak\vskip \baselineskip\nobreak
  {
\nobreak\vskip .5\baselineskip 
\halign{\textcolor{red}{\strut\vrule width 5pt\hskip 20pt #} & # \cr
&\cr
& {$\gg\:$\it Return number of available data bytes in the cb $\ll$}\cr}
\halign{
  
  \textcolor{red}{\strut\vrule width 5pt\hskip 20pt \hfill #} & # \hfil & # & #\hfill & \qquad # \hfill \cr
  &&&& \cr
  $[$ & {\it numbytes  }:  int  & $]=$ & {\bf cb$\_$full } & \cr
  & & & \hskip 10pt {\it cb  }:  circular$\_$buf  & \cr
  }
\halign{\textcolor{red}{\strut\vrule width 5pt\hskip 20pt #} & # \cr
&\cr
& {$\gg\:$\it Return number of available free bytes $\ll$}\cr}
\halign{
  
  \textcolor{red}{\strut\vrule width 5pt\hskip 20pt \hfill #} & # \hfil & # & #\hfill & \qquad # \hfill \cr
  &&&& \cr
  $[$ & {\it numbytes  }:  int  & $]=$ & {\bf cb$\_$empty } & \cr
  & & & \hskip 10pt {\it cb  }:  circular$\_$buf  & \cr
  }
\halign{\textcolor{red}{\strut\vrule width 5pt#}&#\cr&\cr}
}
  \vskip 2\baselineskip
  The {\sl cb$\_$full }function returns $-$1 if there are no bytes available or the 
  circular buffer is closed. The {\sl cb$\_$empty }functions returns $-$1 if the 
  circular buffer is closed. }\end{list}}
    \vskip .5\baselineskip
    \vskip .5\baselineskip 
    {\parindent 0pt \vbox{\sc \label{Functionsforstoringandextractingdata}Functions for 
    storing and extracting data }
    \leftskip 20pt \rightskip 20pt \vskip .5\baselineskip\begin{list}{}{}\item{
    \parindent 0pt 
  \vskip 2\baselineskip
  \vbox{\hfil
\vbox{\hrule
  \hbox{\vrule\qquad
    \textcolor{red}{\rule[-.55em]{0pt}{1.7em}\sc Programming Interface}
    \quad\vrule}
  \hrule}
}
  \nobreak\vskip \baselineskip\nobreak
  {
\nobreak\vskip .5\baselineskip 
\halign{\textcolor{red}{\strut\vrule width 5pt\hskip 20pt #} & # \cr
&\cr
& {$\gg\:$\it Put a char in a circular buffer $\ll$}\cr}
\halign{
  
  \textcolor{red}{\strut\vrule width 5pt\hskip 20pt \hfill #} & # \hfil & # & #\hfill & \qquad # \hfill \cr
  &&&& \cr
  $[$ & {\it status  }:  int  & $]=$ & {\bf cb$\_$putc } & \cr
  & & & \hskip 10pt  $^\sim$circbuf  :  circular$\_$buf $\rightarrow$ & \cr
  & & & \hskip 10pt  $^\sim$chr  :  char  & \cr
  }
\halign{\textcolor{red}{\strut\vrule width 5pt\hskip 20pt #} & # \cr
&\cr
& {$\gg\:$\it Put a string in a circular buffer $\ll$}\cr}
\halign{
  
  \textcolor{red}{\strut\vrule width 5pt\hskip 20pt \hfill #} & # \hfil & # & #\hfill & \qquad # \hfill \cr
  &&&& \cr
  $[$ & {\it status  }:  int  & $]=$ & {\bf cb$\_$puts } & \cr
  & & & \hskip 10pt  $^\sim$circbuf  :  circular$\_$buf $\rightarrow$ & \cr
  & & & \hskip 10pt  $^\sim$str  :  string  & \cr
  }
\halign{\textcolor{red}{\strut\vrule width 5pt\hskip 20pt #} & # \cr
&\cr
& {$\gg\:$\it Get a char from a circular buffer $\ll$}\cr}
\halign{
  
  \textcolor{red}{\strut\vrule width 5pt\hskip 20pt \hfill #} & # \hfil & # & #\hfill & \qquad # \hfill \cr
  &&&& \cr
  $[$ & {\it status  }:  bool $*$ & &  & \cr
  & {\it ch  }:  char  & $]=$ & {\bf cb$\_$getc } & \cr
  & & & \hskip 10pt  circular$\_$buf  & \cr
  }
\halign{\textcolor{red}{\strut\vrule width 5pt\hskip 20pt #} & # \cr
&\cr
& {$\gg\:$\it Try to get a char from a circular buffer $\ll$}\cr}
\halign{
  
  \textcolor{red}{\strut\vrule width 5pt\hskip 20pt \hfill #} & # \hfil & # & #\hfill & \qquad # \hfill \cr
  &&&& \cr
  $[$ & {\it status  }:  bool $*$ & &  & \cr
  & {\it ch  }:  char  & $]=$ & {\bf cb$\_$trygetc } & \cr
  & & & \hskip 10pt  circular$\_$buf  & \cr
  }
\halign{\textcolor{red}{\strut\vrule width 5pt\hskip 20pt #} & # \cr
&\cr
& {$\gg\:$\it Get between minlen and maxlen chars from circbuf $\ll$}\cr}
\halign{
  
  \textcolor{red}{\strut\vrule width 5pt\hskip 20pt \hfill #} & # \hfil & # & #\hfill & \qquad # \hfill \cr
  &&&& \cr
  $[$ & {\it str  }:  string  & $]=$ & {\bf cb$\_$gets } & \cr
  & & & \hskip 10pt  $^\sim$circbuf  :  circular$\_$buf $\rightarrow$ & \cr
  & & & \hskip 10pt  $^\sim$minlen  :  int $\rightarrow$ & \cr
  & & & \hskip 10pt  $^\sim$maxlen  :  int  & \cr
  }
\halign{\textcolor{red}{\strut\vrule width 5pt\hskip 20pt #} & # \cr
&\cr
& {$\gg\:$\it Put a byte value (int) into a circular buffer $\ll$}\cr}
\halign{
  
  \textcolor{red}{\strut\vrule width 5pt\hskip 20pt \hfill #} & # \hfil & # & #\hfill & \qquad # \hfill \cr
  &&&& \cr
  $[$ & {\it status  }:  int  & $]=$ & {\bf cb$\_$putb } & \cr
  & & & \hskip 10pt  $^\sim$circbuf  :  circular$\_$buf $\rightarrow$ & \cr
  & & & \hskip 10pt  $^\sim$byte  :  int  & \cr
  }
\halign{\textcolor{red}{\strut\vrule width 5pt\hskip 20pt #} & # \cr
&\cr
& {$\gg\:$\it Put n bytes into a circular buffer $\ll$}\cr}
\halign{
  
  \textcolor{red}{\strut\vrule width 5pt\hskip 20pt \hfill #} & # \hfil & # & #\hfill & \qquad # \hfill \cr
  &&&& \cr
  $[$ & {\it numbytes  }:  int  & $]=$ & {\bf cb$\_$putbn } & \cr
  & & & \hskip 10pt  $^\sim$circbuf  :  circular$\_$buf $\rightarrow$ & \cr
  & & & \hskip 10pt  $^\sim$srcbuf  :  buffer $\rightarrow$ & \cr
  & & & \hskip 10pt  $^\sim$srcpos  :  int $\rightarrow$ & \cr
  & & & \hskip 10pt  $^\sim$len  :  int  & \cr
  }
\halign{\textcolor{red}{\strut\vrule width 5pt\hskip 20pt #} & # \cr
&\cr
& {$\gg\:$\it Get a byte value (int) from a circular buffer $\ll$}\cr}
\halign{
  
  \textcolor{red}{\strut\vrule width 5pt\hskip 20pt \hfill #} & # \hfil & # & #\hfill & \qquad # \hfill \cr
  &&&& \cr
  $[$ & {\it status  }:  bool $*$ & &  & \cr
  & {\it byte  }:  int  & $]=$ & {\bf cb$\_$getb } & \cr
  & & & \hskip 10pt  $^\sim$circbuf  :  circular$\_$buf  & \cr
  }
\halign{\textcolor{red}{\strut\vrule width 5pt\hskip 20pt #} & # \cr
&\cr
& {$\gg\:$\it Try to get a byte value (int) from a circular buffer $\ll$}\cr}
\halign{
  
  \textcolor{red}{\strut\vrule width 5pt\hskip 20pt \hfill #} & # \hfil & # & #\hfill & \qquad # \hfill \cr
  &&&& \cr
  $[$ & {\it status  }:  bool $*$ & &  & \cr
  & {\it byte  }:  int  & $]=$ & {\bf cb$\_$trygetb } & \cr
  & & & \hskip 10pt  $^\sim$circbuf  :  circular$\_$buf  & \cr
  }
\halign{\textcolor{red}{\strut\vrule width 5pt\hskip 20pt #} & # \cr
&\cr
& {$\gg\:$\it Get n bytes from a circular buffer $\ll$}\cr}
\halign{
  
  \textcolor{red}{\strut\vrule width 5pt\hskip 20pt \hfill #} & # \hfil & # & #\hfill & \qquad # \hfill \cr
  &&&& \cr
  $[$ & {\it numbytes  }:  int  & $]=$ & {\bf cb$\_$getbn } & \cr
  & & & \hskip 10pt  $^\sim$circbuf  :  circular$\_$buf $\rightarrow$ & \cr
  & & & \hskip 10pt  $^\sim$dstbuf  :  buffer $\rightarrow$ & \cr
  & & & \hskip 10pt  $^\sim$dstpos  :  int $\rightarrow$ & \cr
  & & & \hskip 10pt  $^\sim$minlen  :  int $\rightarrow$ & \cr
  & & & \hskip 10pt  $^\sim$maxlen  :  int  & \cr
  }
\halign{\textcolor{red}{\strut\vrule width 5pt#}&#\cr&\cr}
}
  \vskip 2\baselineskip
  The {\sl cb$\_$putc }function returns $-$1 if circular buffer is closed, else 1. 
  The {\sl cb$\_$puts }function returns $-$1 if circular buffer is closed, else the 
  number of written string bytes. The {\sl cb$\_$getc }function returns 
  (false,'$\setminus$000'), if the the circular buffer is closed, else (true,char). 
  The {\sl cb$\_$gets }function is used to extract a string of length {\sl minlen }
  to {\sl maxlen }bytes, and returns the empty string in the case of a closed 
  circular buffer. \\
  The {\sl cb$\_$putb }and {\sl cb$\_$putbn }functions store one bytes or n bytes 
  (int) into a circular buf. Both return $-$1 if cb is closed. The {\sl cb$\_$getb }
  , {\sl cb$\_$trygetb }and {\sl cb$\_$getbn }functions extract one byte or n bytes 
  from a circular buffer. They return the status false if the cb is closed. The 
  {\sl XX$\_$YYYbn }functions use Amoeba buffers either as source or destination. 
  The {\sl XX$\_$trygetY }functions are non blocking functions. They return always 
  immediatley. }\end{list}}
    \vskip .5\baselineskip
    \vskip .5\baselineskip 
    {\parindent 0pt \vbox{\sc \label{Internalfunctions}Internal functions }
    \leftskip 20pt \rightskip 20pt \vskip .5\baselineskip\begin{list}{}{}\item{
    \parindent 0pt 
  \vskip 2\baselineskip
  \vbox{\hfil
\vbox{\hrule
  \hbox{\vrule\qquad
    \textcolor{red}{\rule[-.55em]{0pt}{1.7em}\sc Programming Interface}
    \quad\vrule}
  \hrule}
}
  \nobreak\vskip \baselineskip\nobreak
  {
\nobreak\vskip .5\baselineskip 
\halign{\textcolor{red}{\strut\vrule width 5pt\hskip 20pt #} & # \cr
&\cr
& {$\gg\:$\it Get the position for the next output byte in the cb $\ll$}\cr}
\halign{
  
  \textcolor{red}{\strut\vrule width 5pt\hskip 20pt \hfill #} & # \hfil & # & #\hfill & \qquad # \hfill \cr
  &&&& \cr
  $[$ & {\it num  }:  int $*$ & &  & \cr
  & {\it pos  }:  int  & $]=$ & {\bf cb$\_$getp } & \cr
  & & & \hskip 10pt {\it cb  }:  circular$\_$buf  & \cr
  }
\halign{\textcolor{red}{\strut\vrule width 5pt\hskip 20pt #} & # \cr
&\cr
& {$\gg\:$\it Announce how many bytes are consumed $\ll$}\cr}
\halign{
  
  \textcolor{red}{\strut\vrule width 5pt\hskip 20pt \hfill #} & # \hfil & # & #\hfill & \qquad # \hfill \cr
  &&&& \cr
  $[$ &  unit  & $]=$ & {\bf cb$\_$getpdone } & \cr
  & & & \hskip 10pt {\it cb  }:  circular$\_$buf $\rightarrow$ & \cr
  & & & \hskip 10pt {\it num  }:  int  & \cr
  }
\halign{\textcolor{red}{\strut\vrule width 5pt\hskip 20pt #} & # \cr
&\cr
& {$\gg\:$\it get the position for the next free byte in the cb $\ll$}\cr}
\halign{
  
  \textcolor{red}{\strut\vrule width 5pt\hskip 20pt \hfill #} & # \hfil & # & #\hfill & \qquad # \hfill \cr
  &&&& \cr
  $[$ & {\it num  }:  int $*$ & &  & \cr
  & {\it pos  }:  int  & $]=$ & {\bf cb$\_$putp } & \cr
  & & & \hskip 10pt {\it cb  }:  circular$\_$buf  & \cr
  }
\halign{\textcolor{red}{\strut\vrule width 5pt\hskip 20pt #} & # \cr
&\cr
& {$\gg\:$\it Announce how many bytes were stored $\ll$}\cr}
\halign{
  
  \textcolor{red}{\strut\vrule width 5pt\hskip 20pt \hfill #} & # \hfil & # & #\hfill & \qquad # \hfill \cr
  &&&& \cr
  $[$ &  unit  & $]=$ & {\bf cb$\_$putbdone } & \cr
  & & & \hskip 10pt {\it cb  }:  circular$\_$buf $\rightarrow$ & \cr
  & & & \hskip 10pt {\it num  }:  int  & \cr
  }
\halign{\textcolor{red}{\strut\vrule width 5pt#}&#\cr&\cr}
}
  \vskip 2\baselineskip
  The {\sl cb$\_$getp }function returns ($-$1,$-$1) if cb closed, (0,$-$1) if no 
  bytes available, else (num,pos) of available bytes, but limited to the upper 
  bound of the cb, and the position within the buffer. If nonzero return, a call to 
  {\sl cb$\_$getpdone }must follow to announce how many bytes were actually 
  consumed. The {\sl cb$\_$putb }function returns ($-$1,$-$1) if cb closed, 
  (0,$-$1) if no free bytes available, else (num,pos) of available bytes, but 
  limited to the upper bound of the cb, and the position within the buffer. If 
  nonzero return, a call to {\sl cb$\_$putpdone }must follow to announce how many 
  bytes were actually stored. }\end{list}}
    \vskip .5\baselineskip
    \vskip .5\baselineskip 
    {\parindent 0pt \vbox{\sc \label{ModuleDependencies}Module Dependencies }
    \leftskip 20pt \rightskip 20pt \vskip .5\baselineskip\begin{list}{}{}\item{
    \parindent 0pt 
  \vskip \baselineskip
  \begin{minipage}{.9\textwidth}\vbox{
\begin{itemize}
  \item Amoeba 
  \item Thread 
  \item Mutex 
  \item Sema 
  \end{itemize}
}\end{minipage}
  \vskip \baselineskip
  }\end{list}}
    \vskip .5\baselineskip
    \vfill \eject
    \vbox{\hsize \textwidth
  \hrule
  \vbox{\rule[-.55em]{0pt}{1.7em}\strut \vrule \quad 
\label{Cmdreg}\textcolor{blue}{\sc ML-Module: {\sl Cmdreg }} \hfill 
\textcolor[rgb]{.0,.7,.0}{\sc Package:  VAM Amoeba System}\quad \vrule}\hrule}
  
    \vskip \baselineskip
    \vskip .5\baselineskip 
    {\parindent 0pt \vbox{\sc Description}
    \leftskip 20pt \rightskip 20pt \vskip .5\baselineskip\begin{list}{}{}\item{
    \parindent 0pt 
  This file contains the list of first and last command codes assigned to each 
  registered server. Only registered servers are listed. \\
  The set of error codes is the negative of the command codes. Note that the RPC 
  error codes are in the range RESERVED$\_$FIRST to RESERVED LAST. \\
  Registered commands take numbers in the range: 
  \vskip 2\baselineskip
  \halign{
\hskip3em#\hfil\cr
 \cr
{\tt \hskip 1em1000 to } \cr
{\tt \hskip 1em NON$\_$REGISTERED$\_$FIRST $-$ 1 } \cr
{\tt \hskip 1em \hfil} \cr 
}
  
    \vskip 2\baselineskip
    Developers may use command numbers in the range: 
    \vskip 2\baselineskip
    \halign{
  \hskip3em#\hfil\cr
   \cr
  {\tt \hskip 1emNON$\_$REGISTERED$\_$FIRST to } \cr
  {\tt \hskip 1em NON$\_$REGISTERED$\_$LAST $-$ 1 } \cr
  {\tt \hskip 1em \hfil} \cr 
  }
    
      \vskip 2\baselineskip
      You should make all your command numbers relative to these constants in case they 
      change in Amoeba 4. \\
      Each server is assigned commands in units of 100. If necessary a server may take more 
      two or more consecutive quanta. \\
      Command numbers 1 to 999 are reserved and may NOT be used. The error codes that 
      correspond to these command numbers are for RPC errors. \\
      Command numbers from 1000 to 1999 are reserved for standard commands that all servers 
      should implement where relevant. }\end{list}}
        \vskip .5\baselineskip
        \vskip .5\baselineskip 
        {\parindent 0pt \vbox{\sc \label{Thevalues}The values }
        \leftskip 20pt \rightskip 20pt \vskip .5\baselineskip\begin{list}{}{}\item{
        \parindent 0pt 
      \vskip 2\baselineskip
      \vbox{\hfil
    \vbox{\hrule
  \hbox{\vrule\qquad
\textcolor{red}{\rule[-.55em]{0pt}{1.7em}\sc Programming Interface}\quad\vrule}
  \hrule}
    }
      \nobreak\vskip \baselineskip\nobreak
      {
    \nobreak\vskip .5\baselineskip 
    \halign{\textcolor{red}{\strut\vrule width 5pt\hskip 20pt #} & # \cr
    &\cr
    & {$\gg\:$\it The standard commands that all servers should support $\ll$}\cr}
    \halign{
  
  \textcolor{red}{\strut\vrule width 5pt\hskip 20pt \hfill } # & # \hfil  & \qquad # \cr
  &&\cr
  \textcolor{red}{val} {\bf std$\_$FIRST$\_$COM, std$\_$LAST$\_$COM, 
  std$\_$FIRST$\_$ERR, std$\_$LAST$\_$ERR }: 
  &  int  &  \cr
  }
    \halign{\textcolor{red}{\strut\vrule width 5pt\hskip 20pt #} & # \cr
    &\cr
    & {$\gg\:$\it Directory Server $\ll$}\cr}
    \halign{
  
  \textcolor{red}{\strut\vrule width 5pt\hskip 20pt \hfill } # & # \hfil  & \qquad # \cr
  &&\cr
  \textcolor{red}{val} {\bf sp$\_$FIRST$\_$COM, sp$\_$LAST$\_$COM, 
  sp$\_$FIRST$\_$ERR, sp$\_$LAST$\_$ERR }: 
  &  int  &  \cr
  }
    \halign{\textcolor{red}{\strut\vrule width 5pt\hskip 20pt #} & # \cr
    &\cr
    & {$\gg\:$\it (New) Directory and Name Server $\ll$}\cr}
    \halign{
  
  \textcolor{red}{\strut\vrule width 5pt\hskip 20pt \hfill } # & # \hfil  & \qquad # \cr
  &&\cr
  \textcolor{red}{val} {\bf dns$\_$FIRST$\_$COM, dns$\_$LAST$\_$COM, 
  dns$\_$FIRST$\_$ERR, dns$\_$LAST$\_$ERR }: 
  &  int  &  \cr
  }
    \halign{\textcolor{red}{\strut\vrule width 5pt\hskip 20pt #} & # \cr
    &\cr
    & {$\gg\:$\it Disk server $\ll$}\cr}
    \halign{
  
  \textcolor{red}{\strut\vrule width 5pt\hskip 20pt \hfill } # & # \hfil  & \qquad # \cr
  &&\cr
  \textcolor{red}{val} {\bf disk$\_$FIRST$\_$COM, disk$\_$LAST$\_$COM, 
  disk$\_$FIRST$\_$ERR, disk$\_$LAST$\_$ERR }: 
  &  int  &  \cr
  }
    \halign{\textcolor{red}{\strut\vrule width 5pt\hskip 20pt #} & # \cr
    &\cr
    & {$\gg\:$\it Virtual Circuit Server $\ll$}\cr}
    \halign{
  
  \textcolor{red}{\strut\vrule width 5pt\hskip 20pt \hfill } # & # \hfil  & \qquad # \cr
  &&\cr
  \textcolor{red}{val} {\bf vc$\_$FIRST$\_$COM, vc$\_$LAST$\_$COM, 
  vc$\_$FIRST$\_$ERR, vc$\_$LAST$\_$ERR }: 
  &  int  &  \cr
  }
    \halign{\textcolor{red}{\strut\vrule width 5pt\hskip 20pt #} & # \cr
    &\cr
    & {$\gg\:$\it X Server $\ll$}\cr}
    \halign{
  
  \textcolor{red}{\strut\vrule width 5pt\hskip 20pt \hfill } # & # \hfil  & \qquad # \cr
  &&\cr
  \textcolor{red}{val} {\bf x$\_$FIRST$\_$COM, x$\_$LAST$\_$COM, x$\_$FIRST$\_$ERR, 
  x$\_$LAST$\_$ERR }: 
  &  int  &  \cr
  }
    \halign{\textcolor{red}{\strut\vrule width 5pt#}&#\cr&\cr}
    }
      \vskip 2\baselineskip
      }\end{list}}
        \vskip .5\baselineskip
        \vskip .5\baselineskip 
        {\parindent 0pt \vbox{\sc \label{ModuleDependencies}Module Dependencies }
        \leftskip 20pt \rightskip 20pt \vskip .5\baselineskip\begin{list}{}{}\item{
        \parindent 0pt 
      \vskip \baselineskip
      \begin{minipage}{.9\textwidth}\vbox{
    \begin{itemize}
  \item No special 
  \end{itemize}
    }\end{minipage}
      \vskip \baselineskip
      }\end{list}}
        \vskip .5\baselineskip
        \vfill \eject
        \vbox{\hsize \textwidth
      \hrule
      \vbox{\rule[-.55em]{0pt}{1.7em}\strut \vrule \quad 
    \label{Dblist}\textcolor{blue}{\sc ML-Module: {\sl Dblist }} \hfill 
    \textcolor[rgb]{.0,.7,.0}{\sc Package:  VAM Amoeba System}\quad \vrule}\hrule}
      
        \vskip \baselineskip
        \vskip .5\baselineskip 
        {\parindent 0pt \vbox{\sc Description}
        \leftskip 20pt \rightskip 20pt \vskip .5\baselineskip\begin{list}{}{}\item{
        \parindent 0pt 
      This module provides support for double linked circular lists. The data type can be 
      any type {\it 'a. }}\end{list}}
        \vskip .5\baselineskip
        \vskip .5\baselineskip 
        {\parindent 0pt \vbox{\sc \label{Types}Types }
        \leftskip 20pt \rightskip 20pt \vskip .5\baselineskip\begin{list}{}{}\item{
        \parindent 0pt 
      \vskip 2\baselineskip
      \vbox{\hfil
    \vbox{\hrule
  \hbox{\vrule\qquad
\textcolor{red}{\rule[-.55em]{0pt}{1.7em}\sc Programming Interface}\quad\vrule}
  \hrule}
    }
      \nobreak\vskip \baselineskip\nobreak
      {
    \nobreak\vskip .5\baselineskip 
    \halign{\textcolor{red}{\strut\vrule width 5pt\hskip 20pt #} & # \cr
    &\cr
    & {$\gg\:$\it The dblist node structure $\ll$}\cr}
    \halign{
  
  \textcolor{red}{\strut\vrule width 5pt\hskip 20pt \hfill #} & # \hfil & # \hfil & #\hfill & \qquad # \hfill \cr
  &&&&\cr
  \textcolor{red}{type} & {\bf 'a dblist$\_$node } $=$  $\{$  & 
  &\textcolor{red}{mutable }dbl$\_$data : 'a $;$ & \cr
  && &\textcolor{red}{mutable }dbl$\_$prev : 'a dblist$\_$node $;$ & \cr
  && &\textcolor{red}{mutable }dbl$\_$next : 'a dblist$\_$node  $\}$  & \cr
  }
    \halign{\textcolor{red}{\strut\vrule width 5pt\hskip 20pt #} & # \cr
    &\cr
    & {$\gg\:$\it The double linked list structure $\ll$}\cr}
    \halign{
  
  \textcolor{red}{\strut\vrule width 5pt\hskip 20pt \hfill #} & # \hfil & # \hfil & #\hfill & \qquad # \hfill \cr
  &&&&\cr
  \textcolor{red}{type} & {\bf 'a dblist } $=$  $\{$  & 
  &\textcolor{red}{mutable }dbl$\_$nodes : int $;$&{$\gg\:$\it The number of nodes 
  $\ll$}\cr
  && &\textcolor{red}{mutable }dbl$\_$head : 'a dblist$\_$node option  $\}$ &
  {$\gg\:$\it The list head $\ll$}\cr
  }
    \halign{\textcolor{red}{\strut\vrule width 5pt#}&#\cr&\cr}
    }
      \vskip 2\baselineskip
      }\end{list}}
        \vskip .5\baselineskip
        \vskip .5\baselineskip 
        {\parindent 0pt \vbox{\sc \label{Functions}Functions }
        \leftskip 20pt \rightskip 20pt \vskip .5\baselineskip\begin{list}{}{}\item{
        \parindent 0pt 
      \vskip 2\baselineskip
      \vbox{\hfil
    \vbox{\hrule
  \hbox{\vrule\qquad
\textcolor{red}{\rule[-.55em]{0pt}{1.7em}\sc Programming Interface}\quad\vrule}
  \hrule}
    }
      \nobreak\vskip \baselineskip\nobreak
      {
    \nobreak\vskip .5\baselineskip 
    \halign{\textcolor{red}{\strut\vrule width 5pt\hskip 20pt #} & # \cr
    &\cr
    & {$\gg\:$\it Create a new db list. The first element is not required. $\ll$}\cr}
    \halign{
  
  \textcolor{red}{\strut\vrule width 5pt\hskip 20pt \hfill #} & # \hfil & # & #\hfill & \qquad # \hfill \cr
  &&&& \cr
  $[$ & {\it newlist  }:  'a dblist  & $]=$ & {\bf create } & \cr
  & & & \hskip 10pt  unit  & \cr
  }
    \halign{\textcolor{red}{\strut\vrule width 5pt\hskip 20pt #} & # \cr
    &\cr
    & {$\gg\:$\it Insert a new node before the head node. $\ll$}\cr}
    \halign{
  
  \textcolor{red}{\strut\vrule width 5pt\hskip 20pt \hfill #} & # \hfil & # & #\hfill & \qquad # \hfill \cr
  &&&& \cr
  $[$ &  unit  & $]=$ & {\bf insert$\_$head } & \cr
  & & & \hskip 10pt  $^\sim$dblist  :  'a db$\_$list $\rightarrow$ & \cr
  & & & \hskip 10pt  $^\sim$node  :  'a  & \cr
  }
    \halign{\textcolor{red}{\strut\vrule width 5pt\hskip 20pt #} & # \cr
    &\cr
    & {$\gg\:$\it Insert a new node after the tail node. $\ll$}\cr}
    \halign{
  
  \textcolor{red}{\strut\vrule width 5pt\hskip 20pt \hfill #} & # \hfil & # & #\hfill & \qquad # \hfill \cr
  &&&& \cr
  $[$ &  unit  & $]=$ & {\bf insert$\_$tail } & \cr
  & & & \hskip 10pt  $^\sim$dblist  :  'a db$\_$list $\rightarrow$ & \cr
  & & & \hskip 10pt  $^\sim$node  :  'a  & \cr
  }
    \halign{\textcolor{red}{\strut\vrule width 5pt\hskip 20pt #} & # \cr
    &\cr
    & {$\gg\:$\it Find a node and return the node structure. $\ll$}\cr}
    \halign{
  
  \textcolor{red}{\strut\vrule width 5pt\hskip 20pt \hfill #} & # \hfil & # & #\hfill & \qquad # \hfill \cr
  &&&& \cr
  $[$ & {\it result  }:  'a dblist$\_$node  & $]=$ & {\bf find } & \cr
  & & & \hskip 10pt  $^\sim$f  :  ('a $\rightarrow$ bool) $\rightarrow$ & \cr
  & & & \hskip 10pt  $^\sim$dl  :  'a dblist  & \cr
  }
    \halign{\textcolor{red}{\strut\vrule width 5pt\hskip 20pt #} & # \cr
    &\cr
    & {$\gg\:$\it Find a node and return the data of the node. $\ll$}\cr}
    \halign{
  
  \textcolor{red}{\strut\vrule width 5pt\hskip 20pt \hfill #} & # \hfil & # & #\hfill & \qquad # \hfill \cr
  &&&& \cr
  $[$ & {\it data  }:  'a  & $]=$ & {\bf find$\_$data } & \cr
  & & & \hskip 10pt  $^\sim$f  :  ('a $\rightarrow$ bool) $\rightarrow$ & \cr
  & & & \hskip 10pt  $^\sim$dl  :  'a dblist  & \cr
  }
    \halign{\textcolor{red}{\strut\vrule width 5pt\hskip 20pt #} & # \cr
    &\cr
    & {$\gg\:$\it Remove a node from the list. $\ll$}\cr}
    \halign{
  
  \textcolor{red}{\strut\vrule width 5pt\hskip 20pt \hfill #} & # \hfil & # & #\hfill & \qquad # \hfill \cr
  &&&& \cr
  $[$ &  unit  & $]=$ & {\bf remove } & \cr
  & & & \hskip 10pt  $^\sim$dl  :  'a dblist $\rightarrow$ & \cr
  & & & \hskip 10pt  $^\sim$node  :  'a dblist$\_$node  & \cr
  }
    \halign{\textcolor{red}{\strut\vrule width 5pt\hskip 20pt #} & # \cr
    &\cr
    & {$\gg\:$\it Return the head and remove this node from the list. $\ll$}\cr}
    \halign{
  
  \textcolor{red}{\strut\vrule width 5pt\hskip 20pt \hfill #} & # \hfil & # & #\hfill & \qquad # \hfill \cr
  &&&& \cr
  $[$ & {\it head  }:  'a dblist$\_$node  & $]=$ & {\bf enqueue$\_$head } & \cr
  & & & \hskip 10pt  $^\sim$dl  :  'a dblist  & \cr
  }
    \halign{\textcolor{red}{\strut\vrule width 5pt\hskip 20pt #} & # \cr
    &\cr
    & {$\gg\:$\it Return the tail and remove this node from the list. $\ll$}\cr}
    \halign{
  
  \textcolor{red}{\strut\vrule width 5pt\hskip 20pt \hfill #} & # \hfil & # & #\hfill & \qquad # \hfill \cr
  &&&& \cr
  $[$ & {\it tail  }:  'a dblist$\_$node  & $]=$ & {\bf enqueue$\_$tail } & \cr
  & & & \hskip 10pt  $^\sim$dl  :  'a dblist  & \cr
  }
    \halign{\textcolor{red}{\strut\vrule width 5pt\hskip 20pt #} & # \cr
    &\cr
    & {$\gg\:$\it Iterate the list and apply on each node the function f. $\ll$}\cr}
    \halign{
  
  \textcolor{red}{\strut\vrule width 5pt\hskip 20pt \hfill #} & # \hfil & # & #\hfill & \qquad # \hfill \cr
  &&&& \cr
  $[$ &  unit  & $]=$ & {\bf iter } & \cr
  & & & \hskip 10pt  $^\sim$f  :  ('a $\rightarrow$ unit) $\rightarrow$ & \cr
  & & & \hskip 10pt  $^\sim$dl  :  'a dblist  & \cr
  }
    \halign{\textcolor{red}{\strut\vrule width 5pt#}&#\cr&\cr}
    }
      \vskip 2\baselineskip
      The user defined function {\tt f:'a $\rightarrow$ bool }given to the {\sl find }
      function must select the right one node. }\end{list}}
        \vskip .5\baselineskip
        \vfill \eject
        \vbox{\hsize \textwidth
      \hrule
      \vbox{\rule[-.55em]{0pt}{1.7em}\strut \vrule \quad 
    \label{DNS:DirectoryandNameService}\textcolor{blue}{\sc {\sl DNS: Directory and 
    Name Service }} \hfill
    \quad \vrule}\hrule}
      
        \vskip \baselineskip
        \vskip .5\baselineskip 
        {\parindent 0pt \vbox{\sc Description}
        \leftskip 20pt \rightskip 20pt \vskip .5\baselineskip\begin{list}{}{}\item{
        \parindent 0pt 
      This is the Directory and Name server environment. It's splitted in these modules: 
      \vskip \baselineskip
      \begin{minipage}{.9\textwidth}\vbox{
    \begin{enumerate}
  \item The common module {\sl dns$\_$common }$-$ both for clients and servers 
  \item The server implementation module {\sl dns$\_$server }
  \item The client implementation module {\sl dns$\_$client }
  \item The RPC server loop module {\sl dns$\_$server$\_$rpc }
  \end{enumerate}
    }\end{minipage}
      \vskip \baselineskip
      }\end{list}}
        \vskip .5\baselineskip
        \vskip .5\baselineskip 
        {\parindent 0pt \vbox{\sc \label{Common:Interfacerequests}Common: Interface requests }
        \leftskip 20pt \rightskip 20pt \vskip .5\baselineskip\begin{list}{}{}\item{
        \parindent 0pt 
      \vskip 2\baselineskip
      \begin{center}
    \begin{tabular}{|p{6.6cm}|p{6.6cm}|}\hline
  Name &Description \\\hline
  {\tt dns$\_$LOOKUP }&Traverse a path as far as possible, and return the resulting 
  capability set and the rest of the path. \\\hline
  {\tt dns$\_$SETLOOKUP }&Lookup rownames in a set of directories. \\\hline
  {\tt dns$\_$GETMASKS }&Return the rights masks in a row. \\\hline
  {\tt dns$\_$DISCARD }&Destroy a directory. Simple. Only allow this if the 
  directory is empty. \\\hline
  {\tt dns$\_$LIST }&List a directory. Returns a flattened representation of the 
  number of columns, the number of rows, the names of the columns, the names of the 
  rows and the right masks. hdr$\rightarrow$h$\_$extra indicates the desired 
  starting row, on input, and indicates the first row not returned (due to lack of 
  space), on output. If all rows were returned, hdr$\rightarrow$h$\_$extra is set 
  to a special value to tell the client that another transaction is not necessary. 
  \\\hline
  {\tt dns$\_$CREATE }&Create a directory. The request contains the names of the 
  columns. By counting the names, we know the number of columns. \\\hline
  {\tt dns$\_$APPEND }&Append a row to a directory. The name, right masks, and 
  initial capability set is specified. A lot of work, but rather simple. \\\hline
  {\tt dns$\_$REPLACE }&Replace a capability set. The name and new capability set 
  is specified. \\\hline
  {\tt dns$\_$CHMOD }&Change the rights masks in a row. \\\hline
  {\tt dns$\_$DELETE }&Delete a row. \\\hline
  {\tt dns$\_$INSTALL }&Update a set of directory entries. All entries have to be 
  at this directory service or it won't work. Specified are capability sets for 
  directories, and names within those directories (simple names, no pathnames). 
  Moreover, an old capability can be specified which has to be in the current 
  capability set for the update to succeed. \\\hline
  {\tt dns$\_$GETSEQNR }&Return the sequence number of the directory. \\\hline
  {\tt dns$\_$GETDEFAULTFS }&Get the default file server, if any. \\\hline
  \end{tabular}
    \end{center}
      \vskip 2\baselineskip
      }\end{list}}
        \vskip .5\baselineskip
        \vfill \eject
        \vbox{\hsize \textwidth
      \hrule
      \vbox{\rule[-.55em]{0pt}{1.7em}\strut \vrule \quad 
    \label{Dns$_$client}\textcolor{blue}{\sc ML-Module: {\sl Dns$\_$client }} \hfill 
    \textcolor[rgb]{.0,.7,.0}{\sc Package:  VAM Amoeba System}\quad \vrule}\hrule}
      
        \vskip \baselineskip
        \vskip .5\baselineskip 
        {\parindent 0pt \vbox{\sc Description}
        \leftskip 20pt \rightskip 20pt \vskip .5\baselineskip\begin{list}{}{}\item{
        \parindent 0pt 
      This is the DNS client module. It provides functions to lookup and modify exisiting 
      directories, delete or extract rows (directory entries). }\end{list}}
        \vskip .5\baselineskip
        \vskip .5\baselineskip 
        {\parindent 0pt \vbox{\sc \label{dns$_$LOOKUP}dns$\_$LOOKUP }
        \leftskip 20pt \rightskip 20pt \vskip .5\baselineskip\begin{list}{}{}\item{
        \parindent 0pt 
      This request is used to get a capability set for a path relative to a root directory. 
      A root directory can be {\bf any }directory in the directory tree. The functions 
      loops over the path components step by step and resolve the path. In each run, the 
      next path component is the new parent directory for the next lookup (to another DNS 
      server). 
      \vskip 2\baselineskip
      \vbox{\hfil
    \vbox{\hrule
  \hbox{\vrule\qquad
\textcolor{red}{\rule[-.55em]{0pt}{1.7em}\sc Programming Interface}\quad\vrule}
  \hrule}
    }
      \nobreak\vskip \baselineskip\nobreak
      {
    \nobreak\vskip .5\baselineskip 
    \halign{
  
  \textcolor{red}{\strut\vrule width 5pt\hskip 20pt \hfill #} & # \hfil & # & #\hfill & \qquad # \hfill \cr
  &&&& \cr
  $[$ & {\it rpc$\_$status  }:  status $*$ & &  & \cr
  & {\it cs  }:  capset  & $]=$ & {\bf lookup } & \cr
  & & & \hskip 10pt  $^\sim$root  :  capset $\rightarrow$ & \cr
  & & & \hskip 10pt  $^\sim$path  :  string  & \cr
  }
    \halign{\textcolor{red}{\strut\vrule width 5pt#}&#\cr&\cr}
    }
      \vskip 2\baselineskip
      Client$-$Server transaction: 
      \vskip 2\baselineskip
      \begin{center}
    \begin{tabular}{|p{9.6cm}|}\hline
  h$\_$command $\leftarrow$ dns$\_$LOOKUP \\\hline
  h$\_$port,h$\_$priv $\leftarrow$ [dir$\_$cs].cap $_{\rm i\quad}$\\\hline
  h$\_$size $\leftarrow$ in$\_$size \\\hline
  inbuf $\leftarrow$ buf$\_$put$\_$string path \\\hline
  outbuf \\\hline
  \end{tabular}
    \end{center}
      \vskip 2\baselineskip
      The outbuf structure: 
      \vskip 2\baselineskip
      \begin{center}
    \begin{tabular}{|p{7.5cm}|}\hline
  path $\rightarrow$ buf$\_$get$\_$string \\\hline
  capset $\rightarrow$ buf$\_$get$\_$capset \\\hline
  \end{tabular}
    \end{center}
      \vskip 2\baselineskip
      In the next iteration, capset and path are the new components if path was not fully 
      resolved by the last server. }\end{list}}
        \vskip .5\baselineskip
        \vfill \eject
        \vbox{\hsize \textwidth
      \hrule
      \vbox{\rule[-.55em]{0pt}{1.7em}\strut \vrule \quad 
    \label{Dns$_$server}\textcolor{blue}{\sc ML-Module: {\sl Dns$\_$server }} \hfill 
    \textcolor[rgb]{.0,.7,.0}{\sc Package:  VAM Amoeba System}\quad \vrule}\hrule}
      
        \vskip \baselineskip
        \vskip .5\baselineskip 
        {\parindent 0pt \vbox{\sc Description}
        \leftskip 20pt \rightskip 20pt \vskip .5\baselineskip\begin{list}{}{}\item{
        \parindent 0pt 
      This is the server side implemenatation of the Directory and Name service DNS. This 
      module provides structures and basic functions to build a DNS server. }\end{list}}
        \vskip .5\baselineskip
        \vskip .5\baselineskip 
        {\parindent 0pt \vbox{\sc \label{BasicStructures}Basic Structures }
        \leftskip 20pt \rightskip 20pt \vskip .5\baselineskip\begin{list}{}{}\item{
        \parindent 0pt 
      One row of a directory ($=$ directory entry). 
      \vskip 2\baselineskip
      \vbox{\hfil
    \vbox{\hrule
  \hbox{\vrule\qquad
\textcolor{red}{\rule[-.55em]{0pt}{1.7em}\sc Programming Interface}\quad\vrule}
  \hrule}
    }
      \nobreak\vskip \baselineskip\nobreak
      {
    \nobreak\vskip .5\baselineskip 
    \halign{
  
  \textcolor{red}{\strut\vrule width 5pt\hskip 20pt \hfill #} & # \hfil & # \hfil & #\hfill & \qquad # \hfill \cr
  &&&&\cr
  \textcolor{red}{type} & {\bf dns$\_$row } $=$  $\{$  & 
  &\textcolor{red}{mutable }dr$\_$name : string $;$&{$\gg\:$\it The row name $\ll$}
  \cr
  && &\textcolor{red}{mutable }dr$\_$time : int $;$&{$\gg\:$\it Time stamp $\ll$}\cr
  && &\textcolor{red}{mutable }dr$\_$columns : rights$\_$bits array $;$&{$\gg\:$\it 
  The rights mask $\ll$}\cr
  && &\textcolor{red}{mutable }dr$\_$capset : capset  $\}$ &{$\gg\:$\it Row 
  capability set ot the object $\ll$}\cr
  }
    \halign{\textcolor{red}{\strut\vrule width 5pt#}&#\cr&\cr}
    }
      \vskip 2\baselineskip
      One DNS table entry ($=$ directory). All rows are stored in a double linked list. 
      \vskip 2\baselineskip
      \vbox{\hfil
    \vbox{\hrule
  \hbox{\vrule\qquad
\textcolor{red}{\rule[-.55em]{0pt}{1.7em}\sc Programming Interface}\quad\vrule}
  \hrule}
    }
      \nobreak\vskip \baselineskip\nobreak
      {
    \nobreak\vskip .5\baselineskip 
    \halign{\textcolor{red}{\strut\vrule width 5pt\hskip 20pt #} & # \cr
    &\cr
    & {$\gg\:$\it The state of a directory $\ll$}\cr}
    \halign{
  
  \textcolor{red}{\strut\vrule width 5pt\hskip 20pt \hfill #} & # \hfil & # \hfil & #\hfill & \qquad # \hfill \cr
  &&&&\cr
  \textcolor{red}{type} & {\bf dns$\_$dir$\_$state } $=$  & 
  &DD$\_$cached &{$\gg\:$\it Read from disk and unmodified $\ll$}\cr
  &&  $|$ &DD$\_$modified &{$\gg\:$\it Read from disk and modified $\ll$}\cr
  &&  $|$ &DD$\_$new &{$\gg\:$\it New directory, currently not written to disk 
  $\ll$}\cr
  }
    \halign{
  
  \textcolor{red}{\strut\vrule width 5pt\hskip 20pt \hfill #} & # \hfil & # \hfil & #\hfill & \qquad # \hfill \cr
  &&&&\cr
  \textcolor{red}{type} & {\bf dns$\_$dir } $=$  $\{$  & 
  &\textcolor{red}{mutable }dd$\_$lock : Mutex.t $;$&{$\gg\:$\it Directory lock 
  $\ll$}\cr
  && &\textcolor{red}{mutable }dd$\_$objnum : int $;$&{$\gg\:$\it The directory 
  index number $\ll$}\cr
  && &\textcolor{red}{mutable }dd$\_$ncols : int $;$&{$\gg\:$\it Number of columns 
  $\ll$}\cr
  && &\textcolor{red}{mutable }dd$\_$nrows : int $;$&{$\gg\:$\it Number of rows in 
  this dir $\ll$}\cr
  && &\textcolor{red}{mutable }dd$\_$colnames : string array $;$&{$\gg\:$\it The 
  column names $\ll$}\cr
  && &\textcolor{red}{mutable }dd$\_$random : port $;$&{$\gg\:$\it Random check 
  number $\ll$}\cr
  && &\textcolor{red}{mutable }dd$\_$rows : (dns$\_$row dblist) option $;$&
  {$\gg\:$\it The rows $\ll$}\cr
  && &\textcolor{red}{mutable }dd$\_$state : dns$\_$dir state $;$&{$\gg\:$\it 
  Status of the directory $\ll$}\cr
  && &\textcolor{red}{mutable }dd$\_$time : int $;$&{$\gg\:$\it Time stamp $\ll$}\cr
  && &\textcolor{red}{mutable }dd$\_$live : int  $\}$ &{$\gg\:$\it Live time 
  [0..MAXLIVE] $\ll$}\cr
  }
    \halign{\textcolor{red}{\strut\vrule width 5pt#}&#\cr&\cr}
    }
      \vskip 2\baselineskip
      The main DNS structure: the all known super structure with basic informations about 
      the file system. 
      \vskip 2\baselineskip
      \vbox{\hfil
    \vbox{\hrule
  \hbox{\vrule\qquad
\textcolor{red}{\rule[-.55em]{0pt}{1.7em}\sc Programming Interface}\quad\vrule}
  \hrule}
    }
      \nobreak\vskip \baselineskip\nobreak
      {
    \nobreak\vskip .5\baselineskip 
    \halign{
  
  \textcolor{red}{\strut\vrule width 5pt\hskip 20pt \hfill #} & # \hfil & # \hfil & #\hfill & \qquad # \hfill \cr
  &&&&\cr
  \textcolor{red}{type} & {\bf dns$\_$super } $=$  $\{$  & 
  &\textcolor{red}{mutable }dns$\_$name : int $;$&{$\gg\:$\it The server name $\ll$}
  \cr
  && &\textcolor{red}{mutable }dns$\_$ndirs : int $;$&{$\gg\:$\it Number of total 
  table entries $\ll$}\cr
  && &\textcolor{red}{mutable }dns$\_$nused : int $;$&{$\gg\:$\it Number of used 
  table entries $\ll$}\cr
  && &\textcolor{red}{mutable }dns$\_$freeobjnums : int list $;$&{$\gg\:$\it Free 
  slots list $\ll$}\cr
  && &\textcolor{red}{mutable }dns$\_$newmodified : int list $;$&{$\gg\:$\it 
  New/modified slots list $\ll$}\cr
  && &\textcolor{red}{mutable }dns$\_$nextfree : int $;$&{$\gg\:$\it Next free slot 
  $\ll$}\cr
  && &\textcolor{red}{mutable }dns$\_$getport : port $;$&{$\gg\:$\it Private 
  getport $\ll$}\cr
  && &\textcolor{red}{mutable }dns$\_$putport : port $;$&{$\gg\:$\it Public putport 
  $\ll$}\cr
  && &\textcolor{red}{mutable }dns$\_$checkfield : port $;$&{$\gg\:$\it Private 
  checkfield $\ll$}\cr
  && &\textcolor{red}{mutable }dns$\_$ncols : int $;$&{$\gg\:$\it Number of columns 
  $\ll$}\cr
  && &\textcolor{red}{mutable }dns$\_$col$\_$names : string array $;$&{$\gg\:$\it 
  Column names $\ll$}\cr
  && &\textcolor{red}{mutable }dns$\_$generic$\_$col$\_$maks : rights$\_$bits array 
  $;$&{$\gg\:$\it Column mask $\ll$}\cr
  && &\textcolor{red}{mutable }dns$\_$fs$\_$Server : fs$\_$server  $\}$ &
  {$\gg\:$\it File server $\ll$}\cr
  }
    \halign{\textcolor{red}{\strut\vrule width 5pt#}&#\cr&\cr}
    }
      \vskip 2\baselineskip
      The server structure. 
      \vskip 2\baselineskip
      \vbox{\hfil
    \vbox{\hrule
  \hbox{\vrule\qquad
\textcolor{red}{\rule[-.55em]{0pt}{1.7em}\sc Programming Interface}\quad\vrule}
  \hrule}
    }
      \nobreak\vskip \baselineskip\nobreak
      {
    \nobreak\vskip .5\baselineskip 
    \halign{
  
  \textcolor{red}{\strut\vrule width 5pt\hskip 20pt \hfill #} & # \hfil & # \hfil & #\hfill & \qquad # \hfill \cr
  &&&&\cr
  \textcolor{red}{type} & {\bf dns$\_$server } $=$  $\{$  & 
  &\textcolor{red}{mutable }dns$\_$lock : Mutex.t $;$&{$\gg\:$\it Super protection 
  lock $\ll$}\cr
  && &\textcolor{red}{mutable }dns$\_$super : dns$\_$super option $;$&{$\gg\:$\it 
  The super structure $\ll$}\cr
  && &\textcolor{red}{mutable }dns$\_$read$\_$dir : index:int $\rightarrow$ 
  dns$\_$dir$*$status $;$ & \cr
  && &\textcolor{red}{mutable }dns$\_$write$\_$dir : dir:dns$\_$dir $\rightarrow$ 
  status $;$ & \cr
  && &\textcolor{red}{mutable }dns$\_$create$\_$dir : dir:dns$\_$dir $\rightarrow$ 
  status $;$ & \cr
  && &\textcolor{red}{mutable }dns$\_$delete$\_$dir : dir:dns$\_$dir $\rightarrow$ 
  status $;$ & \cr
  && &\textcolor{red}{mutable }dns$\_$read$\_$super : unit $\rightarrow$ 
  dns$\_$super$*$status $;$ & \cr
  && &\textcolor{red}{mutable }dns$\_$write$\_$super : super:dns$\_$super 
  $\rightarrow$ status $;$ & \cr
  && &\textcolor{red}{mutable }dns$\_$sync : unit $\rightarrow$ status  $\}$  & \cr
  }
    \halign{\textcolor{red}{\strut\vrule width 5pt#}&#\cr&\cr}
    }
      \vskip 2\baselineskip
      The {\sl dns$\_$read$\_$dir }and {\sl dns$\_$write$\_$dir }functions are used to read 
      and write single directories. The {\sl dns$\_$create$\_$dir }and {\sl 
      dns$\_$delete$\_$dir }functions are used to add and delete directories. The {\sl 
      dns$\_$read$\_$super }and {\sl dns$\_$write$\_$super }functions are used to read and 
      write the super structure of the file system. And finally, the {\sl dns$\_$sync }
      flushes all caches of the filesystem if any. The implementation of these functions 
      must be provided by the server. \\
      }\end{list}}
        \vskip .5\baselineskip
        \vskip .5\baselineskip 
        {\parindent 0pt \vbox{\sc \label{Values}Values }
        \leftskip 20pt \rightskip 20pt \vskip .5\baselineskip\begin{list}{}{}\item{
        \parindent 0pt 
      The right bit for each column. Indead, only the first dns$\_$MAXCOLUMNS array 
      elements are used. 
      \vskip 2\baselineskip
      \vbox{\hfil
    \vbox{\hrule
  \hbox{\vrule\qquad
\textcolor{red}{\rule[-.55em]{0pt}{1.7em}\sc Programming Interface}\quad\vrule}
  \hrule}
    }
      \nobreak\vskip \baselineskip\nobreak
      {
    \nobreak\vskip .5\baselineskip 
    \halign{
  
  \textcolor{red}{\strut\vrule width 5pt\hskip 20pt \hfill } # & # \hfil  & \qquad # \cr
  &&\cr
  \textcolor{red}{val} {\bf dns$\_$MAXLIVE }: 
  &  int  &  \cr
  }
    \halign{
  
  \textcolor{red}{\strut\vrule width 5pt\hskip 20pt \hfill } # & # \hfil  & \qquad # \cr
  &&\cr
  \textcolor{red}{val} {\bf dns$\_$col$\_$bits }: 
  &  rights$\_$bits array  &  \cr
  }
    \halign{\textcolor{red}{\strut\vrule width 5pt#}&#\cr&\cr}
    }
      \vskip 2\baselineskip
      }\end{list}}
        \vskip .5\baselineskip
        \vskip .5\baselineskip 
        {\parindent 0pt \vbox{\sc \label{Directorytablemanagement}Directory table management }
        \leftskip 20pt \rightskip 20pt \vskip .5\baselineskip\begin{list}{}{}\item{
        \parindent 0pt 
      \vskip 2\baselineskip
      \vbox{\hfil
    \vbox{\hrule
  \hbox{\vrule\qquad
\textcolor{red}{\rule[-.55em]{0pt}{1.7em}\sc Programming Interface}\quad\vrule}
  \hrule}
    }
      \nobreak\vskip \baselineskip\nobreak
      {
    \nobreak\vskip .5\baselineskip 
    \halign{
  
  \textcolor{red}{\strut\vrule width 5pt\hskip 20pt \hfill #} & # \hfil & # & #\hfill & \qquad # \hfill \cr
  &&&& \cr
  $[$ & {\it row  }:  dns$\_$row option  & $]=$ & {\bf dns$\_$search$\_$row } & \cr
  & & & \hskip 10pt  $^\sim$dir  :  dns$\_$dir $\rightarrow$ & \cr
  & & & \hskip 10pt  $^\sim$name  :  string  & \cr
  }
    \halign{
  
  \textcolor{red}{\strut\vrule width 5pt\hskip 20pt \hfill #} & # \hfil & # & #\hfill & \qquad # \hfill \cr
  &&&& \cr
  $[$ & {\it dir  }:  dns$\_$dir $*$ & &  & \cr
  & {\it err  }:  status  & $]=$ & {\bf dns$\_$create$\_$dir } & \cr
  & & & \hskip 10pt  $^\sim$server  :  dns$\_$server  & \cr
  }
    \halign{
  
  \textcolor{red}{\strut\vrule width 5pt\hskip 20pt \hfill #} & # \hfil & # & #\hfill & \qquad # \hfill \cr
  &&&& \cr
  $[$ & {\it err  }:  status  & $]=$ & {\bf dns$\_$delete$\_$dir } & \cr
  & & & \hskip 10pt  $^\sim$server  :  dns$\_$server $\rightarrow$ & \cr
  & & & \hskip 10pt  $^\sim$dir  :  dns$\_$dir  & \cr
  }
    \halign{
  
  \textcolor{red}{\strut\vrule width 5pt\hskip 20pt \hfill #} & # \hfil & # & #\hfill & \qquad # \hfill \cr
  &&&& \cr
  $[$ & {\it err  }:  status  & $]=$ & {\bf dns$\_$destroy$\_$dir } & \cr
  & & & \hskip 10pt  $^\sim$server  :  dns$\_$server $\rightarrow$ & \cr
  & & & \hskip 10pt  $^\sim$dir  :  dns$\_$dir  & \cr
  }
    \halign{
  
  \textcolor{red}{\strut\vrule width 5pt\hskip 20pt \hfill #} & # \hfil & # & #\hfill & \qquad # \hfill \cr
  &&&& \cr
  $[$ & {\it row  }:  dns$\_$row  & $]=$ & {\bf dns$\_$create$\_$row } & \cr
  & & & \hskip 10pt  $^\sim$name  :  string $\rightarrow$ & \cr
  & & & \hskip 10pt  $^\sim$cols  :  rights$\_$bits array $\rightarrow$ & \cr
  & & & \hskip 10pt  $^\sim$cs  :  capset  & \cr
  }
    \halign{
  
  \textcolor{red}{\strut\vrule width 5pt\hskip 20pt \hfill #} & # \hfil & # & #\hfill & \qquad # \hfill \cr
  &&&& \cr
  $[$ & {\it err  }:  status  & $]=$ & {\bf dns$\_$append$\_$row } & \cr
  & & & \hskip 10pt  $^\sim$server  :  dns$\_$server $\rightarrow$ & \cr
  & & & \hskip 10pt  $^\sim$dir  :  dns$\_$dir $\rightarrow$ & \cr
  & & & \hskip 10pt  $^\sim$row  :  dns$\_$row  & \cr
  }
    \halign{
  
  \textcolor{red}{\strut\vrule width 5pt\hskip 20pt \hfill #} & # \hfil & # & #\hfill & \qquad # \hfill \cr
  &&&& \cr
  $[$ & {\it err  }:  status  & $]=$ & {\bf dns$\_$delete$\_$row } & \cr
  & & & \hskip 10pt  $^\sim$server  :  dns$\_$server $\rightarrow$ & \cr
  & & & \hskip 10pt  $^\sim$dir  :  dns$\_$dir $\rightarrow$ & \cr
  & & & \hskip 10pt  $^\sim$row  :  dns$\_$row  & \cr
  }
    \halign{
  
  \textcolor{red}{\strut\vrule width 5pt\hskip 20pt \hfill #} & # \hfil & # & #\hfill & \qquad # \hfill \cr
  &&&& \cr
  $[$ & {\it cs  }:  capset  & $]=$ & {\bf capset$\_$of$\_$dir } & \cr
  & & & \hskip 10pt  $^\sim$super  :  dns$\_$super $\rightarrow$ & \cr
  & & & \hskip 10pt  $^\sim$dir  :  dns$\_$dir $\rightarrow$ & \cr
  & & & \hskip 10pt  $^\sim$rights  :  rights$\_$bits  & \cr
  }
    \halign{\textcolor{red}{\strut\vrule width 5pt#}&#\cr&\cr}
    }
      \vskip 2\baselineskip
      The {\sl dns$\_$search$\_$row }function searches a directory for the row name. It 
      returns the row on success, else {\tt None }if the directory doesen't have such a row 
      name. \\
      The {\sl dns$\_$create$\_$dir }function creates a new directory. The function returns 
      the new directory structure and the status returned by the servers {\sl 
      dns$\_$dir$\_$create }and {\sl dns$\_$write$\_$super }functions. The super structure 
      is modified, too. \\
      The {\sl dns$\_$delete$\_$dir }functions deletes an empty directory. In contrast, the 
      {\sl dns$\_$destroy$\_$dir }function deletes a directory, either empty or not. Both 
      functions call the servers {\sl dns$\_$delete$\_$dir }and {\sl dns$\_$write$\_$super }
      functions. \\
      The {\sl dns$\_$create$\_$row }function creates, the {\sl dns$\_$append$\_$row }
      function appends a new created row to an existing directory, and the {\sl 
      dns$\_$delete$\_$row }removes a directory from a directory. The last both ones call 
      the server {\sl dns$\_$write$\_$dir }function. \\
      The {\sl capset$\_$of$\_$dir }function returns a capset with one capability derived 
      from the directory with a private field encoded with the {\tt rights }mask. 
      }\end{list}}
        \vskip .5\baselineskip
        \vskip .5\baselineskip 
        {\parindent 0pt \vbox{\sc \label{Acquireandreleaseadirectory}Acquire and release a 
        directory }
        \leftskip 20pt \rightskip 20pt \vskip .5\baselineskip\begin{list}{}{}\item{
        \parindent 0pt 
      There are two function to acquire a directory object with rights check, and to 
      release the exclusive usage of this directory during a client request. 
      \vskip 2\baselineskip
      \vbox{\hfil
    \vbox{\hrule
  \hbox{\vrule\qquad
\textcolor{red}{\rule[-.55em]{0pt}{1.7em}\sc Programming Interface}\quad\vrule}
  \hrule}
    }
      \nobreak\vskip \baselineskip\nobreak
      {
    \nobreak\vskip .5\baselineskip 
    \halign{
  
  \textcolor{red}{\strut\vrule width 5pt\hskip 20pt \hfill #} & # \hfil & # & #\hfill & \qquad # \hfill \cr
  &&&& \cr
  $[$ & {\it dir  }:  dns$\_$dir $*$ & &  & \cr
  & {\it err  }:  status  & $]=$ & {\bf acquire$\_$dir } & \cr
  & & & \hskip 10pt  $^\sim$server  :  dns$\_$server $\rightarrow$ & \cr
  & & & \hskip 10pt  $^\sim$priv  :  privat $\rightarrow$ & \cr
  & & & \hskip 10pt  $^\sim$req  :  rights$\_$bits  & \cr
  }
    \halign{
  
  \textcolor{red}{\strut\vrule width 5pt\hskip 20pt \hfill #} & # \hfil & # & #\hfill & \qquad # \hfill \cr
  &&&& \cr
  $[$ &  unit  & $]=$ & {\bf release$\_$dir } & \cr
  & & & \hskip 10pt  $^\sim$server  :  dns$\_$server $\rightarrow$ & \cr
  & & & \hskip 10pt  $^\sim$dir  :  dns$\_$dir  & \cr
  }
    \halign{
  
  \textcolor{red}{\strut\vrule width 5pt\hskip 20pt \hfill #} & # \hfil & # & #\hfill & \qquad # \hfill \cr
  &&&& \cr
  $[$ & {\it err  }:  status $*$ & &  & \cr
  & {\it priv  }:  privat  & $]=$ & {\bf get$\_$dir } & \cr
  & & & \hskip 10pt  $^\sim$server  :  dns$\_$server $\rightarrow$ & \cr
  & & & \hskip 10pt  $^\sim$dir$\_$cs  :  capset  & \cr
  }
    \halign{\textcolor{red}{\strut\vrule width 5pt#}&#\cr&\cr}
    }
      \vskip 2\baselineskip
      The {\sl get$\_$dir }function returns the private field of a directory capability 
      set. }\end{list}}
        \vskip .5\baselineskip
        \vskip .5\baselineskip 
        {\parindent 0pt \vbox{\sc \label{Directoryrestriction}Directory restriction }
        \leftskip 20pt \rightskip 20pt \vskip .5\baselineskip\begin{list}{}{}\item{
        \parindent 0pt 
      There is a function to create a restricted version of either a directory capability 
      or of object from other servers. 
      \vskip 2\baselineskip
      \vbox{\hfil
    \vbox{\hrule
  \hbox{\vrule\qquad
\textcolor{red}{\rule[-.55em]{0pt}{1.7em}\sc Programming Interface}\quad\vrule}
  \hrule}
    }
      \nobreak\vskip \baselineskip\nobreak
      {
    \nobreak\vskip .5\baselineskip 
    \halign{
  
  \textcolor{red}{\strut\vrule width 5pt\hskip 20pt \hfill #} & # \hfil & # & #\hfill & \qquad # \hfill \cr
  &&&& \cr
  $[$ & {\it err  }:  status $*$ & &  & \cr
  & {\it cs$\_$restr  }:  capset  & $]=$ & {\bf dns$\_$restrict } & \cr
  & & & \hskip 10pt  $^\sim$server  :  dns$\_$server $\rightarrow$ & \cr
  & & & \hskip 10pt  $^\sim$cs$\_$orig  :  capset $\rightarrow$ & \cr
  & & & \hskip 10pt  $^\sim$mask  :  rights$\_$bits  & \cr
  }
    \halign{\textcolor{red}{\strut\vrule width 5pt#}&#\cr&\cr}
    }
      \vskip 2\baselineskip
      The restricted capability is created with the new rights from the mask value. For 
      foreign objects, the {\sl std$\_$restrict }function is used to perform the 
      restriction. }\end{list}}
        \vskip .5\baselineskip
        \vskip .5\baselineskip 
        {\parindent 0pt \vbox{\sc \label{Clientrequesthandlers}Client request handlers }
        \leftskip 20pt \rightskip 20pt \vskip .5\baselineskip\begin{list}{}{}\item{
        \parindent 0pt 
      \vskip 2\baselineskip
      \vbox{\hfil
    \vbox{\hrule
  \hbox{\vrule\qquad
\textcolor{red}{\rule[-.55em]{0pt}{1.7em}\sc Programming Interface}\quad\vrule}
  \hrule}
    }
      \nobreak\vskip \baselineskip\nobreak
      {
    \nobreak\vskip .5\baselineskip 
    \halign{
  
  \textcolor{red}{\strut\vrule width 5pt\hskip 20pt \hfill #} & # \hfil & # & #\hfill & \qquad # \hfill \cr
  &&&& \cr
  $[$ & {\it err  }:  status $*$ & &  & \cr
  & {\it cs  }:  capset $*$ & &  & \cr
  & {\it path$\_$rest  }:  string  & $]=$ & {\bf dns$\_$req$\_$LOOKUP } & \cr
  & & & \hskip 10pt  $^\sim$server  :  dns$\_$server $\rightarrow$ & \cr
  & & & \hskip 10pt  $^\sim$priv  :  privat $\rightarrow$ & \cr
  & & & \hskip 10pt  $^\sim$path  :  string  & \cr
  }
    \halign{
  
  \textcolor{red}{\strut\vrule width 5pt\hskip 20pt \hfill #} & # \hfil & # & #\hfill & \qquad # \hfill \cr
  &&&& \cr
  $[$ & {\it err  }:  status $*$ & &  & \cr
  & {\it nrwos  }:  int $*$ & &  & \cr
  & {\it ncols  }:  int $*$ & &  & \cr
  & {\it colnames  }:  string array $*$ & &  & \cr
  & {\it [dr$\_$name,dr$\_$columns]  }:  (string $*$ rights$\_$bits array) list 
   & $]=$ & {\bf dns$\_$req$\_$LIST } & \cr
  & & & \hskip 10pt  $^\sim$server  :  dns$\_$server $\rightarrow$ & \cr
  & & & \hskip 10pt  $^\sim$priv  :  privat $\rightarrow$ & \cr
  & & & \hskip 10pt  $^\sim$fristrow  :  int  & \cr
  }
    \halign{
  
  \textcolor{red}{\strut\vrule width 5pt\hskip 20pt \hfill #} & # \hfil & # & #\hfill & \qquad # \hfill \cr
  &&&& \cr
  $[$ & {\it err  }:  status  & $]=$ & {\bf dns$\_$req$\_$APPEND } & \cr
  & & & \hskip 10pt  $^\sim$server  :  dns$\_$server $\rightarrow$ & \cr
  & & & \hskip 10pt  $^\sim$priv  :  privat $\rightarrow$ & \cr
  & & & \hskip 10pt  $^\sim$name  :  string $\rightarrow$ & \cr
  & & & \hskip 10pt  $^\sim$cols  :  rights$\_$bits array $\rightarrow$ & \cr
  & & & \hskip 10pt  $^\sim$capset  :  capset  & \cr
  }
    \halign{
  
  \textcolor{red}{\strut\vrule width 5pt\hskip 20pt \hfill #} & # \hfil & # & #\hfill & \qquad # \hfill \cr
  &&&& \cr
  $[$ & {\it err  }:  status $*$ & &  & \cr
  & {\it newcs  }:  capset  & $]=$ & {\bf dns$\_$req$\_$CREATE } & \cr
  & & & \hskip 10pt  $^\sim$server  :  dns$\_$server $\rightarrow$ & \cr
  & & & \hskip 10pt  $^\sim$priv  :  privat $\rightarrow$ & \cr
  & & & \hskip 10pt  $^\sim$colnames  :  string array  & \cr
  }
    \halign{
  
  \textcolor{red}{\strut\vrule width 5pt\hskip 20pt \hfill #} & # \hfil & # & #\hfill & \qquad # \hfill \cr
  &&&& \cr
  $[$ & {\it err  }:  status  & $]=$ & {\bf dns$\_$req$\_$DISCARD } & \cr
  & & & \hskip 10pt  $^\sim$server  :  dns$\_$server $\rightarrow$ & \cr
  & & & \hskip 10pt  $^\sim$priv  :  privat  & \cr
  }
    \halign{
  
  \textcolor{red}{\strut\vrule width 5pt\hskip 20pt \hfill #} & # \hfil & # & #\hfill & \qquad # \hfill \cr
  &&&& \cr
  $[$ & {\it err  }:  status  & $]=$ & {\bf dns$\_$req$\_$DESTROY } & \cr
  & & & \hskip 10pt  $^\sim$server  :  dns$\_$server $\rightarrow$ & \cr
  & & & \hskip 10pt  $^\sim$priv  :  privat  & \cr
  }
    \halign{
  
  \textcolor{red}{\strut\vrule width 5pt\hskip 20pt \hfill #} & # \hfil & # & #\hfill & \qquad # \hfill \cr
  &&&& \cr
  $[$ & {\it err  }:  status  & $]=$ & {\bf dns$\_$req$\_$CHMOD } & \cr
  & & & \hskip 10pt  $^\sim$server  :  dns$\_$server $\rightarrow$ & \cr
  & & & \hskip 10pt  $^\sim$priv  :  privat $\rightarrow$ & \cr
  & & & \hskip 10pt  $^\sim$cols  :  rights$\_$bits array $\rightarrow$ & \cr
  & & & \hskip 10pt  $^\sim$name  :  string  & \cr
  }
    \halign{
  
  \textcolor{red}{\strut\vrule width 5pt\hskip 20pt \hfill #} & # \hfil & # & #\hfill & \qquad # \hfill \cr
  &&&& \cr
  $[$ & {\it err  }:  status  & $]=$ & {\bf dns$\_$req$\_$DELETE } & \cr
  & & & \hskip 10pt  $^\sim$server  :  dns$\_$server $\rightarrow$ & \cr
  & & & \hskip 10pt  $^\sim$priv  :  privat $\rightarrow$ & \cr
  & & & \hskip 10pt  $^\sim$name  :  string  & \cr
  }
    \halign{
  
  \textcolor{red}{\strut\vrule width 5pt\hskip 20pt \hfill #} & # \hfil & # & #\hfill & \qquad # \hfill \cr
  &&&& \cr
  $[$ & {\it err  }:  status  & $]=$ & {\bf dns$\_$req$\_$REPLACE } & \cr
  & & & \hskip 10pt  $^\sim$server  :  dns$\_$server $\rightarrow$ & \cr
  & & & \hskip 10pt  $^\sim$priv  :  privat $\rightarrow$ & \cr
  & & & \hskip 10pt  $^\sim$name  :  string $\rightarrow$ & \cr
  & & & \hskip 10pt  $^\sim$newcs  :  capset  & \cr
  }
    \halign{
  
  \textcolor{red}{\strut\vrule width 5pt\hskip 20pt \hfill #} & # \hfil & # & #\hfill & \qquad # \hfill \cr
  &&&& \cr
  $[$ & {\it err  }:  status  & $]=$ & {\bf dns$\_$req$\_$TOUCH } & \cr
  & & & \hskip 10pt  $^\sim$server  :  dns$\_$server $\rightarrow$ & \cr
  & & & \hskip 10pt  $^\sim$priv  :  privat  & \cr
  }
    \halign{
  
  \textcolor{red}{\strut\vrule width 5pt\hskip 20pt \hfill #} & # \hfil & # & #\hfill & \qquad # \hfill \cr
  &&&& \cr
  $[$ & {\it err  }:  status $*$ & &  & \cr
  & {\it cols  }:  rights$\_$bits array  & $]=$ & {\bf dns$\_$req$\_$GETMASKS } & 
  \cr
  & & & \hskip 10pt  $^\sim$server  :  dns$\_$server $\rightarrow$ & \cr
  & & & \hskip 10pt  $^\sim$priv  :  privat $\rightarrow$ & \cr
  & & & \hskip 10pt  $^\sim$name  :  string  & \cr
  }
    \halign{
  
  \textcolor{red}{\strut\vrule width 5pt\hskip 20pt \hfill #} & # \hfil & # & #\hfill & \qquad # \hfill \cr
  &&&& \cr
  $[$ & {\it rows  }:  (status $*$ int $*$ capset) list  & $]=$ & {\bf 
  dns$\_$req$\_$SETLOOKUP } & \cr
  & & & \hskip 10pt  $^\sim$server  :  dns$\_$server $\rightarrow$ & \cr
  & & & \hskip 10pt  $^\sim$dirs  :  (capset $*$ string) list  & \cr
  }
    \halign{\textcolor{red}{\strut\vrule width 5pt#}&#\cr&\cr}
    }
      \vskip 2\baselineskip
      {\bf dns$\_$LOOKUP }: \\
      Traverse a path as far as possible, and return the resulting capability set and the 
      rest of the path. \\
      {\bf dns$\_$LIST }: \\
      List a directory. Returns a flattened representation of the number of columns, the 
      number of rows, the names of the columns, the names of the rows and the right masks. 
      Return status, the number of total rows and columns, the col names list, the 
      (dr$\_$name,dr$\_$columns) list starting with firstrow. \\
      {\bf dns$\_$APPEND }: \\
      Append a row to a directory. The name, right masks (cols), and initial capability 
      must be specified. \\
      {\bf dns$\_$CREATE }: \\
      Create a new directory table entry. \\
      {\bf dns$\_$DISCARD }: \\
      Remove a directory. Simple. Only allow this if the directory is empty. The {\sl 
      dns$\_$DELRGT }rights are required to perform this operation. \\
      {\bf dns$\_$DESTROY }: \\
      Destroy a directory $-$ either empty or not. The {\sl dns$\_$DELRGT }rights are 
      required to perform this operation. \\
      {\bf dns$\_$CHMOD }: \\
      Change the rights masks in a row. The {\sl dns$\_$MODRGT }rights are required for 
      this operation. \\
      {\bf dns$\_$DELETE }: \\
      Delete a row within a directory. The {\sl dns$\_$MODRGT }rights are required for this 
      operation. \\
      {\bf dns$\_$REPLACE }: \\
      Replace a capability set. The name and new capability set is specified. The {\sl 
      dns$\_$MODRGT }rights are required for this operation. \\
      {\bf dns$\_$TOUCH }: \\
      Set the live time of a directory to the maximal value. \\
      {\bf dns$\_$GETMASKS }: \\
      Return the rights masks of a row. \\
      {\bf dns$\_$SETLOOKUP }: \\
      Lookup rownames in a set of directories. The 'dirs' argument is a list of 
      (dir$\_$cs,rowname) tuples. Return the resolved rows list with (status,time,capset). 
      }\end{list}}
        \vskip .5\baselineskip
        \vskip .5\baselineskip 
        {\parindent 0pt \vbox{\sc \label{Moduledependencies}Module dependencies }
        \leftskip 20pt \rightskip 20pt \vskip .5\baselineskip\begin{list}{}{}\item{
        \parindent 0pt 
      \vskip \baselineskip
      \begin{minipage}{.9\textwidth}\vbox{
    \begin{itemize}
  \item Amoeba 
  \item Capset 
  \item Buf 
  \end{itemize}
    }\end{minipage}
      \vskip \baselineskip
      }\end{list}}
        \vskip .5\baselineskip
        \vfill \eject
        \vbox{\hsize \textwidth
      \hrule
      \vbox{\rule[-.55em]{0pt}{1.7em}\strut \vrule \quad 
    \label{AFS:AtomicFilesystemService}\textcolor{blue}{\sc {\sl AFS: Atomic Filesystem 
    Service }} \hfill
    \quad \vrule}\hrule}
      
        \vskip \baselineskip
        \vskip .5\baselineskip 
        {\parindent 0pt \vbox{\sc Description}
        \leftskip 20pt \rightskip 20pt \vskip .5\baselineskip\begin{list}{}{}\item{
        \parindent 0pt 
      This is the atomic filesystem environment. It's splitted in these modules: 
      \vskip \baselineskip
      \begin{minipage}{.9\textwidth}\vbox{
    \begin{enumerate}
  \item The common module {\sl afs$\_$common }$-$ both for clients and servers 
  \item The server implementation module {\sl afs$\_$server }
  \item The client implementation module {\sl afs$\_$client }
  \item The RPC server loop module {\sl afs$\_$server$\_$rpc }
  \end{enumerate}
    }\end{minipage}
      \vskip \baselineskip
      }\end{list}}
        \vskip .5\baselineskip
        {\sl Common: Interface requests }
        \vskip 2\baselineskip
        \begin{center}
      \begin{tabular}{|p{6.6cm}|p{6.6cm}|}\hline
    Name &Description \\\hline
    afs$\_$CREATE &Create a new file with an inital content. Either the file is 
    immediately commited or can be modified with further requests. After teh file is 
    commited, it's unmutable. \\\hline
    afs$\_$DELETE &Delete a part of the file. If the file is already commited, a new 
    file will be created and modified, or the original one is used directly. \\\hline
    afs$\_$INSERT &Insert a part into a file. If the file is already commited, a new 
    file will be created and modified, or the original one is used directly. \\\hline
    afs$\_$MODIFY &Modify ($=$replace) a part in a file. If the file is already 
    commited, a new file will be created and modified, or the original one is used 
    directly. \\\hline
    afs$\_$FSCK &Check the file system. \\\hline
    afs$\_$READ &Read a part of a file or the whole file. \\\hline
    afs$\_$SIZE &Get the size of a file. \\\hline
    afs$\_$DISK$\_$COMPACT &Compact the filesystem. \\\hline
    afs$\_$SYNC &Flush the fileserver cache. \\\hline
    \end{tabular}
      \end{center}
        \vskip 2\baselineskip
        \vfill \eject
        \vbox{\hsize \textwidth
      \hrule
      \vbox{\rule[-.55em]{0pt}{1.7em}\strut \vrule \quad 
    \label{Afs$_$common}\textcolor{blue}{\sc ML-Module: {\sl Afs$\_$common }} \hfill 
    \textcolor[rgb]{.0,.7,.0}{\sc Package:  VAM Amoeba System}\quad \vrule}\hrule}
      
        \vskip \baselineskip
        \vskip .5\baselineskip 
        {\parindent 0pt \vbox{\sc Description}
        \leftskip 20pt \rightskip 20pt \vskip .5\baselineskip\begin{list}{}{}\item{
        \parindent 0pt 
      This module is the common part of the atomic filesystem service AFS. The AFS service 
      provides generic files without any names or directory trees. The filesystem consists 
      of a number of file object only referenced with their object numbers, that mean with 
      capabilities. It's really a low level filesystem. After a file was commited, it's 
      atomically, that means unmutable. Each time a file must be modified, a new one is 
      created with a new capabilty. }\end{list}}
        \vskip .5\baselineskip
        \vskip .5\baselineskip 
        {\parindent 0pt \vbox{\sc \label{AFSrequests}AFS requests }
        \leftskip 20pt \rightskip 20pt \vskip .5\baselineskip\begin{list}{}{}\item{
        \parindent 0pt 
      \vskip 2\baselineskip
      \vbox{\hfil
    \vbox{\hrule
  \hbox{\vrule\qquad
\textcolor{red}{\rule[-.55em]{0pt}{1.7em}\sc Programming Interface}\quad\vrule}
  \hrule}
    }
      \nobreak\vskip \baselineskip\nobreak
      {
    \nobreak\vskip .5\baselineskip 
    \halign{\textcolor{red}{\strut\vrule width 5pt\hskip 20pt #} & # \cr
    &\cr
    & {$\gg\:$\it Create a new file $\ll$}\cr}
    \halign{
  
  \textcolor{red}{\strut\vrule width 5pt\hskip 20pt \hfill } # & # \hfil  & \qquad # \cr
  &&\cr
  \textcolor{red}{val} {\bf afs$\_$CREATE }: 
  &  command  &  \cr
  }
    \halign{\textcolor{red}{\strut\vrule width 5pt\hskip 20pt #} & # \cr
    &\cr
    & {$\gg\:$\it Delete a part of a file $\ll$}\cr}
    \halign{
  
  \textcolor{red}{\strut\vrule width 5pt\hskip 20pt \hfill } # & # \hfil  & \qquad # \cr
  &&\cr
  \textcolor{red}{val} {\bf afs$\_$DELETE }: 
  &  command  &  \cr
  }
    \halign{\textcolor{red}{\strut\vrule width 5pt\hskip 20pt #} & # \cr
    &\cr
    & {$\gg\:$\it Insert a part into a file $\ll$}\cr}
    \halign{
  
  \textcolor{red}{\strut\vrule width 5pt\hskip 20pt \hfill } # & # \hfil  & \qquad # \cr
  &&\cr
  \textcolor{red}{val} {\bf afs$\_$INSERT }: 
  &  command  &  \cr
  }
    \halign{\textcolor{red}{\strut\vrule width 5pt\hskip 20pt #} & # \cr
    &\cr
    & {$\gg\:$\it Modify a part of a file $\ll$}\cr}
    \halign{
  
  \textcolor{red}{\strut\vrule width 5pt\hskip 20pt \hfill } # & # \hfil  & \qquad # \cr
  &&\cr
  \textcolor{red}{val} {\bf afs$\_$MODIFY }: 
  &  command  &  \cr
  }
    \halign{\textcolor{red}{\strut\vrule width 5pt\hskip 20pt #} & # \cr
    &\cr
    & {$\gg\:$\it Read a file or a part from it $\ll$}\cr}
    \halign{
  
  \textcolor{red}{\strut\vrule width 5pt\hskip 20pt \hfill } # & # \hfil  & \qquad # \cr
  &&\cr
  \textcolor{red}{val} {\bf afs$\_$READ }: 
  &  command  &  \cr
  }
    \halign{\textcolor{red}{\strut\vrule width 5pt\hskip 20pt #} & # \cr
    &\cr
    & {$\gg\:$\it Flush the file server cache $\ll$}\cr}
    \halign{
  
  \textcolor{red}{\strut\vrule width 5pt\hskip 20pt \hfill } # & # \hfil  & \qquad # \cr
  &&\cr
  \textcolor{red}{val} {\bf afs$\_$SYNC }: 
  &  command  &  \cr
  }
    \halign{\textcolor{red}{\strut\vrule width 5pt\hskip 20pt #} & # \cr
    &\cr
    & {$\gg\:$\it Compact the file system $\ll$}\cr}
    \halign{
  
  \textcolor{red}{\strut\vrule width 5pt\hskip 20pt \hfill } # & # \hfil  & \qquad # \cr
  &&\cr
  \textcolor{red}{val} {\bf afs$\_$DISK$\_$COMPACT }: 
  &  command  &  \cr
  }
    \halign{\textcolor{red}{\strut\vrule width 5pt\hskip 20pt #} & # \cr
    &\cr
    & {$\gg\:$\it Check the file system $\ll$}\cr}
    \halign{
  
  \textcolor{red}{\strut\vrule width 5pt\hskip 20pt \hfill } # & # \hfil  & \qquad # \cr
  &&\cr
  \textcolor{red}{val} {\bf afs$\_$FSCK }: 
  &  command  &  \cr
  }
    \halign{\textcolor{red}{\strut\vrule width 5pt#}&#\cr&\cr}
    }
      \vskip 2\baselineskip
      The only operations permitted on umcommitted files are: 
      \vskip \baselineskip
      \begin{minipage}{.9\textwidth}\vbox{
    \begin{itemize}
  \item {\sl std$\_$destroy }
  \item {\sl afs$\_$modify }
  \item {\sl afs$\_$insert }
  \item {\sl afs$\_$delete }
  \end{itemize}
    }\end{minipage}
      \vskip \baselineskip
      Furthermore, any uncommitted file that remains unmodified for more than {\sl 
      afs$\_$CACHE$\_$TIMEOUT }minutes will be destroyed. The {\sl std$\_$tourch }operation 
      on an uncommitted file is not legal and would not prevent from timing out. The only 
      way to avoid timing out an uncommitted file is to make a null$-$modification to it. 
      }\end{list}}
        \vskip .5\baselineskip
        \vskip .5\baselineskip 
        {\parindent 0pt \vbox{\sc \label{AFSrights}AFS rights }
        \leftskip 20pt \rightskip 20pt \vskip .5\baselineskip\begin{list}{}{}\item{
        \parindent 0pt 
      Access to files and administrative functions is determined on the basis of rights in 
      the capability given with each request. There is a special capability known as the 
      {\it super capability }which is primarily for use by the file system administrator, 
      like filesystem checks or garbage collection. 
      \vskip 2\baselineskip
      \vbox{\hfil
    \vbox{\hrule
  \hbox{\vrule\qquad
\textcolor{red}{\rule[-.55em]{0pt}{1.7em}\sc Programming Interface}\quad\vrule}
  \hrule}
    }
      \nobreak\vskip \baselineskip\nobreak
      {
    \nobreak\vskip .5\baselineskip 
    \halign{\textcolor{red}{\strut\vrule width 5pt\hskip 20pt #} & # \cr
    &\cr
    & {$\gg\:$\it The right to create a file $\ll$}\cr}
    \halign{
  
  \textcolor{red}{\strut\vrule width 5pt\hskip 20pt \hfill } # & # \hfil  & \qquad # \cr
  &&\cr
  \textcolor{red}{val} {\bf afs$\_$RGT$\_$CREATE }: 
  &  command  &  \cr
  }
    \halign{\textcolor{red}{\strut\vrule width 5pt\hskip 20pt #} & # \cr
    &\cr
    & {$\gg\:$\it The right to read, copy or get status of a file $\ll$}\cr}
    \halign{
  
  \textcolor{red}{\strut\vrule width 5pt\hskip 20pt \hfill } # & # \hfil  & \qquad # \cr
  &&\cr
  \textcolor{red}{val} {\bf afs$\_$RGT$\_$READ }: 
  &  command  &  \cr
  }
    \halign{\textcolor{red}{\strut\vrule width 5pt\hskip 20pt #} & # \cr
    &\cr
    & {$\gg\:$\it The right to modify a file $\ll$}\cr}
    \halign{
  
  \textcolor{red}{\strut\vrule width 5pt\hskip 20pt \hfill } # & # \hfil  & \qquad # \cr
  &&\cr
  \textcolor{red}{val} {\bf afs$\_$RGT$\_$MODIFY }: 
  &  command  &  \cr
  }
    \halign{\textcolor{red}{\strut\vrule width 5pt\hskip 20pt #} & # \cr
    &\cr
    & {$\gg\:$\it The right to destroy a file $\ll$}\cr}
    \halign{
  
  \textcolor{red}{\strut\vrule width 5pt\hskip 20pt \hfill } # & # \hfil  & \qquad # \cr
  &&\cr
  \textcolor{red}{val} {\bf afs$\_$RGT$\_$DESTROY }: 
  &  command  &  \cr
  }
    \halign{\textcolor{red}{\strut\vrule width 5pt\hskip 20pt #} & # \cr
    &\cr
    & {$\gg\:$\it Administration rights $\ll$}\cr}
    \halign{
  
  \textcolor{red}{\strut\vrule width 5pt\hskip 20pt \hfill } # & # \hfil  & \qquad # \cr
  &&\cr
  \textcolor{red}{val} {\bf afs$\_$RGT$\_$ADMIN }: 
  &  command  &  \cr
  }
    \halign{\textcolor{red}{\strut\vrule width 5pt\hskip 20pt #} & # \cr
    &\cr
    & {$\gg\:$\it All rights $\ll$}\cr}
    \halign{
  
  \textcolor{red}{\strut\vrule width 5pt\hskip 20pt \hfill } # & # \hfil  & \qquad # \cr
  &&\cr
  \textcolor{red}{val} {\bf afs$\_$RGT$\_$ALL }: 
  &  command  &  \cr
  }
    \halign{\textcolor{red}{\strut\vrule width 5pt#}&#\cr&\cr}
    }
      \vskip 2\baselineskip
      The last two are only of significance in the {\it super capability }. }\end{list}}
        \vskip .5\baselineskip
        \vskip .5\baselineskip 
        {\parindent 0pt \vbox{\sc \label{AFScacheflags}AFS cache flags }
        \leftskip 20pt \rightskip 20pt \vskip .5\baselineskip\begin{list}{}{}\item{
        \parindent 0pt 
      These flags are used to control the cache behaviour for new or modified files. 
      \vskip 2\baselineskip
      \vbox{\hfil
    \vbox{\hrule
  \hbox{\vrule\qquad
\textcolor{red}{\rule[-.55em]{0pt}{1.7em}\sc Programming Interface}\quad\vrule}
  \hrule}
    }
      \nobreak\vskip \baselineskip\nobreak
      {
    \nobreak\vskip .5\baselineskip 
    \halign{\textcolor{red}{\strut\vrule width 5pt\hskip 20pt #} & # \cr
    &\cr
    & {$\gg\:$\it Commit a file after the last request $\ll$}\cr}
    \halign{
  
  \textcolor{red}{\strut\vrule width 5pt\hskip 20pt \hfill } # & # \hfil  & \qquad # \cr
  &&\cr
  \textcolor{red}{val} {\bf afs$\_$COMMIT }: 
  &  int  &  \cr
  }
    \halign{\textcolor{red}{\strut\vrule width 5pt\hskip 20pt #} & # \cr
    &\cr
    & {$\gg\:$\it Write through the cache $\ll$}\cr}
    \halign{
  
  \textcolor{red}{\strut\vrule width 5pt\hskip 20pt \hfill } # & # \hfil  & \qquad # \cr
  &&\cr
  \textcolor{red}{val} {\bf afs$\_$SAFETY }: 
  &  int  &  \cr
  }
    \halign{\textcolor{red}{\strut\vrule width 5pt#}&#\cr&\cr}
    }
      \vskip 2\baselineskip
      }\end{list}}
        \vskip .5\baselineskip
        \vskip .5\baselineskip 
        {\parindent 0pt \vbox{\sc \label{Misc.}Misc. }
        \leftskip 20pt \rightskip 20pt \vskip .5\baselineskip\begin{list}{}{}\item{
        \parindent 0pt 
      \vskip 2\baselineskip
      \vbox{\hfil
    \vbox{\hrule
  \hbox{\vrule\qquad
\textcolor{red}{\rule[-.55em]{0pt}{1.7em}\sc Programming Interface}\quad\vrule}
  \hrule}
    }
      \nobreak\vskip \baselineskip\nobreak
      {
    \nobreak\vskip .5\baselineskip 
    \halign{\textcolor{red}{\strut\vrule width 5pt\hskip 20pt #} & # \cr
    &\cr
    & {$\gg\:$\it The maximal transaction buffer size (server side) $\ll$}\cr}
    \halign{
  
  \textcolor{red}{\strut\vrule width 5pt\hskip 20pt \hfill } # & # \hfil  & \qquad # \cr
  &&\cr
  \textcolor{red}{val} {\bf afs$\_$REQBUFSZ }: 
  &  int  &  \cr
  }
    \halign{\textcolor{red}{\strut\vrule width 5pt#}&#\cr&\cr}
    }
      \vskip 2\baselineskip
      }\end{list}}
        \vskip .5\baselineskip
        \vskip .5\baselineskip 
        {\parindent 0pt \vbox{\sc \label{Moduledependencies}Module dependencies }
        \leftskip 20pt \rightskip 20pt \vskip .5\baselineskip\begin{list}{}{}\item{
        \parindent 0pt 
      \vskip \baselineskip
      \begin{minipage}{.9\textwidth}\vbox{
    \begin{itemize}
  \item Amoeba 
  \end{itemize}
    }\end{minipage}
      \vskip \baselineskip
      }\end{list}}
        \vskip .5\baselineskip
        \vfill \eject
        \vbox{\hsize \textwidth
      \hrule
      \vbox{\rule[-.55em]{0pt}{1.7em}\strut \vrule \quad 
    \label{Afs$_$client}\textcolor{blue}{\sc ML-Module: {\sl Afs$\_$client }} \hfill 
    \textcolor[rgb]{.0,.7,.0}{\sc Package:  VAM Amoeba System}\quad \vrule}\hrule}
      
        \vskip \baselineskip
        \vskip .5\baselineskip 
        {\parindent 0pt \vbox{\sc Description}
        \leftskip 20pt \rightskip 20pt \vskip .5\baselineskip\begin{list}{}{}\item{
        \parindent 0pt 
      This is the client side part of the AFS implementation. }\end{list}}
        \vskip .5\baselineskip
        \vskip .5\baselineskip 
        {\parindent 0pt \vbox{\sc \label{Request}Request }
        \leftskip 20pt \rightskip 20pt \vskip .5\baselineskip\begin{list}{}{}\item{
        \parindent 0pt 
      \vskip 2\baselineskip
      \vbox{\hfil
    \vbox{\hrule
  \hbox{\vrule\qquad
\textcolor{red}{\rule[-.55em]{0pt}{1.7em}\sc Programming Interface}\quad\vrule}
  \hrule}
    }
      \nobreak\vskip \baselineskip\nobreak
      {
    \nobreak\vskip .5\baselineskip 
    \halign{
  
  \textcolor{red}{\strut\vrule width 5pt\hskip 20pt \hfill #} & # \hfil & # & #\hfill & \qquad # \hfill \cr
  &&&& \cr
  $[$ & {\it err  }:  status $*$ & &  & \cr
  & {\it size  }:  int  & $]=$ & {\bf afs$\_$size } & \cr
  & & & \hskip 10pt  $^\sim$cap  :  capability  & \cr
  }
    \halign{
  
  \textcolor{red}{\strut\vrule width 5pt\hskip 20pt \hfill #} & # \hfil & # & #\hfill & \qquad # \hfill \cr
  &&&& \cr
  $[$ & {\it err  }:  status $*$ & &  & \cr
  & {\it newcap  }:  capability  & $]=$ & {\bf afs$\_$create } & \cr
  & & & \hskip 10pt  $^\sim$cap  :  capability $\rightarrow$ & \cr
  & & & \hskip 10pt  $^\sim$buf  :  buffer $\rightarrow$ & \cr
  & & & \hskip 10pt  $^\sim$size  :  int $\rightarrow$ & \cr
  & & & \hskip 10pt  $^\sim$commit  :  int  & \cr
  }
    \halign{
  
  \textcolor{red}{\strut\vrule width 5pt\hskip 20pt \hfill #} & # \hfil & # & #\hfill & \qquad # \hfill \cr
  &&&& \cr
  $[$ & {\it err  }:  status $*$ & &  & \cr
  & {\it newcap  }:  capability  & $]=$ & {\bf afs$\_$delete } & \cr
  & & & \hskip 10pt  $^\sim$cap  :  capability $\rightarrow$ & \cr
  & & & \hskip 10pt  $^\sim$offset  :  int $\rightarrow$ & \cr
  & & & \hskip 10pt  $^\sim$size  :  int $\rightarrow$ & \cr
  & & & \hskip 10pt  $^\sim$commit  :  int  & \cr
  }
    \halign{
  
  \textcolor{red}{\strut\vrule width 5pt\hskip 20pt \hfill #} & # \hfil & # & #\hfill & \qquad # \hfill \cr
  &&&& \cr
  $[$ & {\it err  }:  status $*$ & &  & \cr
  & {\it newcap  }:  capability  & $]=$ & {\bf afs$\_$modify } & \cr
  & & & \hskip 10pt  $^\sim$cap  :  capability $\rightarrow$ & \cr
  & & & \hskip 10pt  $^\sim$buf  :  buffer $\rightarrow$ & \cr
  & & & \hskip 10pt  $^\sim$size  :  int $\rightarrow$ & \cr
  & & & \hskip 10pt  $^\sim$offset  :  int $\rightarrow$ & \cr
  & & & \hskip 10pt  $^\sim$commit  :  int  & \cr
  }
    \halign{
  
  \textcolor{red}{\strut\vrule width 5pt\hskip 20pt \hfill #} & # \hfil & # & #\hfill & \qquad # \hfill \cr
  &&&& \cr
  $[$ & {\it err  }:  status $*$ & &  & \cr
  & {\it newcap  }:  capability  & $]=$ & {\bf afs$\_$insert } & \cr
  & & & \hskip 10pt  $^\sim$cap  :  capability $\rightarrow$ & \cr
  & & & \hskip 10pt  $^\sim$buf  :  buffer $\rightarrow$ & \cr
  & & & \hskip 10pt  $^\sim$size  :  int $\rightarrow$ & \cr
  & & & \hskip 10pt  $^\sim$offset  :  int $\rightarrow$ & \cr
  & & & \hskip 10pt  $^\sim$commit  :  int  & \cr
  }
    \halign{
  
  \textcolor{red}{\strut\vrule width 5pt\hskip 20pt \hfill #} & # \hfil & # & #\hfill & \qquad # \hfill \cr
  &&&& \cr
  $[$ & {\it err  }:  status $*$ & &  & \cr
  & {\it readn  }:  int  & $]=$ & {\bf afs$\_$read } & \cr
  & & & \hskip 10pt  $^\sim$cap  :  capability $\rightarrow$ & \cr
  & & & \hskip 10pt  $^\sim$offset  :  int $\rightarrow$ & \cr
  & & & \hskip 10pt  $^\sim$buf  :  buffer $\rightarrow$ & \cr
  & & & \hskip 10pt  $^\sim$size  :  int  & \cr
  }
    \halign{
  
  \textcolor{red}{\strut\vrule width 5pt\hskip 20pt \hfill #} & # \hfil & # & #\hfill & \qquad # \hfill \cr
  &&&& \cr
  $[$ & {\it err  }:  status  & $]=$ & {\bf afs$\_$sync } & \cr
  & & & \hskip 10pt  $^\sim$server  :  capability  & \cr
  }
    \halign{
  
  \textcolor{red}{\strut\vrule width 5pt\hskip 20pt \hfill #} & # \hfil & # & #\hfill & \qquad # \hfill \cr
  &&&& \cr
  $[$ & {\it err  }:  status  & $]=$ & {\bf afs$\_$fsck } & \cr
  & & & \hskip 10pt  $^\sim$server  :  capability  & \cr
  }
    \halign{
  
  \textcolor{red}{\strut\vrule width 5pt\hskip 20pt \hfill #} & # \hfil & # & #\hfill & \qquad # \hfill \cr
  &&&& \cr
  $[$ & {\it err  }:  status  & $]=$ & {\bf afs$\_$disk$\_$compact } & \cr
  & & & \hskip 10pt  $^\sim$server  :  capability  & \cr
  }
    \halign{\textcolor{red}{\strut\vrule width 5pt#}&#\cr&\cr}
    }
      \vskip 2\baselineskip
      {\bf afs$\_$size }: \\
      {\bf afs$\_$create }: \\
      This request creates a new file whose initial contents is the {\sl size }bytes in the 
      {\sl buf }buffer. The capability is returned in {\sl cap }with the {\sl err }status 
      of the operation. The final size of the file is not specified in advance. The {\sl 
      cap }argument must be either the capability of the AFS server or a valid capability 
      for a commited file on the server. In the latter caes, the new file is then compared 
      with the extant file. If they are the same, it will discard the new file and return 
      the capability for the extant file. Otherwise it will create the new file and return 
      a new capability for it. Note that the capability for an existing file must have the 
      read right for comparison to take place. The capability for the modified file is 
      returned. \\
      \vskip 2\baselineskip
      \begin{center}
    \begin{tabular}{|p{9cm}|}\hline
  \multicolumn{1}{|c|}{\textcolor{red}{\sc {\bf Required rights }}}\\\hline\hline
  {\tt afs$\_$RGT$\_$CREATE }\\\hline
  {\tt afs$\_$RGT$\_$READ }(for comparison) \\\hline
  \end{tabular}
    \end{center}
      \vskip 2\baselineskip
      {\bf afs$\_$modify }: \\
      If the file specified by {\sl cap }has been committed, this request makes an 
      uncommitted copy of it. If the file was uncommitted it is used directly. The file is 
      overwritten with the size {\sl size }bytes from the {\sl buf }buffer, beginning at 
      {\sl offset }bytes from the beginning of the file, not the buffer! \\
      If {\sl offset }+ {\sl size }is greater than the file size, then the file will become 
      greater, but the {\sl offset }can't be greater than the current file size. \\
      To commit a file that is already created without adding anything further to it, 
      simply do a {\sl afs$\_$modify }requets with size 0. If the {\sl afs$\_$COMMIT }flag 
      is not set it will restart the cache timeout. This operation is not atomic unless the 
      amount of data sent to the fileserevr was less than or equal to {\sl afs$\_$REQBUFSZ }
      bytes. If more data than this was to be sent and a failure status is returned it's 
      possible that some part of the operation succeeded. The resultant state of the file 
      is not able to be determined. The capability for the modified file is returned. \\
      \vskip 2\baselineskip
      \begin{center}
    \begin{tabular}{|p{5.1cm}|}\hline
  \multicolumn{1}{|c|}{\textcolor{red}{\sc {\bf Required rights }}}\\\hline\hline
  {\tt afs$\_$RGT$\_$MODIFY }\\\hline
  {\tt afs$\_$RGT$\_$READ }\\\hline
  \end{tabular}
    \end{center}
      \vskip 2\baselineskip
      {\bf afs$\_$insert }: \\
      If the file specified by {\sl cap }has been committed, this request makes an 
      uncommitted copy of it. If the file was uncommitted it is used directly. This request 
      inserts {\sl size }bytes from the {\sl buf }buffer, beginning at {\sl offset }bytes 
      from the beginning of the file, not the buffer! The file size of the file increases 
      by {\sl size }bytes. This operation is not atomic. See above. The capability for the 
      modified file is returned. \\
      \vskip 2\baselineskip
      \begin{center}
    \begin{tabular}{|p{5.1cm}|}\hline
  \multicolumn{1}{|c|}{\textcolor{red}{\sc {\bf Required rights }}}\\\hline\hline
  {\tt afs$\_$RGT$\_$MODIFY }\\\hline
  {\tt afs$\_$RGT$\_$READ }\\\hline
  \end{tabular}
    \end{center}
      \vskip 2\baselineskip
      {\bf afs$\_$delete }: \\
      If the file specified by {\sl cap }has been committed, this request makes an 
      uncommitted copy of it. If the file was uncommitted it is used directly. This request 
      deletes {\sl size }bytes, beginning at {\sl offset }bytes from the beginning of the 
      file. The file size of the file decreases by {\sl size }bytes. The capability for the 
      modified file is returned. \\
      \vskip 2\baselineskip
      \begin{center}
    \begin{tabular}{|p{5.1cm}|}\hline
  \multicolumn{1}{|c|}{\textcolor{red}{\sc {\bf Required rights }}}\\\hline\hline
  {\tt afs$\_$RGT$\_$MODIFY }\\\hline
  {\tt afs$\_$RGT$\_$READ }\\\hline
  \end{tabular}
    \end{center}
      \vskip 2\baselineskip
      {\bf afs$\_$read }: \\
      This request read {\sl size }bytes startinf at {\sl offset }bytes from the beginning 
      of the file. It will return less than {\sl size }bytes if the end of the file is 
      encountered before {\sl size }bytes can be read. The {\sl size }argument can be zero! 
      \vskip 2\baselineskip
      \begin{center}
    \begin{tabular}{|p{5.1cm}|}\hline
  \multicolumn{1}{|c|}{\textcolor{red}{\sc {\bf Required rights }}}\\\hline\hline
  {\tt afs$\_$RGT$\_$READ }\\\hline
  \end{tabular}
    \end{center}
      \vskip 2\baselineskip
      {\bf afs$\_$size }: \\
      This function returns the size of the file specified with the {\sl cap }capability. 
      Need no special rights. \\
      }\end{list}}
        \vskip .5\baselineskip
        \vfill \eject
        \vbox{\hsize \textwidth
      \hrule
      \vbox{\rule[-.55em]{0pt}{1.7em}\strut \vrule \quad \hfill
    \label{MlDoC:ML$$Documentation$$System}\textcolor{blue}{\sc {\sl MlDoC: 
    ML$-$Documentation$-$System }} \hfill
    \quad \vrule}\hrule}
      
        \vskip \baselineskip
        The {\sl MlDoC }system is a powerfull but 'simple as needed' documentation tool for ML 
        programming projects. With {\sl MlDoC }, a document is divided in the document head, 
        sections, subsections, units and subunits. Subunits can be used in all parts up, and 
        the lowest structure elements are generic paragraph elements like lists or text 
        attributes. {\bf This document you are currently reading is of course written and 
        printed with the {\sl MlDoC }System. }\\
        The table below shows an overview about {\sl MlDoC }and it's capabilities. 
        \vskip 2\baselineskip
        \begin{center}
      \begin{tabular}{|p{2.2cm}|p{2.2cm}|p{2.2cm}|p{2.2cm}|p{2.2cm}|p{2.2cm}|}\hline
    \multicolumn{6}{|c|}{\textcolor{red}{\sc Document structure }}\\\hline\hline
    {\bf Document head }&{\bf Sections }&{\bf Subsections }&{\bf Units }&{\bf Subunits }
    &{\bf Paragraph elements }\\\hline
    Title &Package &Module &ML$-$Function &Named Paragraph &Text Attributes \\\hline
    TOC &Program &C$-$Lib &ML$-$Type &Interface &Ordered List \\\hline
    &&&ML$-$Value &Example &Unordered List \\\hline
    &&&ML$-$Class &&Table \\\hline
    &&&ML$-$Module &&Name \\\hline
    &&&C$-$Interface &&Link \\\hline
    &Generic S1 &Generic S2 &Generic S3 &Generic S4 &Special Interfaces \\\hline
    Intro &Intro &Intro &Intro &Intro &\\\hline
    \end{tabular}
      \end{center}
        \vskip 2\baselineskip
        It's not necessary to use all structure depths starting with the document head. For 
        example you can start with the {\sl ML$-$Function }unit in a small standalone document. 
        The units are compareable with common manual pages. \\
        Units are sub document headers, and subunits and introductions can be placed everywhere 
        in the document. \\
        The document structure is organized in the way {\sl OCaML }structures libraries, 
        modules and values, classes and sub modules. But {\sl MlDoC }can also be used to 
        provide documentation for programs, C libraries or other nice things you want to write 
        about. For these purposes, generic sections, units and subunits can be used. \\
        The {\sl Special interface }element is used to show ML$-$function, value, type and sub 
        module interfaces with an unique look, mostly embedded in the {\sl Interface }subunit. 
        \\
        All document commands are preceeded with a dot and followed by one or two uppercase 
        characters. This is a formatting language which is similar to {\sl TROFF }macros. Each 
        dot command must have an antidot command to close a partial environment. For example, 
        an ordered list like 
        \vskip \baselineskip
        \begin{minipage}{.9\textwidth}\vbox{
      \begin{enumerate}
    \item Line 1 
    \item Line 2 
    \item ... 
    \end{enumerate}
      }\end{minipage}
        \vskip \baselineskip
        is coded with the following commands: 
        \vskip 2\baselineskip
        \halign{
      \hskip3em#\hfil\cr
       \cr
      {\tt \hskip 1em.OL } \cr
      {\tt \hskip 1em \hskip 1em .LI Line 1 .IL } \cr
      {\tt \hskip 1em \hskip 1em .LI Line 2 .IL } \cr
      {\tt \hskip 1em \hskip 1em .LI ... .IL } \cr
      {\tt \hskip 1em .LO } \cr
      {\tt \hskip 1em \hfil} \cr 
      }
        
          \vskip 2\baselineskip
          The antidot command is build with the reversed order of the two characters from the dot 
          command. There is one excpetion: text attributes and some special commands. They all have 
          only one uppercase character. 
          \vfill \eject
          \vbox{\hsize \textwidth
        \hrule
        \vbox{\rule[-.55em]{0pt}{1.7em}\strut \vrule \quad \hfill
      \label{MlDocdotcommands}\textcolor{blue}{\sc {\sl MlDoc dot commands }} \hfill
      \quad \vrule}\hrule}
        
          \vskip \baselineskip
          In this section, all available dot commands are described. They are sorted into these 
          groups: 
          \vskip \baselineskip
          \begin{minipage}{.9\textwidth}\vbox{
        \begin{enumerate}
      \item Section headers 
      \item Subsection headers 
      \item Unit headers 
      \item Subunit headers 
      \item Interfaces 
      \item Generic paragraph elements 
      \item Text elements and attributes 
      \end{enumerate}
        }\end{minipage}
          \vskip \baselineskip
          \vfill \eject
          \vbox{\hsize \textwidth
        \hrule
        \vbox{\rule[-.55em]{0pt}{1.7em}\strut \vrule \quad 
      \label{Sectionheaders}\textcolor{blue}{\sc {\sl Section headers }} \hfill 
      \textcolor[rgb]{.0,.7,.0}{\sc  MlDoc dot commands}\quad \vrule}\hrule}
        
          \vskip \baselineskip
          Sections together with subsections are used to structure the complete manual document. 
          \vskip 2\baselineskip
          \begin{center}
        \begin{tabular}{|p{3.3cm}|p{1.8cm}|p{1.5cm}|p{3.3cm}|}\hline
      \multicolumn{4}{|c|}{\textcolor{red}{\sc Available sections }}\\\hline\hline
      {\bf Name }&{\bf begin }&{\bf end }&{\bf Comment }\\\hline
      Package &{\tt .PK }&{\tt .KP }&Name argument required \\\hline
      Program &{\tt .PR }&{\tt .RP }&Name argument required \\\hline
      Generic Section S1 &{\tt .S1 }&{\tt .1S }&Name argument required \\\hline
      \end{tabular}
        \end{center}
          \vskip 2\baselineskip
          The {\sl Name }command must follow the {\sl section }command immediately. Here is a short 
          example for a new {\sl Package }section in the document: 
          \vskip 2\baselineskip
          \vbox{\hfil
        \vbox{\hrule
      \hbox{\vrule\qquad
    \textcolor{blue}{\rule[-.55em]{0pt}{1.7em}\sc Programming Example}\quad\vrule}
      \hrule}
        }
          \nobreak\vskip \baselineskip\nobreak
          \halign to \textwidth{
        \textcolor[rgb]{.58,.76,1.0}{\strut\vrule width 5pt#}\tabskip=0pt plus 1fil& 
          #\hfil& #&\hfil#\hfil& 
          \tabskip=0pt#\cr
         &&&\cr 
        & {\tt \hskip 1em.PK } &&& \cr
        & {\tt \hskip 1em \hskip 1em .NA Standard Library .AN } &&& \cr
        & {\tt \hskip 1em \hskip 1em ... } &&& \cr
        & {\tt \hskip 1em .KP } &&& \cr
        & {\tt \hskip 1em \hfil} &&& \cr 
        }
          
            \vskip 2\baselineskip
            Packages can be ML$-$Libraries or other kinds of programming environments. 
            \vfill \eject
            \vbox{\hsize \textwidth
          \hrule
          \vbox{\rule[-.55em]{0pt}{1.7em}\strut \vrule \quad 
        \label{Subsectionheaders}\textcolor{blue}{\sc {\sl Subsection headers }} \hfill 
        \textcolor[rgb]{.0,.7,.0}{\sc  MlDoc dot commands}\quad \vrule}\hrule}
          
            \vskip \baselineskip
            Subsections are used to group for example functions, values or other ML things. Commonly, a 
            subsection is a part of a section. 
            \vskip 2\baselineskip
            \begin{center}
          \begin{tabular}{|p{3.3cm}|p{1.8cm}|p{1.5cm}|p{3.3cm}|}\hline
        \multicolumn{4}{|c|}{\textcolor{red}{\sc Available subsections }}\\\hline\hline
        {\bf Name }&{\bf begin }&{\bf end }&{\bf Comment }\\\hline
        ML$-$Module &{\tt .MD }&{\tt .DM }&Name argument required \\\hline
        C$-$Library &{\tt .CY }&{\tt .YC }&Name argument required \\\hline
        Generic Subsection S2 &{\tt .S2 }&{\tt .2S }&Name argument required \\\hline
        \end{tabular}
          \end{center}
            \vskip 2\baselineskip
            \vfill \eject
            \vbox{\hsize \textwidth
          \hrule
          \vbox{\rule[-.55em]{0pt}{1.7em}\strut \vrule \quad 
        \label{Unitheaders}\textcolor{blue}{\sc {\sl Unit headers }} \hfill 
        \textcolor[rgb]{.0,.7,.0}{\sc  MlDoc dot commands}\quad \vrule}\hrule}
          
            \vskip \baselineskip
            Units are commonly known as manual pages. They are part of a subsection. 
            \vskip 2\baselineskip
            \begin{center}
          \begin{tabular}{|p{3.3cm}|p{1.8cm}|p{1.5cm}|p{3.3cm}|}\hline
        \multicolumn{4}{|c|}{\textcolor{red}{\sc Available units }}\\\hline\hline
        {\bf Name }&{\bf begin }&{\bf end }&{\bf Comment }\\\hline
        ML$-$Function &{\tt .FU }&{\tt .UF }&Name argument required \\\hline
        ML$-$Type &{\tt .TP }&{\tt .PT }&Name argument required \\\hline
        ML$-$Value &{\tt .MV }&{\tt .VM }&Name argument required \\\hline
        ML$-$Class &{\tt .CL }&{\tt .LC }&Name argument required \\\hline
        ML$-$Module &{\tt .MD }&{\tt .DM }&Name argument required \\\hline
        C$-$Interface &{\tt .CI }&{\tt .IC }&Name argument required \\\hline
        Generic Unit S3 &{\tt .S3 }&{\tt .3S }&Name argument required \\\hline
        \end{tabular}
          \end{center}
            \vskip 2\baselineskip
            \vfill \eject
            \vbox{\hsize \textwidth
          \hrule
          \vbox{\rule[-.55em]{0pt}{1.7em}\strut \vrule \quad 
        \label{Subunits}\textcolor{blue}{\sc {\sl Subunits }} \hfill 
        \textcolor[rgb]{.0,.7,.0}{\sc  MlDoc dot commands}\quad \vrule}\hrule}
          
            \vskip \baselineskip
            Subunits are special paragraphs. They can be used in all structure elements above. 
            \vskip 2\baselineskip
            \begin{center}
          \begin{tabular}{|p{3.3cm}|p{1.8cm}|p{1.5cm}|p{3.3cm}|}\hline
        \multicolumn{4}{|c|}{\textcolor{red}{\sc Available subunits (paragraphs) 
        }}\\\hline\hline
        {\bf Name }&{\bf begin }&{\bf end }&{\bf Comment }\\\hline
        Named Paragraph &{\tt .PA }&{\tt .AP }&Name argument optional \\\hline
        Interface &{\tt .IN }&{\tt .NI }&Name argument optional \\\hline
        Example &{\tt .EX }&{\tt .XE }&Name argument optional \\\hline
        Preformatted text &{\tt .\{ }&{\tt .\} }&\\\hline
        Generic Subunit S4 &{\tt .S4 }&{\tt .4S }&Name argument required \\\hline
        \end{tabular}
          \end{center}
            \vskip 2\baselineskip
            Except for a generic subunit, the name argument {\tt .NA XXX .AN }is optional and can be 
            used for a further description head. \\
            The interface subunit is used to wrap function, value, type or other interfaces. This 
            subunit is used in conjunction with the 
            \vskip \baselineskip
            \begin{minipage}{.9\textwidth}\vbox{
          \begin{itemize}
        \item ML$-$Function 
        \item ML$-$Value 
        \item ML$-$Type (abstract types) 
        \item ML$-$Structure 
        \item ML$-$Module 
        \item ML$-$Class 
        \item C$-$Function/variables 
        \end{itemize}
          }\end{minipage}
            \vskip \baselineskip
            special interfaces. See 
            \fbox{\small {\sc Special Interfaces} (p. \pageref{SpecialInterfaces})} for more details. 
            The {\sl example }and the {\sl preformatted }text subunits are used to display source code 
            or other example lines in a new centered paragraph. The {\sl example }subunit is put in a 
            box. The {\sl preformatted }subunit can spawn several pages. 
            \vfill \eject
            \vbox{\hsize \textwidth
          \hrule
          \vbox{\rule[-.55em]{0pt}{1.7em}\strut \vrule \quad 
        \label{SpecialInterfaces}\textcolor{blue}{\sc {\sl Special Interfaces }} \hfill 
        \textcolor[rgb]{.0,.7,.0}{\sc  MlDoc dot commands}\quad \vrule}\hrule}
          
            \vskip \baselineskip
            The special interfaces are used to show ML or C functions, types, submodules and values. 
            Only the content must be given by the user, not the alignment. Commonly, they are collected 
            in an interface subunit. 
            \vskip 2\baselineskip
            \begin{center}
          \begin{tabular}{|p{3.3cm}|p{1.8cm}|p{1.5cm}|p{3.3cm}|}\hline
        \multicolumn{4}{|c|}{\textcolor{red}{\sc Available special interfaces }}\\\hline\hline
        {\bf Name }&{\bf begin }&{\bf end }&{\bf Subarguments ($*$$=$optional, $\#$$=$multiple) 
        }\\\hline
        ML$-$Function &{\tt .IF }&{\tt .FI }&Name, RetVal ($\#$), Arg ($\#$), Val ($*$$\#$) 
        \\\hline
        ML$-$Value &{\tt .IV }&{\tt .VI }&Name, Arg ($\#$) \\\hline
        ML$-$Type (type list) &{\tt .IT }&{\tt .TI }&Name, Arg ($\#$) \\\hline
        ML$-$Structure (type) &{\tt .IS }&{\tt .SI }&Name, Arg ($*$$\#$) \\\hline
        ML$-$Module &{\tt .IM }&{\tt .MI }&Name, other specials \\\hline
        ML$-$Class &{\tt .CS }&{\tt .SC }&Name, Arg [$*$$\#$], Obj \\\hline
        ML$-$Class method &{\tt .MT }&{\tt .TM }&Name, Arg \\\hline
        C$-$Header &{\tt .CH }&{\tt .HC }&\\\hline
        C$-$Function &{\tt .CF }&{\tt .FC }&Name, Arg ($\#$) , RetVal \\\hline
        C$-$Variable &{\tt .CV }&{\tt .VC }&Name, Arg \\\hline
        \end{tabular}
          \end{center}
            \vskip 2\baselineskip
            The followinf subarguments used only in special interfaces, except the {\sl Name }command 
            and the {\sl Comment }command: 
            \vskip 2\baselineskip
            \begin{center}
          \begin{tabular}{|p{3.3cm}|p{3.3cm}|p{1.8cm}|p{1.5cm}|}\hline
        \multicolumn{4}{|c|}{\textcolor{red}{\sc Available subarguments }}\\\hline\hline
        {\bf Name }&{\bf Description }&{\bf begin }&{\bf end }\\\hline
        Name &Name of function,... &{\tt .NA }&{\tt .AN }\\\hline
        Comment &Placed before the special interface &{\tt .($*$ }&{\tt .$*$) }\\\hline
        RetVal &Return value of a function (uncurried subvalue) &{\tt .RV }&{\tt .VR }\\\hline
        Arg &Argument of a function (curried value) &{\tt .AR }&{\tt .RA }\\\hline
        Val &Value argument of a function (uncurried subvalue) &{\tt .AV }&{\tt .VA }\\\hline
        Obj &Class object &{\tt .OB }&{\tt .BO }\\\hline
        \end{tabular}
          \end{center}
            \vskip 2\baselineskip
            Either the curried value form {\bf or }the uncurried value form must be used for the 
            function interface. \\
            Here is an example for an interface with functions: 
            \vskip 2\baselineskip
            \vbox{\hfil
          \vbox{\hrule
        \hbox{\vrule\qquad
      \textcolor{red}{\rule[-.55em]{0pt}{1.7em}\sc Programming Interface: {\sl Function 
      interfaces }}\quad\vrule}
        \hrule}
          }
            \nobreak\vskip \baselineskip\nobreak
            {
          \nobreak\vskip .5\baselineskip 
          \halign{
        
        \textcolor{red}{\strut\vrule width 5pt\hskip 20pt \hfill #} & # \hfil & # & #\hfill & \qquad # \hfill \cr
        &&&& \cr
        $[$ & {\it ret1 }: int $*$ & & &{$\gg\:$\it The first return argument $\ll$}\cr
        & {\it ret2 }: bool  & $]=$ & {\bf myfun1 }&{$\gg\:$\it The second one. $\ll$}\cr
        & & & \hskip 10pt  $^\sim$arg1 : float $\rightarrow$&{$\gg\:$\it The first fun arg 
        $\ll$}\cr
        & & & \hskip 10pt  $^\sim$arg2 : string list &{$\gg\:$\it The second one. $\ll$}\cr
        }
          \halign{
        
        \textcolor{red}{\strut\vrule width 5pt\hskip 20pt \hfill #} & # \hfil & # & #\hfill & \qquad # \hfill \cr
        &&&& \cr
        $[$ & {\it ret1 }: bool  & $]=$ & {\bf myfun2 } & \cr
        & & & \hskip 10pt {\it arg1 }: float $*$ & \cr
        & & & \hskip 10pt {\it arg2 }: string list  & \cr
        }
          \halign{
        \textcolor{red}{\strut\vrule width 5pt\hskip 20pt \hfill #} & # & #\hfill \cr
        &&\cr
        \textcolor{red}{${\rm\#}$include} &$<$ {\bf sys/io.h }$>$& \cr}
          \halign{
        
        \textcolor{red}{\strut\vrule width 5pt\hskip 20pt \hfill #} & # \hfil & # & #\hfill & \qquad # \hfill \cr
        &&&&\cr
        &int  & {\bf cfun }$($&float arg1 ,  & \cr
        & & &char arg2 $);$ & \cr
        }
          \halign{\textcolor{red}{\strut\vrule width 5pt\hskip 20pt #} & # \cr
          &\cr
          & {$\gg\:$\it The first value interface $\ll$}\cr}
          \halign{
        \textcolor{red}{\strut\vrule width 5pt\hskip 20pt \hfill } # & # \hfil  & \qquad # \cr
        &&\cr
        \textcolor{red}{val} {\bf myval }: 
        & arg1 : string  $\rightarrow$ &{$\gg\:$\it The arg string $\ll$} \cr
        & arg2 : int  $\rightarrow$  &  \cr
        & {\it retarg }: float  &  \cr
        }
          \halign{\textcolor{red}{\strut\vrule width 5pt#}&#\cr&\cr}
          }
            \vskip 2\baselineskip
            The first and the second interfaces show a ML$-$Function, and the third one is a 
            C$-$Function interface. This way to display ML$-$functions is different from the one 
            commonly used in ML interfaces, but more readable. There is also a generic ML$-$value 
            interface, shown in the fourth part of the example. In the second ML$-$function an 
            uncurried value tuple is used for the function argument. Either the curried or the 
            uncurried form must be used. The required programming code to get this beauty 
            ML$-$interfaces is easy to create: 
            \vskip 2\baselineskip
            \halign{
          \hskip3em#\hfil\cr
           \cr
          {\tt \hskip 1em.IN } \cr
          {\tt \hskip 1em \hskip 1em .NA Function interfaces .AN } \cr
          {\tt \hskip 1em \hskip 1em .IF } \cr
          {\tt \hskip 1em \hskip 1em \hskip 1em .NA myfun1 .AN } \cr
          {\tt \hskip 1em \hskip 1em \hskip 1em .RV ret1:int .($*$ The first return argument .$*$) .VR } \cr
          {\tt \hskip 1em \hskip 1em \hskip 1em .RV ret2:bool .($*$ The second one .$*$) .VR } \cr
          {\tt \hskip 1em \hskip 1em \hskip 1em .AR $^\sim$arg1:float .($*$ The first fun arg .$*$) .RA } \cr
          {\tt \hskip 1em \hskip 1em \hskip 1em .AR $^\sim$arg2:string list .($*$ The second one .$*$) .RA } \cr
          {\tt \hskip 1em \hskip 1em .FI } \cr
          {\tt \hskip 1em \hskip 1em .IF } \cr
          {\tt \hskip 1em \hskip 1em \hskip 1em .NA myfun2 .AN } \cr
          {\tt \hskip 1em \hskip 1em \hskip 1em .RV ret1:bool .VR } \cr
          {\tt \hskip 1em \hskip 1em \hskip 1em .AV arg1:float .VA } \cr
          {\tt \hskip 1em \hskip 1em \hskip 1em .AV arg2:string list .VA } \cr
          {\tt \hskip 1em \hskip 1em .FI } \cr
          {\tt \hskip 1em \hskip 1em .CH } \cr
          {\tt \hskip 1em \hskip 1em \hskip 1em sys/io.h } \cr
          {\tt \hskip 1em \hskip 1em .HC } \cr
          {\tt \hskip 1em \hskip 1em .CF } \cr
          {\tt \hskip 1em \hskip 1em \hskip 1em .NA cfun .AN } \cr
          {\tt \hskip 1em \hskip 1em \hskip 1em .RV int .VR } \cr
          {\tt \hskip 1em \hskip 1em \hskip 1em .AR float arg1 .RA } \cr
          {\tt \hskip 1em \hskip 1em \hskip 1em .AR char arg2 .RA } \cr
          {\tt \hskip 1em \hskip 1em .FC } \cr
          {\tt \hskip 1em \hskip 1em .IV } \cr
          {\tt \hskip 1em \hskip 1em \hskip 1em .($*$ The first value interface .$*$) } \cr
          {\tt \hskip 1em \hskip 1em \hskip 1em .NA myval .AN } \cr
          {\tt \hskip 1em \hskip 1em \hskip 1em .AR $^\sim$arg1:string .($*$ The arg string .$*$) .RA } \cr
          {\tt \hskip 1em \hskip 1em \hskip 1em .AR $^\sim$arg2:int .RA } \cr
          {\tt \hskip 1em \hskip 1em \hskip 1em .AR retarg:float .RA } \cr
          {\tt \hskip 1em \hskip 1em .VI } \cr
          {\tt \hskip 1em .NI } \cr
          {\tt \hskip 1em \hfil} \cr 
          }
            
              \vskip 2\baselineskip
              Always, labels must be marked with a leading tilde character. All other names appearing in 
              value or function names are symbolic names and are displayed in a slatented font. \\
              And an example for such an interface paragraph with ML$-$types and a ML$-$class: 
              \vskip 2\baselineskip
              \vbox{\hfil
            \vbox{\hrule
          \hbox{\vrule\qquad
        \textcolor{red}{\rule[-.55em]{0pt}{1.7em}\sc Programming Interface}\quad\vrule}
          \hrule}
            }
              \nobreak\vskip \baselineskip\nobreak
              {
            \nobreak\vskip .5\baselineskip 
            \halign{
          
          \textcolor{red}{\strut\vrule width 5pt\hskip 20pt \hfill #} & # \hfil & # \hfil & #\hfill & \qquad # \hfill \cr
          &&&&\cr
          \textcolor{red}{type} & {\bf mytype } $=$  & 
          &Type$\_$1  & \cr
          &&  $|$ &Type$\_$2  & \cr
          &&  $|$ &Type$\_$3  & \cr
          }
            \halign{
          
          \textcolor{red}{\strut\vrule width 5pt\hskip 20pt \hfill #} & # \hfil & # \hfil & #\hfill & \qquad # \hfill \cr
          &&&&\cr
          \textcolor{red}{type} & {\bf mytype } $=$  $\{$  & 
          &\textcolor{red}{mutable }a:int $;$ & \cr
          && &b:float $;$ & \cr
          && &c:int list  $\}$  & \cr
          }
            \halign{
          
          \textcolor{red}{\strut\vrule width 5pt\hskip 20pt \hfill #} & # \hfil & # \hfil & #\hfill & \qquad # \hfill \cr
          &&&&\cr
          \textcolor{red}{exception} & {\bf Failure of string } & 
          \cr
          }
            \halign{
          \textcolor{red}{\strut\vrule width 5pt\hskip 20pt \hfill #} & # \hfil & # \hfil \cr
          &&\cr
          \textcolor{red}{class} & {\bf myclass }: & \cr
           & {\it a }: int $\rightarrow$ & \cr
           & {\it b }: int  & \cr
          & \textcolor{red}{object} & \cr
          &\vbox{
          \halign{
        \hskip 20pt \hfill # & # \hfil  & \qquad # \cr
        \textcolor{red}{val} \textcolor{red}{mutable }{\bf speed }: 
        & {\it dir }: int list  $\rightarrow$  &  \cr
        & {\it vl }:  int list  $\rightarrow$  &  \cr
        &  ?really :  bool  &  \cr
        }
          }&\cr
          &&\cr\noalign{\vglue -1pt}
          &\vbox{
          \halign{
        \hskip 20pt \hfill # & # \hfil  & \qquad # \cr
        \textcolor{red}{val} {\bf align }: 
        & {\it up }: string list  $\rightarrow$  &  \cr
        & {\it down }:  int list  $\rightarrow$  &  \cr
        &  ?really :  bool  &  \cr
        }
          }&\cr
          &&\cr\noalign{\vglue -1pt}
          &\vbox{
          \halign{
        \hskip 20pt \hfill # & # \hfil & # & #\hfill \cr
        \textcolor{red}{method} \textcolor{red}{private }{\bf up }: 
        & & &{\it v }: int  $\rightarrow$ \cr
        & & &{\it name }: string  $\rightarrow$ \cr
        & & &arg3 : int int list list  $\rightarrow$ \cr
        & & & () \cr
        }
          }&\cr
          &&\cr\noalign{\vglue -1pt}
           & \textcolor{red}{end} & \cr
          }
          \halign{
        \textcolor{red}{\strut\vrule width 5pt\qquad#}\tabskip=0pt plus 1fil& 
          #\hfil& #&\hfil#\hfil& 
          \tabskip=0pt#\cr &&&& \cr
         module & {\bf mymod } &&& \cr & sig &&& \cr
        &\vbox{
      \vskip .1\baselineskip
      \halign{
    \hskip 20pt \hfill # & # \hfil & # & # \hfill & \qquad # \hfill \cr
    &&&&\cr
    \textcolor{red}{type} & {\bf mytype } $=$  & 
    &Type$\_$1  & \cr
    &&  $|$ &Type$\_$2  & \cr
    &&  $|$ &Type$\_$3  & \cr
    }
      \vskip .1\baselineskip} &&& \cr \noalign{\vglue -1pt}\noalign{\allowbreak}
        & \vbox{
        \vskip .1\baselineskip
        \halign{
      \hskip 20pt \hfill # & # \hfil  & \qquad # \cr
      \textcolor{red}{val} {\bf align }: 
      & {\it up }: string list  $\rightarrow$  &  \cr
      & {\it down }:  int list  $\rightarrow$  &  \cr
      &  ?really :  bool  &  \cr
      }
        \vskip .1\baselineskip} &&& \cr \noalign{\vglue -1pt}\noalign{\allowbreak}
          & \vbox{
          \vskip .1\baselineskip
          \halign{
        \hskip 20pt \hfill # & # \hfil  & \qquad # \cr
        \textcolor{red}{val} \textcolor{red}{mutable }{\bf speed }: 
        & {\it dir }: int list  $\rightarrow$  &  \cr
        & {\it vl }:  int list  $\rightarrow$  &  \cr
        &  ?really :  bool  &  \cr
        }
          \vskip .1\baselineskip} &&& \cr \noalign{\vglue -1pt}\noalign{\allowbreak}
             & end &&& \cr&&&& \cr
            }
              \halign{\textcolor{red}{\strut\vrule width 5pt#}&#\cr&\cr}
              }
                \vskip 2\baselineskip
                which was produced with this source code: 
                \vskip 2\baselineskip
                \halign{
              \hskip3em#\hfil\cr
               \cr
              {\tt \hskip 1em.IN } \cr
              {\tt \hskip 1em \hskip 1em .IT } \cr
              {\tt \hskip 1em \hskip 1em \hskip 1em .NA mytype .AN } \cr
              {\tt \hskip 1em \hskip 1em \hskip 1em .AR Type$\_$1 .RA } \cr
              {\tt \hskip 1em \hskip 1em \hskip 1em .AR Type$\_$2 .RA } \cr
              {\tt \hskip 1em \hskip 1em \hskip 1em .AR Type$\_$3 .RA } \cr
              {\tt \hskip 1em \hskip 1em .TI } \cr
              {\tt \hskip 1em \hskip 1em .IS } \cr
              {\tt \hskip 1em \hskip 1em \hskip 1em .NA mytype .AN } \cr
              {\tt \hskip 1em \hskip 1em \hskip 1em .AR .MU a:int .RA } \cr
              {\tt \hskip 1em \hskip 1em \hskip 1em .AR b:float .RA } \cr
              {\tt \hskip 1em \hskip 1em \hskip 1em .AR c:int list .RA } \cr
              {\tt \hskip 1em \hskip 1em .SI } \cr
              {\tt \hskip 1em \hskip 1em .CS } \cr
              {\tt \hskip 1em \hskip 1em \hskip 1em .NA myclass .AN } \cr
              {\tt \hskip 1em \hskip 1em \hskip 1em .AR a:int .RA } \cr
              {\tt \hskip 1em \hskip 1em \hskip 1em .AR b:int .RA } \cr
              {\tt \hskip 1em \hskip 1em \hskip 1em .OB } \cr
              {\tt \hskip 1em \hskip 1em \hskip 1em \hskip 1em .IV } \cr
              {\tt \hskip 1em \hskip 1em \hskip 1em \hskip 1em \hskip 1em .NA .MU speed .AN } \cr
              {\tt \hskip 1em \hskip 1em \hskip 1em \hskip 1em \hskip 1em .AR dir:int list .RA } \cr
              {\tt \hskip 1em \hskip 1em \hskip 1em \hskip 1em \hskip 1em .AR vl: int list .RA } \cr
              {\tt \hskip 1em \hskip 1em \hskip 1em \hskip 1em \hskip 1em .AR ?really: bool .RA } \cr
              {\tt \hskip 1em \hskip 1em \hskip 1em \hskip 1em .VI } \cr
              {\tt \hskip 1em \hskip 1em \hskip 1em \hskip 1em .IV } \cr
              {\tt \hskip 1em \hskip 1em \hskip 1em \hskip 1em \hskip 1em .NA align .AN } \cr
              {\tt \hskip 1em \hskip 1em \hskip 1em \hskip 1em \hskip 1em .AR up:string list .RA } \cr
              {\tt \hskip 1em \hskip 1em \hskip 1em \hskip 1em \hskip 1em .AR down: int list .RA } \cr
              {\tt \hskip 1em \hskip 1em \hskip 1em \hskip 1em \hskip 1em .AR ?really: bool .RA } \cr
              {\tt \hskip 1em \hskip 1em \hskip 1em \hskip 1em .VI } \cr
              {\tt \hskip 1em \hskip 1em \hskip 1em \hskip 1em .MT } \cr
              {\tt \hskip 1em \hskip 1em \hskip 1em \hskip 1em \hskip 1em .NA .PV up .AN } \cr
              {\tt \hskip 1em \hskip 1em \hskip 1em \hskip 1em \hskip 1em .AR v:int .RA } \cr
              {\tt \hskip 1em \hskip 1em \hskip 1em \hskip 1em \hskip 1em .AR name:string .RA } \cr
              {\tt \hskip 1em \hskip 1em \hskip 1em \hskip 1em \hskip 1em .AR $^\sim$arg3:int int list list .RA } \cr
              {\tt \hskip 1em \hskip 1em \hskip 1em \hskip 1em \hskip 1em .AR () .RA } \cr
              {\tt \hskip 1em \hskip 1em \hskip 1em \hskip 1em .TM } \cr
              {\tt \hskip 1em \hskip 1em \hskip 1em .BO } \cr
              {\tt \hskip 1em \hskip 1em .SC } \cr
              {\tt \hskip 1em \hskip 1em .IM } \cr
              {\tt \hskip 1em \hskip 1em \hskip 1em .NA mymod .AN } \cr
              {\tt \hskip 1em \hskip 1em \hskip 1em .IT } \cr
              {\tt \hskip 1em \hskip 1em \hskip 1em \hskip 1em .NA mytype .AN } \cr
              {\tt \hskip 1em \hskip 1em \hskip 1em \hskip 1em .AR Type$\_$1 .RA } \cr
              {\tt \hskip 1em \hskip 1em \hskip 1em \hskip 1em .AR Type$\_$2 .RA } \cr
              {\tt \hskip 1em \hskip 1em \hskip 1em \hskip 1em .AR Type$\_$3 .RA } \cr
              {\tt \hskip 1em \hskip 1em \hskip 1em .TI } \cr
              {\tt \hskip 1em \hskip 1em \hskip 1em .IV } \cr
              {\tt \hskip 1em \hskip 1em \hskip 1em \hskip 1em .NA align .AN } \cr
              {\tt \hskip 1em \hskip 1em \hskip 1em \hskip 1em .AR up:string list .RA } \cr
              {\tt \hskip 1em \hskip 1em \hskip 1em \hskip 1em .AR down: int list .RA } \cr
              {\tt \hskip 1em \hskip 1em \hskip 1em \hskip 1em .AR ?really: bool .RA } \cr
              {\tt \hskip 1em \hskip 1em \hskip 1em .VI } \cr
              {\tt \hskip 1em \hskip 1em \hskip 1em .IV } \cr
              {\tt \hskip 1em \hskip 1em \hskip 1em \hskip 1em .NA .MU speed .AN } \cr
              {\tt \hskip 1em \hskip 1em \hskip 1em \hskip 1em .AR dir:int list .RA } \cr
              {\tt \hskip 1em \hskip 1em \hskip 1em \hskip 1em .AR vl: int list .RA } \cr
              {\tt \hskip 1em \hskip 1em \hskip 1em \hskip 1em .AR ?really: bool .RA } \cr
              {\tt \hskip 1em \hskip 1em \hskip 1em .VI } \cr
              {\tt \hskip 1em \hskip 1em .MI } \cr
              {\tt \hskip 1em .NI } \cr
              {\tt \hskip 1em \hfil} \cr 
              }
                
                  \vskip 2\baselineskip
                  The class interface needs the class name, optional arguments, and the object block with values 
                  and methods. Of course, all possible special interfaces can be mixed, and the interface subunit 
                  is not required, but strongly recommended. Currently, only simple module signatures without 
                  functors are supported. 
                  \vfill \eject
                  \vbox{\hsize \textwidth
                \hrule
                \vbox{\rule[-.55em]{0pt}{1.7em}\strut \vrule \quad 
              \label{Paragraphelements}\textcolor{blue}{\sc {\sl Paragraph elements }} \hfill 
              \textcolor[rgb]{.0,.7,.0}{\sc  MlDoc dot commands}\quad \vrule}\hrule}
                
                  \vskip \baselineskip
                  In this section, all generic paragraph elements like lists are described. \\
                  There are three different kinds of list you can use: 
                  \vskip \baselineskip
                  \begin{minipage}{.9\textwidth}\vbox{
                \begin{itemize}
              \item Numbered lists (aka ordered lists) 
              \item Unnumbered lists (aka unordered lists) 
              \item Option lists (aka definition lists) 
              \end{itemize}
                }\end{minipage}
                  \vskip \baselineskip
                  \vskip .5\baselineskip 
                  {\parindent 0pt \vbox{\sc Lists }
                  \leftskip 20pt \vskip .5\baselineskip
                  \begin{list}{}{}\item{\parindent 0pt 
                Lists of different kind can be mixed and stacked to an arbitrary depth. 
                \vskip 2\baselineskip
                \begin{center}
              \begin{tabular}{|p{3.3cm}|p{1.8cm}|p{1.5cm}|p{3.3cm}|}\hline
            \multicolumn{4}{|c|}{\textcolor{red}{\sc Lists }}\\\hline\hline
            {\bf Name }&{\bf begin }&{\bf end }&{\bf Comment }\\\hline
            Ordered (numbered) List &{\tt .OL }&{\tt .LO }&\\\hline
            Unordered List &{\tt .UL }&{\tt .LU }&\\\hline
            Option List &{\tt .PL }&{\tt .LP }&First command in List item must be a Name. \\\hline
            List item &{\tt .LI }&{\tt .IL }&Used inside lists only. \\\hline
            \end{tabular}
              \end{center}
                \vskip 2\baselineskip
                \vskip \baselineskip
                \begin{minipage}{.9\textwidth}\vbox{
              \begin{enumerate}
            \item 1 
            \item 2 
            \end{enumerate}
              }\end{minipage}
                \vskip \baselineskip
                Here is an example for an option list: 
                \vskip 2\baselineskip
                \vbox{\hfil
              \vbox{\hrule
            \hbox{\vrule\qquad
          {\rule[-.55em]{0pt}{1.7em}\sc Options }\qquad\vrule}
            \hrule}
              }
                \nobreak\vskip \baselineskip\nobreak
                \begin{list}{}{}\item{\parindent 0pt 
              \begin{description}
            \item[ $-$a]
          {\hfill\\}
          This program option is used to show all available options. It's not needed in the server 
          mode, but it's possible use this option in client mode. 
            \item[ $-$b]
          {\hfill\\}
          This program option is used to build all available options. 
            \item[ $-$c]
          {\hfill\\}
          This program option is used to convert all available options. 
            \end{description}
              }\end{list}
                \vskip .5\baselineskip
                \vskip \baselineskip
                The required source code for this option list is shown below: 
                \vskip 2\baselineskip
                \halign{
              \hskip3em#\hfil\cr
               \cr
              {\tt \hskip 1em.PL } \cr
              {\tt \hskip 1em \hskip 1em .LI } \cr
              {\tt \hskip 1em \hskip 1em \hskip 1em .NA $-$a .AN } \cr
              {\tt \hskip 1em \hskip 1em \hskip 1em This program option is used to show all available options. } \cr
              {\tt \hskip 1em \hskip 1em .IL } \cr
              {\tt \hskip 1em \hskip 1em .LI } \cr
              {\tt \hskip 1em \hskip 1em \hskip 1em .NA $-$b .AN } \cr
              {\tt \hskip 1em \hskip 1em \hskip 1em This program option is used to build all available options. } \cr
              {\tt \hskip 1em \hskip 1em .IL } \cr
              {\tt \hskip 1em .LP } \cr
              {\tt \hskip 1em \hfil} \cr 
              }
                
                  \vskip 2\baselineskip
                  The {\sl Option List }header can be changed with the {\sl Name }argument, too. }\end{list}}
                    \vskip .5\baselineskip
                    \vskip .5\baselineskip 
                    {\parindent 0pt \vbox{\sc Tables }
                    \leftskip 20pt \vskip .5\baselineskip
                    \begin{list}{}{}\item{\parindent 0pt 
                  There is support for simple tables, too. A {\sl MlDoC }table is comparable with HTML tables. You 
                  need to specifiy the table body, an optional table head, table rows and in each table row the 
                  table column data. 
                  \vskip 2\baselineskip
                  \begin{center}
                \begin{tabular}{|p{3.3cm}|p{1.8cm}|p{1.5cm}|p{3.3cm}|}\hline
              \multicolumn{4}{|c|}{\textcolor{red}{\sc Tables }}\\\hline\hline
              {\bf Name }&{\bf begin }&{\bf end }&{\bf Comment }\\\hline
              Table body &{\tt .TB }&{\tt .BT }&\\\hline
              NoRulers &{\tt .NR }&&Don't put boxes around the table cells. \\\hline
              Table head &{\tt .TH }&{\tt .HT }&Optional table head row. \\\hline
              Table row &{\tt .TR }&{\tt .RT }&\\\hline
              Table columns &{\tt .TC }&{\tt .CT }&Must be included between the row command. \\\hline
              \end{tabular}
                \end{center}
                  \vskip 2\baselineskip
                  Here is a small example for a generic table: 
                  \vskip 2\baselineskip
                  \begin{center}
                \begin{tabular}{|p{1.5cm}|p{1.5cm}|}\hline
              1 &2 \\\hline
              3 &4 \\\hline
              \end{tabular}
                \end{center}
                  \vskip 2\baselineskip
                  and the required source code: 
                  \vskip 2\baselineskip
                  \halign{
                \hskip3em#\hfil\cr
                 \cr
                {\tt \hskip 1em.TB } \cr
                {\tt \hskip 1em \hskip 1em .TR } \cr
                {\tt \hskip 1em \hskip 1em \hskip 1em .TC 1 .CT } \cr
                {\tt \hskip 1em \hskip 1em \hskip 1em .TC 2 .CT } \cr
                {\tt \hskip 1em \hskip 1em .RT } \cr
                {\tt \hskip 1em \hskip 1em .TR } \cr
                {\tt \hskip 1em \hskip 1em \hskip 1em .TC 3 .CT } \cr
                {\tt \hskip 1em \hskip 1em \hskip 1em .TC 4 .CT } \cr
                {\tt \hskip 1em \hskip 1em .RT } \cr
                {\tt \hskip 1em .BT } \cr
                {\tt \hskip 1em \hfil} \cr 
                }
                  
                    \vskip 2\baselineskip
                    The horizontal and vertical rulers are always present and can be removed with the {\sl NoRulers }
                    command. The {\tt .NR }command must follow the {\sl TableBody }command immediately. The above table 
                    looks then like: 
                    \vskip 2\baselineskip
                    \begin{center}
                  \begin{tabular}{p{1.5cm}p{1.5cm}}
                1 &2 \\
                3 &4 \\
                \end{tabular}
                  \end{center}
                    \vskip 2\baselineskip
                    }\end{list}}
                      \vskip .5\baselineskip
                      Misc. elements: 
                      \vskip 2\baselineskip
                      \begin{center}
                    \begin{tabular}{|p{3.3cm}|p{1.8cm}|p{1.5cm}|p{3.3cm}|}\hline
                  {\bf Name }&{\bf begin }&{\bf end }&{\bf Description }\\\hline
                  Name &{\tt .NA }&{\tt .AN }&Outside from special interfaces, this command emphasize names of 
                  functions, types or other important names. \\\hline
                  Link &{\tt .LK }&{\tt .KL }&A link to another section or document \\\hline
                  \end{tabular}
                    \end{center}
                      \vskip 2\baselineskip
                      \vfill \eject
                      \vbox{\hsize \textwidth
                    \hrule
                    \vbox{\rule[-.55em]{0pt}{1.7em}\strut \vrule \quad 
                  \label{Textelements}\textcolor{blue}{\sc {\sl Text elements }} \hfill 
                  \textcolor[rgb]{.0,.7,.0}{\sc  MlDoc dot commands}\quad \vrule}\hrule}
                    
                      \vskip \baselineskip
                      \vskip 2\baselineskip
                      \begin{center}
                    \begin{tabular}{|p{3.3cm}|p{1.8cm}|p{1.5cm}|p{3.3cm}|}\hline
                  {\bf Name }&{\bf begin }&{\bf end }&{\bf Description }\\\hline
                  Newline &{\tt .NL }&&Flush the current text paragraph and force a newline. \\\hline
                  Indentmark &{\tt .$\#$ }&&Increase the current text indention. Only used in preformatted 
                  paragraphs like examples. \\\hline
                  Preformatted text ('as is') &{\tt . [ }&{\tt . ] }&Print inlined raw text. Ignore all dot 
                  commands. \\\hline
                  Comments &{\tt .($*$ }&{\tt .$*$) }&Comments can be placed in {\sl Arguments }. \\\hline
                  \end{tabular}
                    \end{center}
                      \vskip 2\baselineskip
                      See \fbox{\small {\sc Special Interfaces} (p. \pageref{SpecialInterfaces})} for comment examples and 
                      the place to use them. Use the {\sl Newline }command only to finish the current text paragraph, and 
                      not to create space between paragraphs or other structure elements. \\
                      There are several text attribute commands controlling the appearence of text. All font styles must be 
                      switched off with the {\tt .R }regular style command. 
                      \vskip 2\baselineskip
                      \begin{center}
                    \begin{tabular}{|p{6.6cm}|p{2.4cm}|}\hline
                  \multicolumn{2}{|c|}{\textcolor{red}{\sc Text attributes }}\\\hline\hline
                  {\bf Attribute }&{\bf command }\\\hline
                  {\bf Boldface style }&{\tt .B }\\\hline
                  {\it Italic style }&{\tt .I }\\\hline
                  {\bf\sl Boldface and italic style }&{\tt .BI }\\\hline
                  {\tt Typewriter style }&{\tt .T }\\\hline
                  Romane style (default) &{\tt .R }\\\hline
                  Superscript $^{\rm textmode\quad}$&{\tt .SP }{\tt .PS }\\\hline
                  Subscript $_{\rm textmode\quad}$&{\tt .SB }{\tt .BS }\\\hline
                  \end{tabular}
                    \end{center}
                      \vskip 2\baselineskip
                      \vfill \eject
                      \vbox{\hsize \textwidth
                    \hrule
                    \vbox{\rule[-.55em]{0pt}{1.7em}\strut \vrule \quad 
                  \label{Summaryofdotcommands}\textcolor{blue}{\sc {\sl Summary of dot commands }} \hfill 
                  \textcolor[rgb]{.0,.7,.0}{\sc  MlDoc dot commands}\quad \vrule}\hrule}
                    
                      \vskip \baselineskip
                      \vskip .5\baselineskip 
                      {\parindent 0pt \vbox{\sc \label{Textstylesandspecialtext}Text styles and special text }
                      \leftskip 20pt \rightskip 20pt \vskip .5\baselineskip\begin{list}{}{}\item{\parindent 0pt 
                    \vskip 2\baselineskip
                    \begin{center}
                  \begin{tabular}{p{1.5cm}p{1.5cm}p{4.4cm}}
                {\tt .B }&{\tt .R }&Bold text style \\
                {\tt .BI }&{\tt .R }&Bold$-$Italic text style \\
                {\tt .I }&{\tt .R }&Italic text style \\
                {\tt .T }&{\tt .R }&Typewriter text style \\
                {\tt .S }&{\tt .R }&Symbol text style [NA] \\
                {\tt .SB }&{\tt .BS }&Subscript text mode \\
                {\tt .SP }&{\tt .PS }&Superscript text mode \\
                {\tt . [ }&{\tt . ] }&'As$-$Is' text mode \\
                {\tt .NL }&&Newline: End of paragraph \\
                \end{tabular}
                  \end{center}
                    \vskip 2\baselineskip
                    }\end{list}}
                      \vskip .5\baselineskip
                      \vskip .5\baselineskip 
                      {\parindent 0pt \vbox{\sc \label{Specialdotcommands}Special dot commands }
                      \leftskip 20pt \rightskip 20pt \vskip .5\baselineskip\begin{list}{}{}\item{\parindent 0pt 
                    \vskip 2\baselineskip
                    \begin{center}
                  \begin{tabular}{p{1.5cm}p{1.5cm}p{4.4cm}}
                {\tt .MU }&&Special ML$-$attribute {\sl mutable }\\
                {\tt .PV }&&Special ML$-$attribute {\sl private }\\
                {\tt .VT }&&Special ML$-$attribute {\sl virtual }\\
                {\tt .NR }&&No rulers around tables and table cells \\
                {\tt .$\#$ }&&White Space in {\sl example }and {\sl As$-$Is }environments \\
                {\tt .$<$$<$ }&{\tt .$>$$>$ }&Include a {\sl MlDoc }file at the current position \\
                {\tt .LK }&{\tt .KL }&A document link \\
                \end{tabular}
                  \end{center}
                    \vskip 2\baselineskip
                    }\end{list}}
                      \vskip .5\baselineskip
                      \vskip .5\baselineskip 
                      {\parindent 0pt \vbox{\sc \label{Listcommands}List commands }
                      \leftskip 20pt \rightskip 20pt \vskip .5\baselineskip\begin{list}{}{}\item{\parindent 0pt 
                    \vskip 2\baselineskip
                    \begin{center}
                  \begin{tabular}{p{1.5cm}p{1.5cm}p{4.4cm}}
                {\tt .OL }&{\tt .LO }&Numbered (ordered) list body \\
                {\tt .UL }&{\tt .LU }&Unnumbered (unorderd) list body \\
                {\tt .PL }&{\tt .LP }&Option (definition) list \\
                {\tt .LI }&{\tt .IL }&A list item \\
                \end{tabular}
                  \end{center}
                    \vskip 2\baselineskip
                    }\end{list}}
                      \vskip .5\baselineskip
                      \vskip .5\baselineskip 
                      {\parindent 0pt \vbox{\sc \label{Tablecommands}Table commands }
                      \leftskip 20pt \rightskip 20pt \vskip .5\baselineskip\begin{list}{}{}\item{\parindent 0pt 
                    \vskip 2\baselineskip
                    \begin{center}
                  \begin{tabular}{p{1.5cm}p{1.5cm}p{4.4cm}}
                {\tt .TB }&{\tt .BT }&Table body \\
                {\tt .TR }&{\tt .RT }&Table row \\
                {\tt .TC }&{\tt .CT }&Table column \\
                {\tt .TH }&{\tt .HT }&Optional table head \\
                \end{tabular}
                  \end{center}
                    \vskip 2\baselineskip
                    }\end{list}}
                      \vskip .5\baselineskip
                      \vskip .5\baselineskip 
                      {\parindent 0pt \vbox{\sc \label{SpecialInterfacescommands}Special Interfaces commands }
                      \leftskip 20pt \rightskip 20pt \vskip .5\baselineskip\begin{list}{}{}\item{\parindent 0pt 
                    \vskip 2\baselineskip
                    \begin{center}
                  \begin{tabular}{p{1.5cm}p{1.5cm}p{4.4cm}}
                {\tt .IF }&{\tt .FI }&ML$-$Function Interface \\
                {\tt .IV }&{\tt .VI }&ML$-$Value Interface \\
                {\tt .IT }&{\tt .TI }&ML$-$Type Interface \\
                {\tt .IS }&{\tt .SI }&ML$-$Structure Interface \\
                {\tt .IX }&{\tt .XI }&ML$-$Exception Interface \\
                {\tt .IM }&{\tt .MI }&ML$-$Module Interface \\
                {\tt .CH }&{\tt .HC }&C$-$Header Interface \\
                {\tt .CF }&{\tt .FC }&C$-$Function Interface \\
                {\tt .CV }&{\tt .VC }&C$-$Variable Interface \\
                {\tt .CS }&{\tt .SC }&ML$-$Class Interface \\
                {\tt .OB }&{\tt .BO }&ML$-$Object (Class) Interface \\
                {\tt .MT }&{\tt .TM }&ML$-$Method (Class) Interface \\
                {\tt }&{\tt }&\\
                {\tt .NA }&{\tt .AN }&Name argument \\
                {\tt .AR }&{\tt .RA }&Curried value argument \\
                {\tt .AV }&{\tt .VA }&Uncurried (tuple) value argument \\
                {\tt .RV }&{\tt .VR }&Return argument of a function \\
                \end{tabular}
                  \end{center}
                    \vskip 2\baselineskip
                    }\end{list}}
                      \vskip .5\baselineskip
                      \vskip .5\baselineskip 
                      {\parindent 0pt \vbox{\sc \label{Sectioncommands}Section commands }
                      \leftskip 20pt \rightskip 20pt \vskip .5\baselineskip\begin{list}{}{}\item{\parindent 0pt 
                    \vskip 2\baselineskip
                    \begin{center}
                  \begin{tabular}{p{1.5cm}p{1.5cm}p{4.4cm}}
                {\tt .PR }&{\tt .RP }&Program Section \\
                {\tt .PK }&{\tt .KP }&Package Section \\
                \end{tabular}
                  \end{center}
                    \vskip 2\baselineskip
                    }\end{list}}
                      \vskip .5\baselineskip
                      \vskip .5\baselineskip 
                      {\parindent 0pt \vbox{\sc \label{Subsectioncommands}Subsection commands }
                      \leftskip 20pt \rightskip 20pt \vskip .5\baselineskip\begin{list}{}{}\item{\parindent 0pt 
                    \vskip 2\baselineskip
                    \begin{center}
                  \begin{tabular}{p{1.5cm}p{1.5cm}p{3cm}}
                {\tt .MD }&{\tt .DM }&ML$-$Module \\
                {\tt .CY }&{\tt .YC }&C$-$Library \\
                \end{tabular}
                  \end{center}
                    \vskip 2\baselineskip
                    }\end{list}}
                      \vskip .5\baselineskip
                      \vskip .5\baselineskip 
                      {\parindent 0pt \vbox{\sc \label{UnitsofaSubsectionManualPagecommands}Units of a Subsection (Manual 
                      Page) commands }
                      \leftskip 20pt \rightskip 20pt \vskip .5\baselineskip\begin{list}{}{}\item{\parindent 0pt 
                    \vskip 2\baselineskip
                    \begin{center}
                  \begin{tabular}{p{1.5cm}p{1.5cm}p{4.4cm}}
                {\tt .FU }&{\tt .UF }&ML$-$Function \\
                {\tt .MV }&{\tt .VM }&ML$-$Value \\
                {\tt .TP }&{\tt .PT }&ML$-$Type (generic) \\
                {\tt .CL }&{\tt .LC }&ML$-$Class \\
                {\tt .CI }&{\tt .IC }&C \\
                \end{tabular}
                  \end{center}
                    \vskip 2\baselineskip
                    }\end{list}}
                      \vskip .5\baselineskip
                      \vskip .5\baselineskip 
                      {\parindent 0pt \vbox{\sc \label{Subunitcommands}Subunit commands }
                      \leftskip 20pt \rightskip 20pt \vskip .5\baselineskip\begin{list}{}{}\item{\parindent 0pt 
                    \vskip 2\baselineskip
                    \begin{center}
                  \begin{tabular}{p{1.5cm}p{1.5cm}p{4.4cm}}
                {\tt .IN }&{\tt .NI }&Generic Interface Paragraph \\
                {\tt .EX }&{\tt .XE }&Example Paragraph \\
                {\tt .\{ }&{\tt .\} }&Preformatted Paragraph \\
                {\tt .PA }&{\tt .AP }&A named Paragraph \\
                {\tt .IO }&{\tt .OI }&An Introduction Paragraph \\
                \end{tabular}
                  \end{center}
                    \vskip 2\baselineskip
                    }\end{list}}
                      \vskip .5\baselineskip
                      \vskip .5\baselineskip 
                      {\parindent 0pt \vbox{\sc \label{GenericSectionSubsectionUnitSubunitcommands}Generic Section, 
                      Subsection, Unit, Subunit commands }
                      \leftskip 20pt \rightskip 20pt \vskip .5\baselineskip\begin{list}{}{}\item{\parindent 0pt 
                    \vskip 2\baselineskip
                    \begin{center}
                  \begin{tabular}{p{1.5cm}p{1.5cm}p{4.4cm}}
                {\tt .S1 }&{\tt .1S }&Generic Section \\
                {\tt .S2 }&{\tt .2S }&Generic Subsection \\
                {\tt .S3 }&{\tt .3S }&Generic Unit \\
                {\tt .S4 }&{\tt .4S }&Generic Subunit \\
                \end{tabular}
                  \end{center}
                    \vskip 2\baselineskip
                    }\end{list}}
                      \vskip .5\baselineskip
                      \vfill \eject
                      \vbox{\hsize \textwidth
                    \hrule
                    \vbox{\rule[-.55em]{0pt}{1.7em}\strut \vrule \quad 
                  \label{CompilationandusageoftheMlDoCpackage}\textcolor{blue}{\sc {\sl Compilation and usage of 
                  the MlDoC package }} \hfill 
                  \textcolor[rgb]{.0,.7,.0}{\sc  MlDoc dot commands}\quad \vrule}\hrule}
                    
                      \vskip \baselineskip
                      First you must compile the package, consisting of the core module, the backend and the frontend 
                      modules. Here the steps needed for the case you use the generic OCaML system, version 3.XX: 
                      \vskip 2\baselineskip
                      \halign{
                    \hskip3em#\hfil\cr
                     \cr
                    {\tt \hskip 1emocamlc $-$c doc$\_$core.ml } \cr
                    {\tt \hskip 1em ocamlc $-$c doc$\_$html.ml } \cr
                    {\tt \hskip 1em ocamlc $-$c doc$\_$latex.ml } \cr
                    {\tt \hskip 1em ocamlc $-$c doc$\_$text.ml } \cr
                    {\tt \hskip 1em ocamlc $-$c doc$\_$help.ml } \cr
                    {\tt \hskip 1em ocamlc $-$a $*$.cmo $-$o mldoc.cma } \cr
                    {\tt \hskip 1em \hfil} \cr 
                    }
                      
                        \vskip 2\baselineskip
                        You need the {\sl Terminfo }interface file {\tt terminfo.cmi }to be able to compile the {\sl 
                        Doc$\_$text }module. \\
                        For {\sl VamSyS }users: 
                        \vskip 2\baselineskip
                        \halign{
                      \hskip3em#\hfil\cr
                       \cr
                      {\tt \hskip 1emvamcomp doc$\_$core.ml } \cr
                      {\tt \hskip 1em vamcomp doc$\_$html.ml } \cr
                      {\tt \hskip 1em vamcomp doc$\_$latex.ml } \cr
                      {\tt \hskip 1em vamcomp doc$\_$text.ml } \cr
                      {\tt \hskip 1em vamcomp doc$\_$help.ml } \cr
                      {\tt \hskip 1em vamar $*$.cmo $-$o mldoc.cma } \cr
                      {\tt \hskip 1em \hfil} \cr 
                      }
                        
                          \vskip 2\baselineskip
                          \vfill \eject
                          \vbox{\hsize \textwidth
                        \hrule
                        \vbox{\rule[-.55em]{0pt}{1.7em}\strut \vrule \quad \hfill
                      \label{mldoc}\textcolor{blue}{\sc Package: {\sl mldoc }} \hfill
                      \quad \vrule}\hrule}
                        
                          \vskip \baselineskip
                          The {\sl mldoc }library is the heart of the {\sl MlDoC }system. It's divided in the core module and 
                          various backend modules and some frontend modules. 
                          \vfill \eject
                          \vbox{\hsize \textwidth
                        \hrule
                        \vbox{\rule[-.55em]{0pt}{1.7em}\strut \vrule \quad 
                      \label{doc$_$core}\textcolor{blue}{\sc ML-Module: {\sl doc$\_$core }} \hfill 
                      \textcolor[rgb]{.0,.7,.0}{\sc Package:  mldoc}\quad \vrule}\hrule}
                        
                          \vskip \baselineskip
                          \vskip .5\baselineskip 
                          {\parindent 0pt \vbox{\sc \label{Description}Description }
                          \leftskip 20pt \rightskip 20pt \vskip .5\baselineskip\begin{list}{}{}\item{\parindent 0pt 
                        The core module of {\sl MlDoC }. This module contains all basic functions and types used by all other 
                        frontend and backend modules. }\end{list}}
                          \vskip .5\baselineskip
                          Flow of processing string text: 
                          \vskip \baselineskip
                          \begin{minipage}{.9\textwidth}\vbox{
                        \begin{enumerate}
                      \item Split the string text in atoms divided by blank spaces. 
                      \item Translate these atoms in a structure tree. 
                      \item Convert the structure tree into the desired output format. 
                      \end{enumerate}
                        }\end{minipage}
                          \vskip \baselineskip
                          First the interface of the main {\sl structure$\_$content }type and the the main {\sl structure$\_$block }
                          : 
                          \vskip 2\baselineskip
                          \vbox{\hfil
                        \vbox{\hrule
                      \hbox{\vrule\qquad
                    \textcolor{red}{\rule[-.55em]{0pt}{1.7em}\sc Programming Interface}\quad\vrule}
                      \hrule}
                        }
                          \nobreak\vskip \baselineskip\nobreak
                          {
                        \nobreak\vskip .5\baselineskip 
                        \halign{
                      
                      \textcolor{red}{\strut\vrule width 5pt\hskip 20pt \hfill #} & # \hfil & # \hfil & #\hfill & \qquad # \hfill \cr
                      &&&&\cr
                      \textcolor{red}{type} & {\bf structure$\_$content } $=$  & 
                      &S$\_$Empty &{$\gg\:$\it Empty element $\ll$}\cr
                      &&  $|$ &S$\_$Text of (string list) &{$\gg\:$\it Generic text: basic element $\ll$}\cr
                      &&  $|$ &S$\_$Body &{$\gg\:$\it Document root $\ll$}\cr
                      &&  $|$ &S$\_$Title &{$\gg\:$\it Document title $\ll$}\cr
                      &&  $|$ &S$\_$TOC &{$\gg\:$\it Table of contents $\ll$}\cr
                      &&  $|$ &S$\_$Intro &{$\gg\:$\it An intro paragraph $\ll$}\cr
                      &&  $|$ &S$\_$Package &{$\gg\:$\it Package$=$Library $\ll$}\cr
                      &&  $|$ &S$\_$Program &{$\gg\:$\it A program description $\ll$}\cr
                      &&  $|$ &S$\_$S1 &{$\gg\:$\it Generic named section $\ll$}\cr
                      &&  $|$ &S$\_$Module &{$\gg\:$\it A ML$-$module $\ll$}\cr
                      &&  $|$ &S$\_$Clib &{$\gg\:$\it A C$-$Library $\ll$}\cr
                      &&  $|$ &S$\_$S2 &{$\gg\:$\it Generic subsection $\ll$}\cr
                      &&  $|$ &S$\_$Function &{$\gg\:$\it A function subsection $\ll$}\cr
                      &&  $|$ &S$\_$Type &{$\gg\:$\it A type subsection $\ll$}\cr
                      &&  $|$ &S$\_$Value &{$\gg\:$\it A value subsection $\ll$}\cr
                      &&  $|$ &S$\_$Class &{$\gg\:$\it A class subsection $\ll$}\cr
                      &&  $|$ &S$\_$C &{$\gg\:$\it C functions and variables $\ll$}\cr
                      &&  $|$ &S$\_$S3 &{$\gg\:$\it Generic unit $\ll$}\cr
                      &&  $|$ &S$\_$Paragraph &{$\gg\:$\it A named paragraph $\ll$}\cr
                      &&  $|$ &S$\_$Interface &{$\gg\:$\it Generic Interface $\ll$}\cr
                      &&  $|$ &S$\_$Example &{$\gg\:$\it Example preformatted paragraph $\ll$}\cr
                      &&  $|$ &S$\_$Preform &{$\gg\:$\it Preformatted paragraph $\ll$}\cr
                      &&  $|$ &S$\_$S4 &{$\gg\:$\it Generic subunit $\ll$}\cr
                      &&  $|$ &S$\_$Fun$\_$Interface &{$\gg\:$\it ML$-$Function interface $\ll$}\cr
                      &&  $|$ &S$\_$Val$\_$Interface &{$\gg\:$\it ML$-$Value interface $\ll$}\cr
                      &&  $|$ &S$\_$Type$\_$Interface &{$\gg\:$\it A ML$-$type definition $\ll$}\cr
                      &&  $|$ &S$\_$Exc$\_$Interface &{$\gg\:$\it Exception $\ll$}\cr
                      &&  $|$ &S$\_$Struc$\_$Interface &{$\gg\:$\it A type structure $\ll$}\cr
                      &&  $|$ &S$\_$C$\_$Hdr$\_$Interface &{$\gg\:$\it C header interface $\ll$}\cr
                      &&  $|$ &S$\_$C$\_$Fun$\_$Interface &{$\gg\:$\it C function interface $\ll$}\cr
                      &&  $|$ &S$\_$C$\_$Var$\_$Interface &{$\gg\:$\it C variable interface $\ll$}\cr
                      &&  $|$ &S$\_$Class$\_$Interface &{$\gg\:$\it A ML$-$Class $\ll$}\cr
                      &&  $|$ &S$\_$Method$\_$Interface &{$\gg\:$\it A ML method of a class $\ll$}\cr
                      &&  $|$ &S$\_$Object$\_$Interface &{$\gg\:$\it A ML object of a class $\ll$}\cr
                      &&  $|$ &S$\_$Module$\_$Interface &{$\gg\:$\it A ML module interface $\ll$}\cr
                      &&  $|$ &S$\_$Attribute &{$\gg\:$\it Text or other attributes changes $\ll$}\cr
                      &&  $|$ &S$\_$OList &{$\gg\:$\it Ordered List $\ll$}\cr
                      &&  $|$ &S$\_$UList &{$\gg\:$\it Unordered List $\ll$}\cr
                      &&  $|$ &S$\_$OpList &{$\gg\:$\it Definition List $\ll$}\cr
                      &&  $|$ &S$\_$List$\_$Item &{$\gg\:$\it Unordered List item $\ll$}\cr
                      &&  $|$ &S$\_$Table &{$\gg\:$\it Table body $\ll$}\cr
                      &&  $|$ &S$\_$TableHead &{$\gg\:$\it Table header $\ll$}\cr
                      &&  $|$ &S$\_$TableRow &{$\gg\:$\it A new table row $\ll$}\cr
                      &&  $|$ &S$\_$TableCol &{$\gg\:$\it Table column data $\ll$}\cr
                      &&  $|$ &S$\_$TableNoRulers &{$\gg\:$\it Don't put boxes around the table cells $\ll$}\cr
                      &&  $|$ &S$\_$Name &{$\gg\:$\it Function, Type,... name $\ll$}\cr
                      &&  $|$ &S$\_$Mutable &{$\gg\:$\it Mutable prefix $\ll$}\cr
                      &&  $|$ &S$\_$Private &{$\gg\:$\it Provate prefix $\ll$}\cr
                      &&  $|$ &S$\_$Virtual &{$\gg\:$\it Virtual prefix $\ll$}\cr
                      &&  $|$ &S$\_$CurArg &{$\gg\:$\it Argument of a curried list $\ll$}\cr
                      &&  $|$ &S$\_$UnCurArg &{$\gg\:$\it Argument of an uncurried list:tuple $\ll$}\cr
                      &&  $|$ &S$\_$RetArg &{$\gg\:$\it Return argument of a function $\ll$}\cr
                      &&  $|$ &S$\_$NL &{$\gg\:$\it Newline $\ll$}\cr
                      &&  $|$ &S$\_$TAB &{$\gg\:$\it Text tabulation $\ll$}\cr
                      &&  $|$ &S$\_$Link &{$\gg\:$\it Link $\ll$}\cr
                      &&  $|$ &S$\_$Comment &{$\gg\:$\it A comment $\ll$}\cr
                      }
                        \halign{
                      
                      \textcolor{red}{\strut\vrule width 5pt\hskip 20pt \hfill #} & # \hfil & # \hfil & #\hfill & \qquad # \hfill \cr
                      &&&&\cr
                      \textcolor{red}{type} & {\bf structure$\_$block } $=$  $\{$  & 
                      &\textcolor{red}{mutable }s$\_$parent: structure$\_$block option $;$ & \cr
                      && &\textcolor{red}{mutable }s$\_$childs: structure$\_$block list $;$ & \cr
                      && &\textcolor{red}{mutable }s$\_$content: structure$\_$content $;$ & \cr
                      && &\textcolor{red}{mutable }s$\_$attr: structure$\_$attr list $;$ & \cr
                      && &\textcolor{red}{mutable }s$\_$line: int $;$&{$\gg\:$\it For error tracking $\ll$}\cr
                      && &\textcolor{red}{mutable }s$\_$name: string ref  $\}$ &{$\gg\:$\it For error tracking $\ll$}\cr
                      }
                        \halign{\textcolor{red}{\strut\vrule width 5pt#}&#\cr&\cr}
                        }
                          \vskip 2\baselineskip
                          A string can be converted to an atom list with the {\sl atoms$\_$of$\_$text }function, and the {\sl 
                          atoms$\_$of$\_$file }functions reads text from a file and convert it to an atom list: 
                          \vskip 2\baselineskip
                          \vbox{\hfil
                        \vbox{\hrule
                      \hbox{\vrule\qquad
                    \textcolor{red}{\rule[-.55em]{0pt}{1.7em}\sc Programming Interface}\quad\vrule}
                      \hrule}
                        }
                          \nobreak\vskip \baselineskip\nobreak
                          {
                        \nobreak\vskip .5\baselineskip 
                        \halign{
                      
                      \textcolor{red}{\strut\vrule width 5pt\hskip 20pt \hfill #} & # \hfil & # & #\hfill & \qquad # \hfill \cr
                      &&&& \cr
                      $[$ & {\it atom$\_$list }:  string list  & $]=$ & {\bf atoms$\_$of$\_$text } & \cr
                      & & & \hskip 10pt  $^\sim$text :  string  & \cr
                      }
                        \halign{
                      
                      \textcolor{red}{\strut\vrule width 5pt\hskip 20pt \hfill #} & # \hfil & # & #\hfill & \qquad # \hfill \cr
                      &&&& \cr
                      $[$ & {\it atom$\_$list }:  string list  & $]=$ & {\bf atoms$\_$of$\_$file } & \cr
                      & & & \hskip 10pt  $^\sim$fname :  string  & \cr
                      }
                        \halign{\textcolor{red}{\strut\vrule width 5pt#}&#\cr&\cr}
                        }
                          \vskip 2\baselineskip
                          The atom list can now be converted to a {\sl struture$\_$block }tree with the {\sl tree$\_$of$\_$atoms }
                          function. This function uses the {\sl parse }generic text parser function. Finally, a {\sl 
                          structure$\_$block }tree can be printed in symbolic form with the {\sl print$\_$tree }function. 
                          \vskip 2\baselineskip
                          \vbox{\hfil
                        \vbox{\hrule
                      \hbox{\vrule\qquad
                    \textcolor{red}{\rule[-.55em]{0pt}{1.7em}\sc Programming Interface}\quad\vrule}
                      \hrule}
                        }
                          \nobreak\vskip \baselineskip\nobreak
                          {
                        \nobreak\vskip .5\baselineskip 
                        \halign{
                      
                      \textcolor{red}{\strut\vrule width 5pt\hskip 20pt \hfill #} & # \hfil & # & #\hfill & \qquad # \hfill \cr
                      &&&& \cr
                      $[$ &  unit  & $]=$ & {\bf parse } & \cr
                      & & & \hskip 10pt {\it atom$\_$list }:  string list $\rightarrow$ & \cr
                      & & & \hskip 10pt {\it cur$\_$block }:  structure$\_$block  & \cr
                      }
                        \halign{
                      
                      \textcolor{red}{\strut\vrule width 5pt\hskip 20pt \hfill #} & # \hfil & # & #\hfill & \qquad # \hfill \cr
                      &&&& \cr
                      $[$ & {\it struc$\_$tree }:  structure$\_$block  & $]=$ & {\bf tree$\_$of$\_$atoms } & \cr
                      & & & \hskip 10pt  $^\sim$atoms :  string list  & \cr
                      }
                        \halign{
                      
                      \textcolor{red}{\strut\vrule width 5pt\hskip 20pt \hfill #} & # \hfil & # & #\hfill & \qquad # \hfill \cr
                      &&&& \cr
                      $[$ &  unit  & $]=$ & {\bf print$\_$tree } & \cr
                      & & & \hskip 10pt  structure$\_$block  & \cr
                      }
                        \halign{\textcolor{red}{\strut\vrule width 5pt#}&#\cr&\cr}
                        }
                          \vskip 2\baselineskip
                          Sometimes it's usefull to convert only a part of the full document. For this purpose, all back end 
                          functions have the {\sl section$\_$names }list argument to tell the backend the current section 
                          environment. The {\sl Doc$\_$core }module defines the {\sl section }structure and the {\sl 
                          section$\_$names }types. The structure is only for internal use in the backend. 
                          \vskip 2\baselineskip
                          \vbox{\hfil
                        \vbox{\hrule
                      \hbox{\vrule\qquad
                    \textcolor{red}{\rule[-.55em]{0pt}{1.7em}\sc Programming Interface}\quad\vrule}
                      \hrule}
                        }
                          \nobreak\vskip \baselineskip\nobreak
                          {
                        \nobreak\vskip .5\baselineskip 
                        \halign{
                      
                      \textcolor{red}{\strut\vrule width 5pt\hskip 20pt \hfill #} & # \hfil & # \hfil & #\hfill & \qquad # \hfill \cr
                      &&&&\cr
                      \textcolor{red}{type} & {\bf section$\_$names } $=$  & 
                      &Sec$\_$s1 of string  & \cr
                      &&  $|$ &Sec$\_$s2 of string  & \cr
                      &&  $|$ &Sec$\_$s3 of string  & \cr
                      &&  $|$ &Sec$\_$s4 of string  & \cr
                      &&  $|$ &Sec$\_$package of string  & \cr
                      &&  $|$ &Sec$\_$program of string  & \cr
                      &&  $|$ &Sec$\_$module of string  & \cr
                      &&  $|$ &Sec$\_$function of string  & \cr
                      &&  $|$ &Sec$\_$type of string  & \cr
                      &&  $|$ &Sec$\_$val of string  & \cr
                      &&  $|$ &Sec$\_$class of string  & \cr
                      &&  $|$ &Sec$\_$cint of string  & \cr
                      }
                        \halign{
                      
                      \textcolor{red}{\strut\vrule width 5pt\hskip 20pt \hfill #} & # \hfil & # \hfil & #\hfill & \qquad # \hfill \cr
                      &&&&\cr
                      \textcolor{red}{type} & {\bf section } $=$  $\{$  & 
                      &\textcolor{red}{mutable }sec$\_$parent: section $;$ & \cr
                      && &\textcolor{red}{mutable }sec$\_$childs: section list $;$ & \cr
                      && &\textcolor{red}{mutable }sec$\_$name: string $;$ & \cr
                      && &\textcolor{red}{mutable }sec$\_$type: string  $\}$  & \cr
                      }
                        \halign{\textcolor{red}{\strut\vrule width 5pt#}&#\cr&\cr}
                        }
                          \vskip 2\baselineskip
                          A short example follows to show the processing flow to translate a help text file to a {\sl 
                          structure$\_$block }tree. 
                          \vskip 2\baselineskip
                          \vbox{\hfil
                        \vbox{\hrule
                      \hbox{\vrule\qquad
                    \textcolor{blue}{\rule[-.55em]{0pt}{1.7em}\sc Programming Example}\quad\vrule}
                      \hrule}
                        }
                          \nobreak\vskip \baselineskip\nobreak
                          \halign to \textwidth{
                        \textcolor[rgb]{.58,.76,1.0}{\strut\vrule width 5pt#}\tabskip=0pt plus 1fil& 
                          #\hfil& #&\hfil#\hfil& 
                          \tabskip=0pt#\cr
                         &&&\cr 
                        & {\tt \hskip 1emlet al $=$ atoms$\_$of$\_$file "mldoc.man" ;; } &&& \cr
                        & {\tt \hskip 1em let st $=$ tree$\_$of$\_$atoms al ;; } &&& \cr
                        & {\tt \hskip 1em html$\_$of$\_$tree ds [HTML$\_$single$\_$doc "mldoc$\_$man.html"] [];; } &&& \cr
                        & {\tt \hskip 1em tex$\_$of$\_$tree ds [TEX$\_$doc$\_$name "mldoc.tex"; } &&& \cr
                        & {\tt \hskip 1em \hskip 1em \hskip 1em TEX$\_$color; } &&& \cr
                        & {\tt \hskip 1em \hskip 1em \hskip 1em TEX$\_$link$\_$ref; } &&& \cr
                        & {\tt \hskip 1em \hskip 1em ] []; } &&& \cr
                        & {\tt \hskip 1em text$\_$of$\_$tree ds [TEXT$\_$doc$\_$name "mldoc.txt"] []; } &&& \cr
                        & {\tt \hskip 1em \hfil} &&& \cr 
                        }
                          
                            \vskip 2\baselineskip
                            \vfill \eject
                            \vbox{\hsize \textwidth
                          \hrule
                          \vbox{\rule[-.55em]{0pt}{1.7em}\strut \vrule \quad 
                        \label{doc$_$html}\textcolor{blue}{\sc ML-Module: {\sl doc$\_$html }} \hfill 
                        \textcolor[rgb]{.0,.7,.0}{\sc Package:  mldoc}\quad \vrule}\hrule}
                          
                            \vskip \baselineskip
                            \vskip .5\baselineskip 
                            {\parindent 0pt \vbox{\sc \label{Description}Description }
                            \leftskip 20pt \rightskip 20pt \vskip .5\baselineskip\begin{list}{}{}\item{\parindent 0pt 
                          This is the HTML backend module used to convert {\sl MlDoC }documents in HTML$-$4 (transitional) format. 
                          This {\sl html$\_$of$\_$tree }function is used to convert the {\sl structure$\_$block }graph to HTML 
                          output. }\end{list}}
                            \vskip .5\baselineskip
                            \vskip 2\baselineskip
                            \vbox{\hfil
                          \vbox{\hrule
                        \hbox{\vrule\qquad
                      \textcolor{red}{\rule[-.55em]{0pt}{1.7em}\sc Programming Interface}\quad\vrule}
                        \hrule}
                          }
                            \nobreak\vskip \baselineskip\nobreak
                            {
                          \nobreak\vskip .5\baselineskip 
                          \halign{
                        
                        \textcolor{red}{\strut\vrule width 5pt\hskip 20pt \hfill #} & # \hfil & # & #\hfill & \qquad # \hfill \cr
                        &&&& \cr
                        $[$ &  unit  & $]=$ & {\bf html$\_$of$\_$tree } & \cr
                        & & & \hskip 10pt  $^\sim$ds :  structure$\_$block $\rightarrow$ & \cr
                        & & & \hskip 10pt  $^\sim$options :  html$\_$options list $\rightarrow$ & \cr
                        & & & \hskip 10pt  $^\sim$sections :  section$\_$names list  & \cr
                        }
                          \halign{\textcolor{red}{\strut\vrule width 5pt#}&#\cr&\cr}
                          }
                            \vskip 2\baselineskip
                            The {\sl html$\_$of$\_$tree }function needs three arguments: 
                            \vskip 2\baselineskip
                            \vbox{\hfil
                          \vbox{\hrule
                        \hbox{\vrule\qquad
                      {\rule[-.55em]{0pt}{1.7em}\sc  Arguments }\qquad\vrule}
                        \hrule}
                          }
                            \nobreak\vskip \baselineskip\nobreak
                            \begin{list}{}{}\item{\parindent 0pt 
                          \begin{description}
                        \item[ ds]
                      {\hfill\\}
                      This is the structure block tree previously generated with the {\sl atoms$\_$of$\_$ }function family. 
                        \item[ options]
                      {\hfill\\}
                      Several options can control the converting behaviour and the output. Available options: 
                      \vskip 2\baselineskip
                      \begin{center}
                    \begin{tabular}{|p{6.6cm}|p{6.6cm}|}\hline
                  {\tt HTML$\_$single$\_$doc of string }&Only create a single HTML file instead of a collection of 
                  files, each for one section. The string is the file name. \\\hline
                  \end{tabular}
                    \end{center}
                      \vskip 2\baselineskip
                      
                        \item[ sections]
                      {\hfill\\}
                      To convert only a part of a manual, for example a module like this, some informations are needed 
                      about sections above this section. 
                        \end{description}
                          }\end{list}
                            \vskip .5\baselineskip
                            \vskip \baselineskip
                            \vskip .5\baselineskip 
                            {\parindent 0pt \vbox{\sc \label{Moduledependencies:}Module dependencies: }
                            \leftskip 20pt \rightskip 20pt \vskip .5\baselineskip\begin{list}{}{}\item{\parindent 0pt 
                          \vskip \baselineskip
                          \begin{minipage}{.9\textwidth}\vbox{
                        \begin{itemize}
                      \item {\sl Doc$\_$core }
                      \end{itemize}
                        }\end{minipage}
                          \vskip \baselineskip
                          }\end{list}}
                            \vskip .5\baselineskip
                            \vfill \eject
                            \vbox{\hsize \textwidth
                          \hrule
                          \vbox{\rule[-.55em]{0pt}{1.7em}\strut \vrule \quad 
                        \label{doc$_$latex}\textcolor{blue}{\sc ML-Module: {\sl doc$\_$latex }} \hfill 
                        \textcolor[rgb]{.0,.7,.0}{\sc Package:  mldoc}\quad \vrule}\hrule}
                          
                            \vskip \baselineskip
                            \vskip .5\baselineskip 
                            {\parindent 0pt \vbox{\sc \label{Description}Description }
                            \leftskip 20pt \rightskip 20pt \vskip .5\baselineskip\begin{list}{}{}\item{\parindent 0pt 
                          This is the Tex backend module used to convert {\sl MlDoC }documents in Plain$-$TeX/Latex format. This 
                          {\sl tex$\_$of$\_$tree }function is used to convert the {\sl structure$\_$block }graph to TeX output. 
                          }\end{list}}
                            \vskip .5\baselineskip
                            \vskip 2\baselineskip
                            \vbox{\hfil
                          \vbox{\hrule
                        \hbox{\vrule\qquad
                      \textcolor{red}{\rule[-.55em]{0pt}{1.7em}\sc Programming Interface}\quad\vrule}
                        \hrule}
                          }
                            \nobreak\vskip \baselineskip\nobreak
                            {
                          \nobreak\vskip .5\baselineskip 
                          \halign{
                        
                        \textcolor{red}{\strut\vrule width 5pt\hskip 20pt \hfill #} & # \hfil & # & #\hfill & \qquad # \hfill \cr
                        &&&& \cr
                        $[$ &  unit  & $]=$ & {\bf tex$\_$of$\_$tree } & \cr
                        & & & \hskip 10pt  $^\sim$ds :  structure$\_$block $\rightarrow$ & \cr
                        & & & \hskip 10pt  $^\sim$options :  tex$\_$options list $\rightarrow$ & \cr
                        & & & \hskip 10pt  $^\sim$sections :  section$\_$names list  & \cr
                        }
                          \halign{\textcolor{red}{\strut\vrule width 5pt#}&#\cr&\cr}
                          }
                            \vskip 2\baselineskip
                            The {\sl tex$\_$of$\_$tree }function needs three arguments: 
                            \vskip 2\baselineskip
                            \vbox{\hfil
                          \vbox{\hrule
                        \hbox{\vrule\qquad
                      {\rule[-.55em]{0pt}{1.7em}\sc  Arguments }\qquad\vrule}
                        \hrule}
                          }
                            \nobreak\vskip \baselineskip\nobreak
                            \begin{list}{}{}\item{\parindent 0pt 
                          \begin{description}
                        \item[ ds]
                      {\hfill\\}
                      This is the structure block tree previously generated with the {\sl atoms$\_$of$\_$ }function family. 
                        \item[ options]
                      {\hfill\\}
                      Several options can control the converting behaviour and the output. Available options: 
                      \vskip 2\baselineskip
                      \begin{center}
                    \begin{tabular}{|p{6.6cm}|p{6.6cm}|}\hline
                  {\tt TEX$\_$doc$\_$name of string }&The name of the output file. The default setting is {\tt 
                  manual.tex }. \\\hline
                  {\tt TEX$\_$head$\_$inline }&Don't perform a page break on section boundaries. \\\hline
                  {\tt TEX$\_$color }&Ouput colored manual pages. Special color commands for {\sl dvips }are 
                  created. Default is color off. \\\hline
                  {\tt TEX$\_$no$\_$toc }&Don't print a table of content. Default is print a TOC. \\\hline
                  {\tt TEX$\_$link$\_$ref }&Generate links in the manual (page references). Default is no links. 
                  \\\hline
                  \end{tabular}
                    \end{center}
                      \vskip 2\baselineskip
                      
                        \item[ sections]
                      {\hfill\\}
                      To convert only a part of a manual, for example a module like this, some informations are needed 
                      about sections above this section. 
                        \end{description}
                          }\end{list}
                            \vskip .5\baselineskip
                            \vskip \baselineskip
                            \vskip .5\baselineskip 
                            {\parindent 0pt \vbox{\sc \label{LaTeXpackagedependencies}LaTeX package dependencies }
                            \leftskip 20pt \rightskip 20pt \vskip .5\baselineskip\begin{list}{}{}\item{\parindent 0pt 
                          The following packages are used by the {\sl Doc$\_$latex }module: 
                          \vskip \baselineskip
                          \begin{minipage}{.9\textwidth}\vbox{
                        \begin{itemize}
                      \item {\tt newcent }
                      \item {\tt pifont }
                      \item {\tt color[dvips] }
                      \end{itemize}
                        }\end{minipage}
                          \vskip \baselineskip
                          Most common LaTeX/TeX distributions are shipped with these packages. }\end{list}}
                            \vskip .5\baselineskip
                            \vskip .5\baselineskip 
                            {\parindent 0pt \vbox{\sc \label{Moduledependencies:}Module dependencies: }
                            \leftskip 20pt \rightskip 20pt \vskip .5\baselineskip\begin{list}{}{}\item{\parindent 0pt 
                          \vskip \baselineskip
                          \begin{minipage}{.9\textwidth}\vbox{
                        \begin{itemize}
                      \item {\sl Doc$\_$core }
                      \end{itemize}
                        }\end{minipage}
                          \vskip \baselineskip
                          }\end{list}}
                            \vskip .5\baselineskip
                            \vfill \eject
                            \vbox{\hsize \textwidth
                          \hrule
                          \vbox{\rule[-.55em]{0pt}{1.7em}\strut \vrule \quad 
                        \label{doc$_$text}\textcolor{blue}{\sc ML-Module: {\sl doc$\_$text }} \hfill 
                        \textcolor[rgb]{.0,.7,.0}{\sc Package:  mldoc}\quad \vrule}\hrule}
                          
                            \vskip \baselineskip
                            \vskip .5\baselineskip 
                            {\parindent 0pt \vbox{\sc \label{Description}Description }
                            \leftskip 20pt \rightskip 20pt \vskip .5\baselineskip\begin{list}{}{}\item{\parindent 0pt 
                          This is the ASCII$-$Text backend module used to convert {\sl MlDoC }documents in generic Text format. 
                          This {\sl text$\_$of$\_$tree }function is used to convert the {\sl structure$\_$block }graph to text 
                          output. }\end{list}}
                            \vskip .5\baselineskip
                            \vskip 2\baselineskip
                            \vbox{\hfil
                          \vbox{\hrule
                        \hbox{\vrule\qquad
                      \textcolor{red}{\rule[-.55em]{0pt}{1.7em}\sc Programming Interface}\quad\vrule}
                        \hrule}
                          }
                            \nobreak\vskip \baselineskip\nobreak
                            {
                          \nobreak\vskip .5\baselineskip 
                          \halign{
                        
                        \textcolor{red}{\strut\vrule width 5pt\hskip 20pt \hfill #} & # \hfil & # & #\hfill & \qquad # \hfill \cr
                        &&&& \cr
                        $[$ &  unit  & $]=$ & {\bf text$\_$of$\_$tree } & \cr
                        & & & \hskip 10pt  $^\sim$ds :  structure$\_$block $\rightarrow$ & \cr
                        & & & \hskip 10pt  $^\sim$options :  text$\_$options list $\rightarrow$ & \cr
                        & & & \hskip 10pt  $^\sim$sections :  section$\_$names list  & \cr
                        }
                          \halign{\textcolor{red}{\strut\vrule width 5pt#}&#\cr&\cr}
                          }
                            \vskip 2\baselineskip
                            The {\sl text$\_$of$\_$tree }function needs three arguments: 
                            \vskip 2\baselineskip
                            \vbox{\hfil
                          \vbox{\hrule
                        \hbox{\vrule\qquad
                      {\rule[-.55em]{0pt}{1.7em}\sc  Arguments }\qquad\vrule}
                        \hrule}
                          }
                            \nobreak\vskip \baselineskip\nobreak
                            \begin{list}{}{}\item{\parindent 0pt 
                          \begin{description}
                        \item[ ds]
                      {\hfill\\}
                      This is the structure block tree previously generated with the {\sl atoms$\_$of$\_$ }function family. 
                        \item[ options]
                      {\hfill\\}
                      Several options can control the converting behaviour and the output. Available options: 
                      \vskip 2\baselineskip
                      \begin{center}
                    \begin{tabular}{|p{6.6cm}|p{6.6cm}|}\hline
                  {\tt TEXT$\_$doc$\_$name of string }&The output file name if any. Without this option, the output 
                  is printied to the current standard out channel. \\\hline
                  {\tt TEXT$\_$terminal }&Enhanced text ouput with special terminal control command 
                  (underlined...). Default is no special commands. \\\hline
                  {\tt TEXT$\_$notoc }&Don't print a table of content. \\\hline
                  \end{tabular}
                    \end{center}
                      \vskip 2\baselineskip
                      
                        \item[ sections]
                      {\hfill\\}
                      To convert only a part of a manual, for example a module like this, some informations are needed 
                      about sections above this section. 
                        \end{description}
                          }\end{list}
                            \vskip .5\baselineskip
                            \vskip \baselineskip
                            \vskip .5\baselineskip 
                            {\parindent 0pt \vbox{\sc \label{Moduledependencies:}Module dependencies: }
                            \leftskip 20pt \rightskip 20pt \vskip .5\baselineskip\begin{list}{}{}\item{\parindent 0pt 
                          \vskip \baselineskip
                          \begin{minipage}{.9\textwidth}\vbox{
                        \begin{itemize}
                      \item {\tt Terminfo }(OCaML) 
                      \item {\tt Doc$\_$core }
                      \end{itemize}
                        }\end{minipage}
                          \vskip \baselineskip
                          }\end{list}}
                            \vskip .5\baselineskip
                            \vfill \eject
                            \vbox{\hsize \textwidth
                          \hrule
                          \vbox{\rule[-.55em]{0pt}{1.7em}\strut \vrule \quad 
                        \label{doc$_$help}\textcolor{blue}{\sc ML-Module: {\sl doc$\_$help }} \hfill 
                        \textcolor[rgb]{.0,.7,.0}{\sc Package:  mldoc}\quad \vrule}\hrule}
                          
                            \vskip \baselineskip
                            \vskip .5\baselineskip 
                            {\parindent 0pt \vbox{\sc \label{Description}Description }
                            \leftskip 20pt \rightskip 20pt \vskip .5\baselineskip\begin{list}{}{}\item{\parindent 0pt 
                          This is the {\sl Help }frontend. It can be used to handle {\sl MlDoc }documents with various backend 
                          modules. First, a document must be loaded. Either, you use the {\sl atoms$\_$of$\_$file }and the {\sl 
                          tree$\_$of$\_$atoms }functions from {\sl Doc$\_$core }module, finally loaded with {\sl help$\_$load }
                          function, or you use the {\sl help$\_$file }function to load a document. }\end{list}}
                            \vskip .5\baselineskip
                            \vskip 2\baselineskip
                            \vbox{\hfil
                          \vbox{\hrule
                        \hbox{\vrule\qquad
                      \textcolor{red}{\rule[-.55em]{0pt}{1.7em}\sc Programming Interface}\quad\vrule}
                        \hrule}
                          }
                            \nobreak\vskip \baselineskip\nobreak
                            {
                          \nobreak\vskip .5\baselineskip 
                          \halign{
                        
                        \textcolor{red}{\strut\vrule width 5pt\hskip 20pt \hfill #} & # \hfil & # \hfil & #\hfill & \qquad # \hfill \cr
                        &&&&\cr
                        \textcolor{red}{type} & {\bf h$\_$sec } $=$  $\{$  & 
                        &\textcolor{red}{mutable }h$\_$sec$\_$keywords: string list $;$ & \cr
                        && &\textcolor{red}{mutable }h$\_$sec$\_$ds: structure$\_$block ref $;$ & \cr
                        && &\textcolor{red}{mutable }h$\_$sec$\_$env: section$\_$names list $;$ & \cr
                        && &\textcolor{red}{mutable }h$\_$type: string  $\}$  & \cr
                        }
                          \halign{
                        
                        \textcolor{red}{\strut\vrule width 5pt\hskip 20pt \hfill #} & # \hfil & # \hfil & #\hfill & \qquad # \hfill \cr
                        &&&&\cr
                        \textcolor{red}{type} & {\bf s$\_$help } $=$  $\{$  & 
                        &\textcolor{red}{mutable }h$\_$sections: h$\_$sec list $;$ & \cr
                        && &\textcolor{red}{mutable }h$\_$subsections: h$\_$sec list $;$ & \cr
                        && &\textcolor{red}{mutable }h$\_$units: h$\_$sec list $;$ & \cr
                        && &\textcolor{red}{mutable }h$\_$main: structure$\_$block  $\}$  & \cr
                        }
                          \halign{
                        
                        \textcolor{red}{\strut\vrule width 5pt\hskip 20pt \hfill #} & # \hfil & # & #\hfill & \qquad # \hfill \cr
                        &&&& \cr
                        $[$ & {\it help }:  s$\_$help  & $]=$ & {\bf help$\_$of$\_$doc } & \cr
                        & & & \hskip 10pt  $^\sim$ds :  structure$\_$block  & \cr
                        }
                          \halign{
                        
                        \textcolor{red}{\strut\vrule width 5pt\hskip 20pt \hfill #} & # \hfil & # & #\hfill & \qquad # \hfill \cr
                        &&&& \cr
                        $[$ &  unit  & $]=$ & {\bf help$\_$load } & \cr
                        & & & \hskip 10pt  $^\sim$ds :  structure$\_$block  & \cr
                        }
                          \halign{
                        
                        \textcolor{red}{\strut\vrule width 5pt\hskip 20pt \hfill #} & # \hfil & # & #\hfill & \qquad # \hfill \cr
                        &&&& \cr
                        $[$ &  unit  & $]=$ & {\bf help$\_$file } & \cr
                        & & & \hskip 10pt  $^\sim$fname :  string  & \cr
                        }
                          \halign{
                        
                        \textcolor{red}{\strut\vrule width 5pt\hskip 20pt \hfill #} & # \hfil & # \hfil & #\hfill & \qquad # \hfill \cr
                        &&&&\cr
                        \textcolor{red}{type} & {\bf help$\_$Device } $=$  & 
                        &Help$\_$TTY &{$\gg\:$\it Print to stdout $\ll$}\cr
                        &&  $|$ &Help$\_$ASCII &{$\gg\:$\it Print to file: ASCII text $\ll$}\cr
                        &&  $|$ &Help$\_$HTML &{$\gg\:$\it Print to file: HTML $\ll$}\cr
                        &&  $|$ &Help$\_$LATEX &{$\gg\:$\it Print to file: Latex $\ll$}\cr
                        }
                          \halign{
                        
                        \textcolor{red}{\strut\vrule width 5pt\hskip 20pt \hfill #} & # \hfil & # & #\hfill & \qquad # \hfill \cr
                        &&&& \cr
                        $[$ &  unit  & $]=$ & {\bf help$\_$dev } & \cr
                        & & & \hskip 10pt  $^\sim$dev :  help$\_$device  & \cr
                        }
                          \halign{
                        
                        \textcolor{red}{\strut\vrule width 5pt\hskip 20pt \hfill #} & # \hfil & # & #\hfill & \qquad # \hfill \cr
                        &&&& \cr
                        $[$ &  unit  & $]=$ & {\bf help } & \cr
                        & & & \hskip 10pt  $^\sim$name :  string  & \cr
                        }
                          \halign{\textcolor{red}{\strut\vrule width 5pt#}&#\cr&\cr}
                          }
                            \vskip 2\baselineskip
                            The {\sl help$\_$dev }function can be used to change the output device. Default is the {\sl Help$\_$TTY }
                            device. The {\sl help }function uses the keyword or name string to search the help database. The search 
                            results will be shown. 
                            \vskip 2\baselineskip
                            \vbox{\hfil
                          \vbox{\hrule
                        \hbox{\vrule\qquad
                      \textcolor{blue}{\rule[-.55em]{0pt}{1.7em}\sc Programming Example}\quad\vrule}
                        \hrule}
                          }
                            \nobreak\vskip \baselineskip\nobreak
                            \halign to \textwidth{
                          \textcolor[rgb]{.58,.76,1.0}{\strut\vrule width 5pt#}\tabskip=0pt plus 1fil& 
                            #\hfil& #&\hfil#\hfil& 
                            \tabskip=0pt#\cr
                           &&&\cr 
                          & {\tt \hskip 1emlet at $=$ atoms$\_$of$\_$file "mldoc.man" in } &&& \cr
                          & {\tt \hskip 1em let ds $=$ tree$\_$of$\_$atoms at in } &&& \cr
                          & {\tt \hskip 1em help$\_$load ds; } &&& \cr
                          & {\tt \hskip 1em \hfil} &&& \cr 
                          }
                            
                              \vskip 2\baselineskip
                              \vskip .5\baselineskip 
                              {\parindent 0pt \vbox{\sc \label{Moduledependencies}Module dependencies }
                              \leftskip 20pt \rightskip 20pt \vskip .5\baselineskip\begin{list}{}{}\item{\parindent 0pt 
                            \vskip \baselineskip
                            \begin{minipage}{.9\textwidth}\vbox{
                          \begin{itemize}
                        \item {\sl Doc$\_$core }
                        \item {\sl Doc$\_$html }
                        \item {\sl Doc$\_$latex }
                        \item {\sl Doc$\_$text }
                        \end{itemize}
                          }\end{minipage}
                            \vskip \baselineskip
                            }\end{list}}
                              \vskip .5\baselineskip
                              \vfill \eject
                              \vbox{\hsize \textwidth
                            \hrule
                            \vbox{\rule[-.55em]{0pt}{1.7em}\strut \vrule \quad \hfill
                          \textcolor{blue}{\sc Table of Content } \hfill
                          \quad \vrule}\hrule}
                            
                              \vskip \baselineskip
                              \vbox{\sc Package:  VAM Amoeba System\quad\dotfill\quad\pageref{VAMAmoebaSystem}}
                              \vskip .2\baselineskip
                              \vbox{\leftskip20pt \vbox{\sc ML-Module:  Amoeba\quad\dotfill\quad\pageref{Amoeba}}
                              \vskip .2\baselineskip
                              \vbox{\leftskip40pt \vbox{\sc  Basic types\quad\dotfill\quad\pageref{Basictypes}}
                              \vbox{\sc  Basic functions and values\quad\dotfill\quad\pageref{Basicfunctionsandvalues}}
                              \vbox{\sc  Encryption and Rights\quad\dotfill\quad\pageref{EncryptionandRights}}
                              }
                              \vskip .2\baselineskip
                              \vbox{\sc ML-Module:  Ar\quad\dotfill\quad\pageref{Ar}}
                              \vskip .2\baselineskip
                              \vbox{\leftskip40pt \vbox{\sc  Amoeba to string\quad\dotfill\quad\pageref{Amoebatostring}}
                              \vbox{\sc  String to Amoeba\quad\dotfill\quad\pageref{StringtoAmoeba}}
                              \vbox{\sc  Module Dependencies\quad\dotfill\quad\pageref{ModuleDependencies}}
                              }
                              \vskip .2\baselineskip
                              \vbox{\sc ML-Module:  Buf\quad\dotfill\quad\pageref{Buf}}
                              \vskip .2\baselineskip
                              \vbox{\leftskip40pt \vbox{\sc  Put functions\quad\dotfill\quad\pageref{Putfunctions}}
                              \vbox{\sc  Get functions\quad\dotfill\quad\pageref{Getfunctions}}
                              \vbox{\sc  Module Dependencies\quad\dotfill\quad\pageref{ModuleDependencies}}
                              }
                              \vskip .2\baselineskip
                              \vbox{\sc ML-Module:  Cache\quad\dotfill\quad\pageref{Cache}}
                              \vbox{\sc ML-Module:  Cap$\_$env\quad\dotfill\quad\pageref{Cap$_$env}}
                              \vskip .2\baselineskip
                              \vbox{\leftskip40pt \vbox{\sc  Module Dependencies\quad\dotfill\quad\pageref{ModuleDependencies}}
                              }
                              \vskip .2\baselineskip
                              \vbox{\sc ML-Module:  Capset\quad\dotfill\quad\pageref{Capset}}
                              \vskip .2\baselineskip
                              \vbox{\leftskip40pt \vbox{\sc  Structures\quad\dotfill\quad\pageref{Structures}}
                              \vbox{\sc  Functions\quad\dotfill\quad\pageref{Functions}}
                              \vbox{\sc  Module Dependencies\quad\dotfill\quad\pageref{ModuleDependencies}}
                              }
                              \vskip .2\baselineskip
                              \vbox{\sc ML-Module:  Circbuf\quad\dotfill\quad\pageref{Circbuf}}
                              \vskip .2\baselineskip
                              \vbox{\leftskip40pt \vbox{\sc  Structures and types\quad\dotfill\quad\pageref{Structuresandtypes}}
                              
                              \vbox{\sc  Generic Functions to create and close circular buffers\quad\dotfill\quad\pageref{GenericFunctionstocreateandclosecircularbuffers}}
                              
                              \vbox{\sc  Functions to get the state of a circular buffer\quad\dotfill\quad\pageref{Functionstogetthestateofacircularbuffer}}
                              
                              \vbox{\sc  Functions for storing and extracting data\quad\dotfill\quad\pageref{Functionsforstoringandextractingdata}}
                              \vbox{\sc  Internal functions\quad\dotfill\quad\pageref{Internalfunctions}}
                              \vbox{\sc  Module Dependencies\quad\dotfill\quad\pageref{ModuleDependencies}}
                              }
                              \vskip .2\baselineskip
                              \vbox{\sc ML-Module:  Cmdreg\quad\dotfill\quad\pageref{Cmdreg}}
                              \vskip .2\baselineskip
                              \vbox{\leftskip40pt \vbox{\sc  The values\quad\dotfill\quad\pageref{Thevalues}}
                              \vbox{\sc  Module Dependencies\quad\dotfill\quad\pageref{ModuleDependencies}}
                              }
                              \vskip .2\baselineskip
                              \vbox{\sc ML-Module:  Dblist\quad\dotfill\quad\pageref{Dblist}}
                              \vskip .2\baselineskip
                              \vbox{\leftskip40pt \vbox{\sc  Types\quad\dotfill\quad\pageref{Types}}
                              \vbox{\sc  Functions\quad\dotfill\quad\pageref{Functions}}
                              }
                              \vskip .2\baselineskip
                              \vbox{\sc  DNS: Directory and Name Service\quad\dotfill\quad\pageref{DNS:DirectoryandNameService}}
                              \vskip .2\baselineskip
                              \vbox{\leftskip40pt \vbox{\sc  Common: Interface requests\quad\dotfill\quad\pageref{Common:Interfacerequests}}
                              }
                              \vskip .2\baselineskip
                              \vbox{\sc ML-Module:  Dns$\_$client\quad\dotfill\quad\pageref{Dns$_$client}}
                              \vskip .2\baselineskip
                              \vbox{\leftskip40pt \vbox{\sc  dns$\_$LOOKUP\quad\dotfill\quad\pageref{dns$_$LOOKUP}}
                              }
                              \vskip .2\baselineskip
                              \vbox{\sc ML-Module:  Dns$\_$server\quad\dotfill\quad\pageref{Dns$_$server}}
                              \vskip .2\baselineskip
                              \vbox{\leftskip40pt \vbox{\sc  Basic Structures\quad\dotfill\quad\pageref{BasicStructures}}
                              \vbox{\sc  Values\quad\dotfill\quad\pageref{Values}}
                              \vbox{\sc  Directory table management\quad\dotfill\quad\pageref{Directorytablemanagement}}
                              \vbox{\sc  Acquire and release a directory\quad\dotfill\quad\pageref{Acquireandreleaseadirectory}}
                              \vbox{\sc  Directory restriction\quad\dotfill\quad\pageref{Directoryrestriction}}
                              \vbox{\sc  Client request handlers\quad\dotfill\quad\pageref{Clientrequesthandlers}}
                              \vbox{\sc  Module dependencies\quad\dotfill\quad\pageref{Moduledependencies}}
                              }
                              \vskip .2\baselineskip
                              \vbox{\sc  AFS: Atomic Filesystem Service\quad\dotfill\quad\pageref{AFS:AtomicFilesystemService}}
                              \vbox{\sc ML-Module:  Afs$\_$common\quad\dotfill\quad\pageref{Afs$_$common}}
                              \vskip .2\baselineskip
                              \vbox{\leftskip40pt \vbox{\sc  AFS requests\quad\dotfill\quad\pageref{AFSrequests}}
                              \vbox{\sc  AFS rights\quad\dotfill\quad\pageref{AFSrights}}
                              \vbox{\sc  AFS cache flags\quad\dotfill\quad\pageref{AFScacheflags}}
                              \vbox{\sc  Misc.\quad\dotfill\quad\pageref{Misc.}}
                              \vbox{\sc  Module dependencies\quad\dotfill\quad\pageref{Moduledependencies}}
                              }
                              \vskip .2\baselineskip
                              \vbox{\sc ML-Module:  Afs$\_$client\quad\dotfill\quad\pageref{Afs$_$client}}
                              \vskip .2\baselineskip
                              \vbox{\leftskip40pt \vbox{\sc  Request\quad\dotfill\quad\pageref{Request}}
                              }
                              \vskip .2\baselineskip
                              }
                              \vskip .2\baselineskip
                              \vbox{\sc  MlDoC: ML$-$Documentation$-$System\quad\dotfill\quad\pageref{MlDoC:ML$$Documentation$$System}}
                              \vbox{\sc  MlDoc dot commands\quad\dotfill\quad\pageref{MlDocdotcommands}}
                              \vskip .2\baselineskip
                              \vbox{\leftskip20pt \vbox{\sc  Section headers\quad\dotfill\quad\pageref{Sectionheaders}}
                              \vbox{\sc  Subsection headers\quad\dotfill\quad\pageref{Subsectionheaders}}
                              \vbox{\sc  Unit headers\quad\dotfill\quad\pageref{Unitheaders}}
                              \vbox{\sc  Subunits\quad\dotfill\quad\pageref{Subunits}}
                              \vbox{\sc  Special Interfaces\quad\dotfill\quad\pageref{SpecialInterfaces}}
                              \vbox{\sc  Paragraph elements\quad\dotfill\quad\pageref{Paragraphelements}}
                              \vbox{\sc  Text elements\quad\dotfill\quad\pageref{Textelements}}
                              \vbox{\sc  Summary of dot commands\quad\dotfill\quad\pageref{Summaryofdotcommands}}
                              \vskip .2\baselineskip
                              \vbox{\leftskip40pt 
                              \vbox{\sc  Text styles and special text\quad\dotfill\quad\pageref{Textstylesandspecialtext}}
                              \vbox{\sc  Special dot commands\quad\dotfill\quad\pageref{Specialdotcommands}}
                              \vbox{\sc  List commands\quad\dotfill\quad\pageref{Listcommands}}
                              \vbox{\sc  Table commands\quad\dotfill\quad\pageref{Tablecommands}}
                              \vbox{\sc  Special Interfaces commands\quad\dotfill\quad\pageref{SpecialInterfacescommands}}
                              \vbox{\sc  Section commands\quad\dotfill\quad\pageref{Sectioncommands}}
                              \vbox{\sc  Subsection commands\quad\dotfill\quad\pageref{Subsectioncommands}}
                              
                              \vbox{\sc  Units of a Subsection (Manual Page) commands\quad\dotfill\quad\pageref{UnitsofaSubsectionManualPagecommands}}
                              \vbox{\sc  Subunit commands\quad\dotfill\quad\pageref{Subunitcommands}}
                              
                              \vbox{\sc  Generic Section, Subsection, Unit, Subunit commands\quad\dotfill\quad\pageref{GenericSectionSubsectionUnitSubunitcommands}}
                              }
                              \vskip .2\baselineskip
                              
                              \vbox{\sc  Compilation and usage of the MlDoC package\quad\dotfill\quad\pageref{CompilationandusageoftheMlDoCpackage}}
                              }
                              \vskip .2\baselineskip
                              \vbox{\sc Package:  mldoc\quad\dotfill\quad\pageref{mldoc}}
                              \vskip .2\baselineskip
                              \vbox{\leftskip20pt \vbox{\sc ML-Module:  doc$\_$core\quad\dotfill\quad\pageref{doc$_$core}}
                              \vskip .2\baselineskip
                              \vbox{\leftskip40pt \vbox{\sc  Description\quad\dotfill\quad\pageref{Description}}
                              }
                              \vskip .2\baselineskip
                              \vbox{\sc ML-Module:  doc$\_$html\quad\dotfill\quad\pageref{doc$_$html}}
                              \vskip .2\baselineskip
                              \vbox{\leftskip40pt \vbox{\sc  Description\quad\dotfill\quad\pageref{Description}}
                              \vbox{\sc  Module dependencies:\quad\dotfill\quad\pageref{Moduledependencies:}}
                              }
                              \vskip .2\baselineskip
                              \vbox{\sc ML-Module:  doc$\_$latex\quad\dotfill\quad\pageref{doc$_$latex}}
                              \vskip .2\baselineskip
                              \vbox{\leftskip40pt \vbox{\sc  Description\quad\dotfill\quad\pageref{Description}}
                              \vbox{\sc  LaTeX package dependencies\quad\dotfill\quad\pageref{LaTeXpackagedependencies}}
                              \vbox{\sc  Module dependencies:\quad\dotfill\quad\pageref{Moduledependencies:}}
                              }
                              \vskip .2\baselineskip
                              \vbox{\sc ML-Module:  doc$\_$text\quad\dotfill\quad\pageref{doc$_$text}}
                              \vskip .2\baselineskip
                              \vbox{\leftskip40pt \vbox{\sc  Description\quad\dotfill\quad\pageref{Description}}
                              \vbox{\sc  Module dependencies:\quad\dotfill\quad\pageref{Moduledependencies:}}
                              }
                              \vskip .2\baselineskip
                              \vbox{\sc ML-Module:  doc$\_$help\quad\dotfill\quad\pageref{doc$_$help}}
                              \vskip .2\baselineskip
                              \vbox{\leftskip40pt \vbox{\sc  Description\quad\dotfill\quad\pageref{Description}}
                              \vbox{\sc  Module dependencies\quad\dotfill\quad\pageref{Moduledependencies}}
                              }
                              \vskip .2\baselineskip
                              }
                              \vskip .2\baselineskip
                              
                                \end{document}
                                